% Sample apostrophy's to remove team's 





\chapter{Introduction}
\label{IntroductionChapter}


Imagine that you are a Vice President of a multi-million dollar software development business unit managing 100 software engineers and 20 product managers. You have 25 teams working on various interrelated software products. The entire organization balances 1) adding new features to the next release, 2) solving customer escalations from installed products, and 3) handling maintenance chores such as routinely upgrading dependencies. Your predecessor accrued significant tech debt. It will take years to pay it down. There is never enough time to get it all done. 


In the middle of all this, one team is challenging to manage, Team Competent. The team is composed of highly skilled software engineers. They write well tested, clean code. The test suite is slow, but is thorough and catches issues before they get to the customer. In fact, there has never been a serious customer escalation.


The team resists collaborating with others. The team developed an \quotes{us versus them} or \quotes{us versus management} attitude. The rest of your organization does not want to work with them. Their product manager would like the team to be in the office more to increase the product manager’s collaboration with the team. While the organization expects teams to be in the office during core business hours, this team believes that as long as the work gets done, why should anyone care where or when they accomplish the work. The product manager is growing frustrated with the team’s attitude about features. The product manager would like to discuss how the feature will work, yet the engineers want just to own the problem and deliver something that works. Sometimes when they deliver code, the implementation does not solve the customer’s needs or works the way the customer expects. When asked, \quotes{how does the customer use the flux capacitor?} the engineers respond, \quotes{we do not know what the customers want.} Recently, the team’s productivity has decreased. The team all came together from a previous company, and you are worried that when one of them leaves, all of them will exit the company.


How did software development come to this? Why is software development so hard? Since software development is a socio-technical activity, why is collaboration so difficult? How can we build teams that embrace collaboration, help others solve the problems, and train others to do their work?


The complexities of software development emerge, in part, due to 1) uncertainty about product to market fit resulting in changing features, 2) constantly changing dependent technologies, 3) the baggage of maintaining legacy code, 4) the lack of physical constraints resulting in boundless solutions, 5) the absence of visible progress, and 6) people write software. Teams require coordination and collaboration. Building high performing teams requires effort and care. Unlike assembly line work, engineers are not fungible. Even if two engineers have the same competency in a set of technologies, their individual temperaments and personalities affect team dynamics. Building software is challenging. 


While it appears that any set of processes will work for tiny teams in the short term, creating an engineering culture that will deliver value week after week, month after month regardless of team disruption and changing circumstances is no easy task. Any company that achieves proficiency in software development becomes an interesting research opportunity. We choose Pivotal because 1) it is successful; 2) it is unusual in its continued use and evolution of Extreme Programming \cite{BeckExtremeProgramming2004}; 3) it is accessible and cooperative with research. Chapter \ref{ResearchContextChapter} describes Pivotal, our research context. 


In order to understand how Pivotal develops software, we employed Constructivist Grounded Theory as the research method. Chapter \ref{ConstructivistGroundedTheoryChapter} describes Constructivist Grounded Theory and Chapter \ref{ResearchMethodChapter} describes our data sources and particular use of Grounded Theory. This study had an unusually ambitious scope. In many Grounded Theory studies, the researcher spends a few weeks interviewing participants and analyzing data. Here, one researcher spent \durationOfResearchStudyPlural{} working with participants 5 days per week on \numberOfObservedProjects{} teams. 


Applying Grounded Theory at Pivotal, unearthed the theory of Sustainable Software Development, an explanation on how Pivotal teams can thrive despite team disruption and successfully deliver software to their stakeholders. Chapter \ref{SustainableSoftwareDevelopmentChapter} describes the theory’s principles, policies, and practices. The theory explains that by actively removing knowledge silos and caretaking the code teams survive disruption. At the heart of the Pivotal process is team code ownership. Many of Pivotal’s practices have the effect of promoting team code ownership while removing individual code ownership. Chapter \ref{TeamCodeOwnershipChapter} identifies five factors that affect the team’s sense of code ownership and explains why psychological ownership causes some engineers to struggle with the transition from individual code ownership to team code ownership. 


The Grounded Theory study also identified that Pivotal teams actively detect and remove waste. Since software development is an organic and messy problem space, finding all kinds of waste is no surprise. Chapter \ref{SoftwareEngineeringWasteChapter} conducts the first empirical research study into a waste taxonomy and contrasts it with Lean Software Development’s waste taxonomy. \sout{Some of the wastes types, such as \quotes{unnecessary cognitive effort,} are rarely salient unless one is specifically looking for them.}


Chapter \ref{ConclusionChapter} discusses future research and concludes the research study. 


Appendix \ref{AppendixChainOfEvidence} provides the chain of evidence for the waste taxonomy after iteratively applying constant comparison. 


Appendix \ref{AppendixInterviews} showcases interview data. The initial interviews began with the question, \quotes{draw your view of Pivotal’s software development process.} The appendix includes a sample of these drawings with interview transcript snippets where the interviewee describes the drawing. Once team code ownership emerged as a core category, many interviews began with the question, \quotes{draw how you feel about the code.} Section \ref{AppendixFeelAboutTheCode} provides some of these drawings. When the pictures are not obvious, a brief narrative explains the illustration.


% opportunities to remove waste are apparent even for companies that actively remove waste. 


\chapter{Research Context}
\label{ResearchContextChapter}


\section{Pivotal Labs}
Pivotal Labs is a division of Pivotal\textemdash a large American software company (with 17 offices around the world). Pivotal Labs provides teams of agile developers, product managers, and interaction designers to other firms. Its mission is not only to deliver highly-crafted software products but also to help transform clients' engineering cultures. To change the client's development process, Pivotal combines the client's software engineers with Pivotal's engineers at a Pivotal office where they can experience Extreme Programming \cite{BeckExtremeProgramming2004} in an environment conducive to agile development. 


Typical teams include six developers, one interaction designer, and a product manager. The largest project in the history of the Palo Alto office had 28 developers while the smallest had two. Larger projects are organized into smaller coordinating teams with one product manager per team and one or two interaction designers per team.


Interaction designers identify user needs predominately through user interviews; create and validate user experience with mockups; determine the visual design of a product; and support engineering during implementation. Product managers are responsible for identifying and prioritizing features, converting features into stories, prioritizing stories in a backlog, and communicating the stories to the engineers. Software engineers implement the solution. 


Pivotal Labs has followed Extreme Programming \cite{BeckExtremeProgramming2004} since the late 1990's. While each team autonomously decides what is best for each project, the company culture strongly suggests following all of the core practices of Extreme Programming, including pair programming, test-driven development, weekly retrospectives, daily stand-ups, a prioritized backlog, and team code ownership. We only observed teams at Pivotal Labs. Other teams, especially teams in other divisions, might have a different culture and follow different software practices.


\section{Software Development Culture}
When compared to other companies, Pivotal’s software development culture is marked by a high degree of collaboration and constant improvement.


\subsection{Collaboration}
The culture promotes collaboration by encouraging asking for help, interacting across disciplines, creating an ego-less environment, rotating who works on each part of the system, and pair programming. Experienced engineers model the culture so that new Pivotal or client engineers learn a new way of working.


Since \textit{asking for help} is a regular part of the day, welcoming interruptions is desirable. Asking another pair for help is a sign of respect as engineers enjoy that someone values their opinion. On teams with 10 or more engineers, it is possible for one pair to be interrupted too frequently. When the interruptions consume too much time, making forward progress on a story becomes challenging. 


Pivotal forms \textit{cross-functional teams} with software engineers, product managers, and interaction designers sharing a common goal. The \textit{balanced team} movement espouses the same philosophy as Pivotal.


An \textit{egoless culture} is promoted through a meritocracy. The team wants the best solution for the product regardless of who presents ideas. There is no discussion or awareness about job titles or seniority. While the HR system has engineering levels for comparing compensations across the organization, in day-to-day practice everyone is on the same level. Pivotal has a flat engineering culture. The rotation of project roles also reinforces an egoless organization, an engineer leading a project today will be an individual contributor on the next project. 


The \textit{rotation of engineers through different parts of the system} from the overlapping pair rotation practice (Section \ref{OverlappingPairRotationSection}) helps remove individual psychological ownership of the code and promote collective ownership. Ideally, anyone on the team can work on any part of the team’s code. Adopting practices that promote team code ownership (Chapter \ref{TeamCodeOwnershipChapter}) helps build a collaborating environment.


\textit{Pair programming} conditions individuals to work well with the rest of the team. A healthy pair programming dynamics is one comprised of listening, empathy, teaching, and learning. These then transfer from interactions with the partner to the rest of the team. The pair works on everything together from software development, asking for help, taking breaks, and attending meetings. If an individual is pulled into a project related meeting, they take their partner with them. Since the team desires to spread knowledge and context to the entire team, the team prefers for both to be involved instead of one person soloing. Pair programming can foster healthy habits that serve the entire team well.


\subsection{Constant improvement}


The culture embodies constant improvement through tight feedback loops. The feedback loops help the teams identify and remove waste such as those identified in Chapter \ref{SoftwareEngineeringWasteChapter}. 
Observed feedback loops at Pivotal include:
\begin{itemize}
  \item Daily standups
  \item Weekly retros
  \item Daily pair programming feedback for improving personal interactions
  \item User research for identifying user persona and user needs
  \item User validation for verifying product or features
  \item Usability testing for validating sketches and mock-ups
  \item Product managers accepting or rejecting a story for validating development work
  \item Tests for confirming successful code refactorings 
  \item Difficult to change code for revealing design problems
  \item Difficult to test component for exposing design problems
  \item Rework as feedback for features was not adequately described or not properly implemented
  \item Bugs as feedback for improving development process or testing strategy 
\end{itemize}


Ideal feedback is provided promptly so that the individual or team learn from the situation. Delaying course corrections delay the benefits. For example, one team had a ten-minute build which periodically annoyed them and slowed their development workflow down. On the last week of the project, the team examined the slowness of the build and realized that a simple resequencing of the tests could bring it down to three minutes. The pair lamented, \quotes{why didn’t we do this sooner?} as they could have benefitted from the results throughout the project. 


On a typical project, there often are many things that can be improved. The art is in prioritizing which pain points to address first. 


The weekly retro becomes a \quotes{catch-all} event for identifying any possible team improvement. Pivotal retros are open, cathartic experiences for the team, managed with seasoned facilitators. 
