\documentclass[oneside,letterpaper]{memoir}

\usepackage{citthesis}

\usepackage[T1]{fontenc}
% Load lmodern for bold \ttfamily
\usepackage[]{lmodern}
%\usepackage[lighttt]{lmodern}
%\usepackage[bitstream-charter]{mathdesign}
%\usepackage[urw-garamond]{mathdesign}
\usepackage[sc]{mathpazo}
%\usepackage{fourier}
%\usepackage{lmodern}

\usepackage{stmaryrd}
\usepackage{graphicx}

\usepackage[colorlinks,linkcolor=blue,filecolor=blue,citecolor=blue,urlcolor=blue,backref=page]{hyperref}

%\usepackage{caption}
%\let\subcaption\undefined
%\let\subfloat\undefined
\newsubfloat{figure}

%Todd added for fractions
\usepackage{xfrac}
%Todd added this for strikeout \sout{}
\usepackage[normalem]{ulem}
%Todd added this for the reference to essence.sv.cmu.edu
\usepackage{hyperref}
%Todd added this to fix tables with an H
\usepackage{float}
%for no orphan lines
\usepackage[all]{nowidow}
 %for lableling a table as a figure
\usepackage{caption}
% for ul which allows underlines to be split across a line break
\usepackage{soul}
% for rotating a table into landscape
\usepackage{longtable}
\usepackage{lscape} 


%Todd's commands
\newcommand{\strikeout}[1]{\sout{#1}}
\newcommand{\quotes}[1]{``#1''}
\newcommand{\participantQuote}[1]{\textit{``#1''}}
\newcommand{\singleQuote}[1]{`#1'}
\newcommand{\emphasis}[1]{\emph{#1}}
\newcommand{\ignore}[1]{}

\newcommand{\oneColumnWidth}{3.4in}
\newcommand{\twoColumnWidth}{6.3in}

\newcommand{\numberOfObservedProjects}{eight}
\newcommand{\numberOfInterviews}{28}

\setauthor{Todd Sedano}
%\settitle{Empirical Study of Iterative Software Development in Academia and Industry: Effectiveness, Optimization, and Extension of the Essence Kernel}
% \settitle{Empirical Study of Iterative Software Development in Industry}
\settitle{Iterating on Iterative Software Development: Evolving Extreme Programming}

\doctors
\setchair{Dr.\  C\'ecile P\'eraire}
\setdept{Electrical and Computer Engineering}
\setdegrees{B.S., Mathematics and Computer Science, Carnegie Mellon University\\
M.S., Software Engineering, Carnegie Mellon University}
\setdefdate{May 2017}
\setgraddate{May 2017}
\setcopyyear{2017}

\begin{document}

\frontmatter

\thetitlepage
\copyrightpage

\section*{Acknowledgements}
I want to thank my committee members for their guidance and expertise: my chair
 C\'ecile P\'eraire as well as Paul Ralph, Martin Griss, and To Be Determined. 

I wish to thank the leadership of Pivotal for making this research possible; thank you Rob Mee, David Goudreau, Ryan Richard, and Zach Larson. Thank you to Karina Sils for creating Figure \ref{PlannedDeveloperStaffing} and Figure \ref{DeveloperStaffing} using Sketch. Thanks to Ben Christel for his assistance in sorting retro topics and helping with the initial analysis for Software Engineering Waste.

\newpage
\section*{Abstract}
\textit{Context:} Software development is a complex socio-technical endeavor that involves coordinating different disciplines and skill sets. Practitioners experiment with and adopt processes and practices with a goal of making their work more effective.

\textit{Objective:} The purpose of this research is to observe, describe, and analyze software development processes and practices in an industrial setting. Our goal is to generate a descriptive theory of software engineering development, which is rooted in empirical data.

\textit{Method:} Following Constructivist Grounded Theory, I conducted a two-year five-month participant-observation of several software development projects at Pivotal, a software development company. I interviewed 26 software engineers, interaction designers, and product managers, and analyzed one  year  of  retrospection  topics.  I  iterated  between  data collection, data analysis  and  theoretical  sampling  until achieving theoretical  saturation and generating a descriptive theory.

\textit{Results:} 1)  This  research  introduces  a  descriptive  theory  of  Sustainable  Software Development.  The theory encompasses principles, policies, and practices aiming at removing knowledge silos and improving code quality (including discoverability and readability), hence leading to development sustainability. 2) At the heart of Sustainable Software Development is team code ownership. This research widens the current definition and understanding of team code ownership. It identifies five factors that affect ownership.  Developers achieve higher team code ownership when they understand the system context, have contributed to the code in question, perceive code quality as high, believe the product will satisfy the user needs, and perceive high team  cohesion.   3)  This  research  introduces  the  first  evidence-based  waste  taxonomy,  identifying  eight wastes along with causes and tensions within wastes. It also provides a comparison with the taxonomy of wastes found in Lean Software Development.

\textit{Limitations:} While the results are highly relevant to the observed company, Pivotal, the outcomes might not apply to organizations with different software development cultures.

\textit{Conclusion:} 1) The Sustainable Software Development theory refines and extends our understanding of Extreme Programming by adding new principles, policies, and practices (including Overlapping Pair Rotation) and aligning them with the business goal of sustainability. 2) Team code ownership is rooted in numerous cognitive, emotional, contextual and technical factors and cannot be achieved simply by policy. 3) The waste taxonomy serves as a starting point for waste identification and elimination. Comparing this taxonomy to Lean Software Development’s list of wastes revealed our taxonomy’s parsimony and expressiveness while illustrating wastes not covered by previous work. 

Introducing feedback cycles is one method to identify and remove waste, enabling software teams to iterate on iterative software development.  This research reveals team code ownership as the heart of collaborative software delivery. 


%Given the plethora of practices and methods, the Software Engineering Method and Theory (SEMAT) community created the Essence kernel as a unifying framework for describing and analyzing software engineering endeavors. 

%My research goal is to evaluate the Essence kernel for practical use on academic and industrial software development projects, identify issues, and research solutions grounded in empirical evidence. 

%At Carnegie Mellon University in Silicon Valley, I conducted a field study with masters of science in software engineering students as they completed team-based capstone projects using the Essence kernel. During weekly Essence Reflection meetings, the Essence kernel checklists helped students identify relevant goals to achieve, which enabled the team to steer the project to higher states. The student teams found value during project inception. However, teams found less value during the construction phase of iterative projects, as the Essence kernel offered few new goals hence loosing its ability to help the team steer the project. The original Essence kernel is method agnostic and does not directly support iterative development. Since most of agile software development occurs via iterative software development, adapting the Essence kernel to have goals for an iterative construction phase would increase its value to software development teams.

%Following these results in academia, my objective is to continue my research in industry, with a focus on the practices at Pivotal. One of my next goals is to generate a process model that accurately describes iterative software projects at Pivotal. My plan is to conduct participant-observation of several software development projects at Pivotal. I will interview many software engineers and product managers to collect additional data. I'll iterate my research by incorporating this feedback. My expected result is a process model grounded in empirical data that supports iterative software development. 

\newpage
\tableofcontents
\listoftables
\listoffigures

\mainmatter


%%
%% Start line numbering here if you want
%%
%%\linenumbers

%note that import will do a clearfix
% \chapter{Essence Reflection Meetings}
\section{Abstract}

This paper presents an empirical evaluation of the team reflection support provided by the Software Engineering Method and Theory (SEMAT) Essence framework, and compares Essence reflection meetings to other types of team reflection meetings. The researchers conducted a field study involving seven graduate master student teams running Essence reflection meetings throughout their practicum projects aiming at delivering a working product for an industry client. The main result validates that Essence meetings generate reflective team discussions through a thinking framework that is holistic, state-based, goal- driven, and method-agnostic. Student teams benefit from stepping back and assessing the project holistically throughout its lifecycle. The goals set by the framework's checklists lead the teams to address critical aspects of the project that have not been considered. All team members are encouraged to express their views and influence the various project dimensions. Essence reflection meetings are comparable and complementary to Agile retrospectives, and project teams might want to leverage both techniques. The value added by Essence reflections is to surface unknown issues, help monitor progress, steer the project to a higher state, and prevent retrospectives from being repetitive by varying styles.

\section{Introduction}

The authors investigated a novel approach to monitoring and steering software development projects provided by the Software Engineering Method and Theory (SEMAT) Essence framework \cite{SEMATKernel}. Among the various benefits, team reflection stands out as being the most appreciated aspect of the approach from a student point of view. Therefore this paper elaborates on this result by focusing specifically on Essence team reflection.

There exists different types of reflection meetings. Some, like post-mortems or project retrospectives, are conducted once at the end of the project (or release). Others, like Agile retrospectives, are conducted throughout the project lifecycle, typically at the end of each iteration or Sprint. There are many variations or styles of Agile retrospectives \cite{Derby2006, KuaRetrospectiveHandbook}, and different authors refer to them using different names, including iteration retrospectives, Sprint retrospectives, or heartbeat retrospectives. In this paper we explain why Essence reflection meetings are comparable to Agile retrospectives, highlight the similarities and differences between the two, and suggest how project teams could leverage both techniques in a complementary fashion.

This paper introduces the SEMAT's Essence framework, presents the field study, and reports on the field study results with a focus on team reflection.

\section{SEMAT Essence Overview}
The core idea of the Software Engineering Method and Theory (SEMAT) Essence framework \cite{SEMATKernel} is that software projects exhibit universal behavior and transition through identifiable states as they progress. The states are grouped by software engineering dimensions called \quotes{alphas.} Essence identifies seven alphas as core to every software engineering project: \textbf{Stakeholders}, \textbf{Opportunity}, \textbf{Requirements}, \textbf{Software System}, \textbf{Team}, \textbf{Way of Working}, and \textbf{Work}. These seven alphas serve as the Essence kernel. Each alpha progresses through a number of states during the project lifecycle. For example, the \textbf{Stakeholders} alpha progresses through the states \textit{Recognized}, \textit{Represented}, \textit{Involved}, \textit{In Agreement}, \textit{Satisfied for Deployment}, and \textit{Satisfied in Use}. Each state includes a checklist to help determine if the project has achieved that state or not. Table \ref{EssenceReflectionMeetings} shows the checklist related to the Bounded state of the \textbf{Requirements} alpha.




\section{Field Study Description}
The field study aims at evaluating the effectiveness of the SEMAT Essence's approach. A complete description is available in \cite{ICSE2014}. The study includes seven student teams: three geographically distributed student teams and four co-located student teams. Each team worked on creating or evolving a software solution for a different industry client, like an electric car fleet management system or a survivable social network. By design, the projects had a medium to high level of technical complexity, as they often involved multiple technologies or platforms or integrate with legacy systems. The practicum projects ran for 12 to 15 weeks, during which each student dedicated about 20 hours per week to the project. Students worked in teams of two to five members. Teams determined their own software development approach. Most students had a reasonable knowledge of a diverse set of generally accepted software engineering practices, and the ability to execute these practices somehow effectively. All projects adopted an iterative lifecycle.
   
The teams were asked to leverage Essence throughout their project. Each team met on a regular basis (mostly weekly) for a 30 minutes Essence session. During each session, the team covered most or all of the alphas. For each alpha, the team identified their project current state, target state, and any work items necessary to transition from the current to the target state. In order to avoid anchoring bias, the current state identification was performed using a \quotes{poker game} approach \cite{ICSE2014}. In that context, each participant secretly determines the current state and all reveal their current state at the same time. In case of disagreement, the team discusses the different points of view until the participants reach an agreement. Table \ref{EssenceReflectionMeetings} provides a conceptual representation of how Essence was used in practice by each team.

A faculty member was present to facilitate each session. Faculty involvement was kept to a minimum to limit influencing the students. The faculty's role was constrained to recording progress, guiding the team through the application of the approach, and validating the objectivity of the team's self-assessment of their project state. At the end of each project a survey was sent to the students to collect their feedback on the application of the approach.

The qualitative and quantitative value of Essence refection meetings was measured primarily based on students' feedback collected during the weekly meetings and final survey.

\section{Field Study Results}
\textbf{Research Question:} How does Essence support team reflection?

The original intent of each Essence session was to monitor the team's progress and steer the project towards higher Essence states. The sessions also provided a natural setting for team reflection. Indeed, a majority of students (72\%) spontaneously mentioned reflection or retrospectives in the survey responses (80\% of the students participated in the survey).

For instance, one student mentioned: \participantQuote{What I liked most about Essence is that it invoked reflective discussion.} Another student mentioned: \participantQuote{The team was pleased to see that Essence also covered `The Way of Working' as well as `The Team'. These two topics generated useful team introspection at the beginning of the practicum and were nice reminders that the team does constant checkups for the overall condition of the members and the project.} Overall, the survey responses touch upon the following key ingredients of Essence reflection meetings:

\textbf{Holistic Thinking Framework}. The seven alphas, together with their states and checklists, provide the team with a thinking framework encouraging the team to think about the project in a holistic fashion, based on seven project dimensions (a.k.a. alphas). One student noted: \participantQuote{Essence enabled the team to keep an eye on the status of the project by zooming out and assessing the overall picture.} Stepping back and looking at the project holistically provides the distance and perspectives needed to understand a situation, reflect, and make informed decisions.

\textbf{State-based \& Goal-driven Thinking Framework}. The Essence thinking framework evolves throughout the project lifecycle, based on the project's specific alpha states. At each state, new checklist items (goals) are presented to the team, encouraging the team members to think about and address aspects of the project that are relevant to the current state. One student noted: \participantQuote{I like the fact that Essence provides a structured way of thinking about critical aspects of the project at different stages of the project.}

\textbf{Team Discussion}. Essence reflection meetings enable all team members to express their views and influence the different aspects of the project. Here is an illustrating quote: \participantQuote{Essence meetings allowed everyone on the team to have a say in the different aspects of the project.} Another student added: \participantQuote{It allows us to reflect on where we stand in the project and remind us of the points we are missing.}

\begin{figure}[t]
\centering
\includegraphics[width=3.4in]{reflection_meeting_images/EssenceDiamondEffect.png}
\caption{ Essence kernel's diamond effect}
\label{EssenceDiamondEffect}
\end{figure}

\textbf{Method-Agnostic}. The team decides what to do to reach the goals set by the target states. The team has the flexibility to leverage any software development method or set of practices that best suit their needs. This is illustrated in Figure \ref{EssenceDiamondEffect} with the Essence kernel's \quotes{diamond effect,} where the kernel alphas \quotes{radiate} to enable reflective discussions touching the many facets of the project throughout its lifecycle, independently of the software development method adopted by the team.

In conclusion, Essence supports team reflection by generating reflective team discussions through a thinking framework that is holistic, state-based, goal-driven, and method-agnostic. The teams benefit from stepping back and assessing the project holistically throughout its lifecycle. The goals set by the alpha state checklists lead the team to address critical aspects of the project that have been neglected. These aspects go beyond technology by including elements like \textbf{Team}, \textbf{Way of Working}, or \textbf{Stakeholders}.

\textbf{Research Question}: How does Essence reflection meetings compare to other types of reflection meetings?

Essence reflection meetings follow a state-based approach, with states covering the entire project lifecycle. Consequently, Essence reflection meetings are most effective if conducted on a regular basis throughout the entire project lifecycle. Therefore, Essence reflection meetings are not comparable to post-mortems or project retrospectives that are conducted only once at the end of the project (or release). Essence reflection meetings could be compared to Agile retrospectives \cite{Derby2006, KuaRetrospectiveHandbook}, because they are also conducted throughout the project lifecycle, typically at the end of each iteration or Sprint.

In this section we are comparing Agile retrospectives and Essence reflection meetings in terms of purpose, frequency, duration, structure, content, outcome, and facilitation concerns.

\textbf{Purpose}. The goal of an Agile retrospective is for the team to contemplate what worked and did not work during the last iteration in order to adapt the methods and teamwork moving forward. The focus is mostly on the past. The goal of an Essence reflection is for the team to consider various project dimensions in order to bring the whole project towards a higher state. The focus is mostly on the future.
  
\textbf{Frequency}. Both Agile retrospectives and Essence reflections can be conducted at the end of an iteration or Sprint, or at other intervals defined by the project team. During our field study, each team generally met on a weekly basis. We recommend frequent sessions early in the project when many issues arise. Later on, once a team becomes a high-performing team producing high quality outcome, the team needs less support and the frequency of the sessions could decrease.

\textbf{Duration}. Both Agile retrospectives and Essence reflections can be time boxed to a short session ranging from 30 minutes to a few hours. During our field study, each team generally met for a 30- minute session. We recommend adjusting the duration based on the team size and any other known parameters that might influence the length of the conversations, like team dynamic, issues and uncertainty, or session frequency.

\textbf{Structure}. While facilitators may run Agile retrospectives differently, many adopt a structure similar to the one proposed by Derby and Larsen in \cite{Derby2006}. Derby and Larsen generalize the stages of Agile retrospectives as: (1) Set the stage, (2) Gather data, (3) Generate insights, (4) Decide what to do, and (5) Close the retrospective. Even though Agile retrospectives and Essence reflections have a different structure, there are some similarities. During an Essence reflection meeting, the team repeats the key steps of gathering of data, generation of insights, and deciding what to do for each alpha. With the Essence kernel's seven alphas, this produces seven focused passes through the Agile retrospective stages. This structure is illustrated in Table \ref{ReflectionMeetingStructure}.

\begin{table}[t]
\renewcommand{\arraystretch}{1.5}
\centering
\caption{Essence reflection meeting structure}
\label{ReflectionMeetingStructure}
\begin{tabular}{p{3in}}
\hline
\textbf{Set the stage} (done informally) \\
\textit{For each alpha}:
  \begin{itemize}
  \item \textbf{Gather data (alpha states)} 
  
   Discuss alpha-related work since last session
   and agree on current and target states
  
  \item \textbf{Generate insights}

  Understand \textit{why} the target state is not achieved
  
  \item \textbf{Decide what to do}
  
  Set some goals to reach the target state 
  and agree on how to reach the goals

  \end{itemize}
    
\textbf{Close the retrospective} (done informally) \\
\hline
\end{tabular}
\end{table}


\textbf{Content}. One difference between Agile retrospectives and Essence reflections relates to the elicitation of topics to be covered during a session. During Agile retrospectives the topics discussed are elicited by the participants, while during Essence reflections the topics are determined by the alphas and their corresponding checklists. Issues emerge once the related alphas are covered. As a consequence, Agile retrospectives tend to focus on known issues while Essence reflections tend to make unknown issues apparent by covering the project holistically and reminding participants of \participantQuote{critical areas that are sometimes neglected.}

\textbf{Outcome}. Both Agile retrospectives and Essence reflections result in a small number of work items to be addressed, ideally before the next session. During an Agile retrospective, participants typically generate many possible work items that are prioritized and then limited to a few high value items to be addressed during the next iteration. During our field study, an average of 5 work items were generated per session. The identified work items were added to the team's work item list or backlog, and fed into the next planning activity when applicable.

\textbf{Facilitation}. Both Agile retrospectives and Essence reflections benefit from being conducted by an experienced and neutral facilitator. While this is often recommended for Agile retrospectives \cite{KuaRetrospectiveHandbook}, the need for a facilitator is reduced with Essence reflections as the Essence alphas and their checklists guide the discussions. A facilitator is only required during the initial sessions for training purposes. Similarly, it is generally recommended to prepare for Agile retrospectives ahead of time \cite{Derby2006, KuaRetrospectiveHandbook}. Essence reflection meetings might require the facilitator to print the cards ahead of time. We are currently leveraging an open source tool (available at http://essence.sv.cmu.edu) developed internally that provides digital cards, hence freeing us from any preparation. With such a tool, Essence reflection meetings are conducted very effectively with geographically distributed teams.

In conclusion, Essence reflection meetings could be compared to Agile retrospectives. Despite similarities between the two approaches, there are some key differences in terms of purpose and content. While Agile retrospectives aim at inspecting the last iteration in order to adapt the methods and teamwork (with a focus on the past), Essence reflections aim at considering various project dimensions in order to bring the whole project towards a higher state (with a focus on the future). While most styles of Agile retrospectives tend to focus on known issues, Essence reflections tend to make unknown issues apparent by covering the project holistically and reminding participants of critical areas that might be overlooked. These differences make Essence reflections and Agile retrospectives complementary. This is illustrated by the following student quote: \participantQuote{Though the team was holding retrospectives every week already, having Essence discussions be a part of it allowed the team to touch on important aspects of the project; aspects which would otherwise be ignored.}

\section {Conclusion}
Essence reflections are valuable and complementary to Agile retrospectives. Facilitators and project teams might want to leverage both. For instance, one might decide to conduct regular Essence reflection meetings during project initiation when the monitoring and steering mechanisms are the most effective \cite{ICSE2014}, then alternate between Essence reflections and other styles of Agile retrospectives. The value added by Essence reflections is to surface unknown issues, help monitor and steer the project towards a higher state, and prevent retrospectives from being repetitive by varying styles.

The results presented in this paper are limited to Essence reflection meetings with a facilitator. More research is necessary to assess the meetings' effectiveness without facilitators. Following the field study, we have been observing eight additional practicum teams. Our observations are consistent with the ones presented in the paper. We continue to collect data to evaluate the SEMAT Essence's framework with a focus on both effectiveness of the approach and accuracy of the model.

\begin{table*}[t]
\caption{How Essence is used in practice by a student team}
\centering
\begin{tabular}{l|l}
\includegraphics[width=3.2in]{reflection_meeting_images/EssenceMeetingStep1.png} & 
\includegraphics[width=3.2in]{reflection_meeting_images/EssenceMeetingStep2.png} \\
\hline
\includegraphics[width=3.2in]{reflection_meeting_images/EssenceMeetingStep3.png} &
\includegraphics[width=3.2in]{reflection_meeting_images/EssenceMeetingStep4.png} \\
\hline
\includegraphics[width=3.2in]{reflection_meeting_images/EssenceMeetingStep5.png} &
\includegraphics[width=3.2in]{reflection_meeting_images/EssenceMeetingStep6.png} \\
\end{tabular}
\label{EssenceReflectionMeetings}
\end{table*}
% \input{essence-steering}
% \chapter{Essence Green Lighting}
\section{Abstract}

Many software engineering curriculum conclude with a
practicum or capstone project course. For courses involving external
clients, the course owner typically follows a Request for Proposal
process to vet (or green-light) qualified clients and projects.

Even though green-lighting projects does not guarantee project success, 
the goal is to reduce risks by systematically examining each proposal to identify
potential problems that the instructor could solve, mitigate against, 
or simply decide not to deal with by rejecting the proposal.

We propose and evaluate a Green-Lighting Approach based on the
SEMAT (Software Engineering Method and Theory) Essence framework. 
Our objective is to identify if such a framework could improve the Request for Proposal process 
at Carnegie Mellon University in Silicon Valley and other universities.

We conducted a case study by observing and interviewing the
course owner, examining a group of proposals, and identifying issues
with the current proposal process and practicum projects.
We proposed a green-lighting project state that, based upon Essence Alphas,
describes the minimal and ideal states that a project proposal should achieve to be accepted. 

The Green-Lighting Approach generated conversations among the faculty 
that clarified the guidelines for accepting and prioritizing proposals and identified deficiencies in our Request for Proposal. Additional work is required to refine the proposed Green-Lighting Approach based on current findings and further validate the approach.
  
Using Essence for green-lighting practicum projects in academia presents some limitations. 
The framework does not explicitly factor in business forces that affect proposal selection,
might be overly complex for the task, and might require modification with partial Alpha states. 
However, Essence provides a systematic approach for evaluating proposals based on various project dimensions. 
This approach could be used as an inspiration for deriving simpler custom green-lighting checklists. 

\section{Introduction}
\label{Introduction}

Many software engineering (SE) curricula finish with a practicum course
or capstone project \cite{GSWE}. At Carnegie Mellon
University in Silicon Valley, the curriculum culminates with a practicum
course in order for the students to demonstrate mastery of the
curriculum and to learn client management skills
\cite{Katz}. The practicum allows students to reinforce
their learning of core software engineering knowledge by applying this
knowledge to a different problem or domain. The practicum serves as
confirmation that the student has mastered the material. Earlier in the
curriculum, faculty manage the students project courses by playing the
customer or management role. The practicum provides an opportunity for
the students to work with a real client and practice client management
skills. Students actively manage the client engagement while faculty
observe and coach students without interfering unless necessary. The
students perform as a consulting team delivering a product that addresses the client's opportunity.

The university wants to increase its impact through the practicum
projects. The practicum course owner needs to find projects that
balance these goals:

\begin{itemize}
\itemsep1pt\parskip0pt\parsep0pt
\item
  maximize the students' learning experience, 
\item
  achieve a business goal or positively impact society, and
\item
  increase collaborations between university and industry. 
\end{itemize}

In order to accomplish the practicum goals, the faculty want to offer
a portfolio of projects including

\begin{itemize}
\itemsep1pt\parskip0pt\parsep0pt
\item
  a mix of project domains,
\item
  a mix of startups and established companies, and 
\item
  a mix of exploratory research and well defined
  endeavors.
\end{itemize}

Since we have a limited number of students and thus student teams, the
course owner needs to be selective about proposals with respect to these
goals. The course owner needs a systematic, non biased technique to
filter proposals.

\section{Related Work}
\label{Related Work}
Many software engineering professors have described their approach to providing a practical hands on experience. Most of the literature describes in detail how to run a team-based project course yet there is little discussion about the client selection process. 

In 1991, Shaw and Tomayko \cite{shaw1991models} examined hundreds of undergraduate software engineering courses and interviewed scores of instructors. In their technical report, they identify two major decisions that the instructor must make: 1) deciding the mixture of lecture and project components and 2) deciding the balance technical and managerial skills taught in the course. 

They describe several course models used in academia including the ``small group project'' model, ``large project team,'' model, and  ``the project only'' model. The ``small group project'' and  ``large project team'' models infuse lectures with a course long project. The ``project only'' model is typically a capstone course that focuses on the project experience.

Shaw and Tomayko discusses the importance of finding an interesting project that will motivate the students for the duration of the course, but provides no model for client selection.

In 2001, Cal Poly \cite {Turner2001} introduced a year long capstone project course. For the first two years, they relied on one industrial partner for the entire course, but due to coordination difficulties replaced the external project with a university project. No guidance is provided for project selection.

In 2002, Chamillard and Braun \cite{Chamillard2002} reflect on an undergraduate course at the U.S. Air Force Academy. They focus on tradeoffs such as the amount of guidance, documentation formats versus documentation examples, and focusing software development process versus product. They do not discuss their client selection process.

In 2002, Umphress, Hendrix, and Cross \cite{Umphress2002} reflect on their 18 years of experience and describe the transformation of processes from ad hoc, to MIL-STD-498 to IEEE 1074 to the Team Software Process to Extreme Programming. No information is provided about client selection.

In 2006, Coppit \cite{coppit2006} describes his strategies for overcoming the difficulties in running a large project team course including how to assess student performance. In his section on project selection, he briefly recommends that the amount of work should be commiserate with the length of the course, the scope needs to be flexible to allow cutting of features at the course's end, and have significant parallelizable work. 

In 2010, Ziv and Patil \cite {ziv2010capstone} discuss the experience of transitioning a capstone from one quarter to a three quarter course. The course follows a ``small group project'' model with external clients. The instructors typically have enough projects for all the enrolled students. Regarding selection criteria, the instructors take most projects within reasonable size, scope, goals and objectives, and reject only when the instructors notice a complete misunderstanding of the nature and purpose of a college-level undergraduate-level student project. 

Overall, criteria for client selection in the context of academic projects is only briefly discussed in the literature. 

\section{Background}
\label{Background}
For the past four semesters, we applied the Project Monitoring and
Steering Approach ~of the Essence framework
\cite{EssenceBook} in weekly Essence Reflection
Meetings \cite{EASE2014}. The approach provides student
teams with a simple, lightweight, non- prescriptive and method-agnostic
way to examine their projects holistically, structure team reflections,
manage risks, monitor progress and steer their projects
\cite{ICSE2014}.

During the research of applying the Project Monitoring and Steering
Approach, we observed practicum issues regarding stakeholder
representation, understanding the value of the opportunity, and undefined project scope and success criteria. Some aspects of these issues might be attributed to some extent to the way in which the practicum proposals are initiated and accepted. For example, one client believed that she could represent several different user personas and was unwilling for student teams to interview potential users. Our motivation is to examine our Request for Proposal process could help address these issues.

We proposed and applied a Green-Lighting Approach based on the Essence framework \cite{EssenceBook}. Our objective is to identify if such a framework could potentially improve the Request for Proposal process at Carnegie Mellon University in Silicon Valley and other universities.

The Green-lighting Approach
uses the Essence Alphas to describe how ready the project should be in
each Alpha, which we'll call ``green-lighting project state'' in this
paper. The green-lighting project state serves as a gating function,
filtering out unready projects.

In this paper, we will introduce the Green-lighting Approach which
provides a gating function for proposals. Section
\ref{Field Study Description} describes the field study
including the research goal, the current Request for Proposal, the
proposed change, and the study protocol. Section
\ref{Results} examines seven research questions
supporting the research goal. Section \ref{Conclusion}
summarizes that the findings and discusses future work.


\section{Green-lightning Approach of the Essence framework}
\label{Green-lightning Approach of the Essence framework}

In 2012, the SEMAT community released the Essence kernel
\cite{OMGStandard}. The Essence kernel describes a
software project through different dimensions called Alphas. For
example, the \textbf{Stakeholder} Alpha advances through the states
\textit{Recognized}, \textit{Represented}, \textit{Involved}, 
\textit{In Agreement}, \textit{Satisfied for Deployment}, and \textit{Satisfied in Use}. Each state has a set of checklist items. For example, \textit{Recognized} contains these three checklist items:

\begin{itemize}
\itemsep1pt\parskip0pt\parsep0pt
\item
  Possible stakeholders groups are identified
\item
  Team agrees on relevant stakeholder groups to be represented
\item
  Responsibilities of stakeholder representatives are defined
\end{itemize}

A project achieves a state when the team can check all the checklist
items for a state. This means that Essence represents projects through a
collection of linear state machines where the states are partially
ordered.

We defined the Green-lighting Approach based upon an example of using
the Essence framework from the Essence Book. Section 12: ``Running a
software endeavor: From idea to Product''
\cite{EssenceBook} describes how to use the Essence
Kernel Alpha States to define a staging process for a hypothetical
project. The example divides the project into four stages ``Getting
Ready to Start,'' ``Starting Up,'' ``Running Development,'' and
``Done.'' As the project progresses through these stages, the Alpha
states progress. In an organization managing many projects, one could
expect the projects to achieve certain states before making it into the
next stage. We define the Green-Lighting Approach to evaluate the
transitions between multiple stages.

The Green-Lighting Approach has three steps:

\begin{itemize}
\itemsep1pt\parskip0pt\parsep0pt
\item
  Evaluate each proposal and determine its state in each Alpha. For
  example, considering the  \textbf{Stakeholders} Alpha, a proposal that meets the three checklist items tor \textit{Recognized}
  and the four checklist items for \textit{Represented} would be marked as \textit{Represented} as seen in Figure
  \ref{EssenceAlpha}. This is repeated for each Alpha.
  This data can be represented as a hash where the keys are the Alphas
  and the values are the Alpha states, such as: \{stakeholders:
  ``represented'', opportunity: ``identified'', requirements:
  ``conceived'', software system: ``architecture selected''\}. The check marks in Figure
  \ref{EssenceAlpha} indicate the proposal's project
  state.
\item
  Determine the ``green-lighting project state,'' which is the minimal
  acceptable state for each Alpha. This too can be represented as a
  hash. In our example, the green-lighting project state could be:
  \{stakeholders: ``recognized'', opportunity: ``solution needed'',
  requirements: ``conceived'', software system: ``none''\}. The circled
  states in Figure \ref{EssenceAlpha} represent the
  green-lighting project state.
\item
  Applying the green-lighting project state to a proposal determines
  the project's readiness. A project is ready if the proposal has a
  larger or equal state for each Alpha in the green-lighting project
  state. Using the hash example, we compare each key of the hash and
  make sure the proposal is ``larger or equal'' to each corresponding
  key in the green-lighting project state. In the given example, the
  proposal is ready in each Alpha except Opportunity. In this regard, the approval process
  is a function with two inputs and one boolean output. F(Hash proposal,
  Hash greenlight\_project\_state) =\textgreater Boolean
  accept\_propopsal
\end{itemize}

% \begin{figure}[h]
% \includegraphics[scale=0.25]{EssenceAlpha.jpg}
% \caption{Essence alphas checked for a proposal and shaded for
% green-lighting project state}\label{EssenceAlpha}
% \end{figure}


\begin{figure}[!t]
\centering
\includegraphics[width=3.45in]{essence_green_lighting_images/EssenceAlpha.png}
\caption{Essence Alphas checked for a proposal and circled for
green-lighting project state}
\label{EssenceAlpha}
\end{figure}

In examining the seven Essence Alphas, only four Alphas,
\textbf{Stakeholders}, \textbf{Opportunity}, \textbf{Requirements}, and \textbf{Software System} are relevant to evaluating practicum proposals. The OMG standard defines these four Alphas as: \cite{OMGStandard}

\begin{itemize}
\itemsep1pt\parskip0pt\parsep0pt
\item
  \textbf{Stakeholders}: The people, groups, or organizations who affect or
  are affected by the software system.
\item
  \textbf{Opportunity}: The set of circumstances that makes it appropriate to
  develop or change a software system.
\item
  \textbf{Requirements}: What the software system should do to address the
  opportunity and satisfy the stakeholders
\item
  \textbf{Software System}: A system made up of software, hardware, and data
  that provides its primary value by the execution of the software.
\end{itemize}

The other three Essence Alphas (\textbf{Team, Way of Working} and \textbf{Work}) only make sense once a student team is
assigned to the project. Indeed, the \textbf{Team} has not been assembled yet,
hence it does not have a \textbf{Way of Working} and has not performed any
\textbf{Work}. By course design, the client does not determine the team's
\textbf{Way of Working}.


\section{Field Study Description}
\label{Field Study Description}

Following recommendations for reporting research done in the empirical
software engineering community
\cite{GQM, Shaw}, we formed our
research goal using Goal/Question/Metric:
\cite{GQM}

\begin{table}[h]
\renewcommand{\arraystretch}{1.3}
\centering
\begin{tabular}{|p{1.00in}|p{2.10in}|}
\hline
Analyze & SEMAT Essence’s Green-lighting Approach provided by the kernel Alphas and their states \\ \hline
for the purpose of & evaluation \\ \hline
with respect to its & effectiveness \\ \hline
from the point of view of the & educator and researcher \\ \hline
in the context of  & the software engineering practicum graduate course at Carnegie Mellon University. \\
\hline
\end{tabular}
\end{table}


This paper decomposes this goal into the following questions:


\begin{itemize}
\itemsep1pt\parskip0pt\parsep0pt
\item
  \textbf{Research Question 1}: What is the initial state of most projects
  based on the current proposals?
\item
  \textbf{Research Question 2}: What are some problems with the practicum that could potentially be mitigated to some extent prior to the start of the project?
\item
  \textbf{Research Question 3}: From the faculty perspective, what should be
  the minimum or ideal initial state of a project?
\item
  \textbf{Research Question 4}: How do the proposals compare against the
  proposed minimum and ideal green-lighting project states?
\item
  \textbf{Research Question 5}: Do we need to improve the Request for
  Proposal?
% \item
%   \textbf{Research Question 6}: What is the effect of using the Essence
%   Green-lighting Approach on the time it takes to accept or reject a
%   project?
\item
  \textbf{Research Question 6}: What are limits to the approach and
  limitations to its effectiveness?
\end{itemize}


\subsection{Current Practicum Request for Proposal}
\label{ProposalQuestions}

Prior to the start of the course, the course owner solicits project
proposals from industry and colleagues. The current practicum Request
for Proposal asks sponsors to create and provide a document including
the following information:

\begin{itemize}
\itemsep1pt\parskip0pt\parsep0pt
\item
  {Name of the project}
\item
  {Summary of the project}
\item
  {Overview of the sponsoring organization}
\item
  {Background and problem context }
\item
  {Relevance and Opportunity: why is it important and who benefits}
\item
  {Proposed scope of work}
\item
  {Major project goals and objectives }
\item
  {Technologies and skill sets requirements }
\item
  {Expected team size }
\item
  {Currently known obstacles}
\item
  {Nature of working relationship with sponsoring organization}
\item
  {Expected use of deliverables at project completion}
\item
  {Preliminary project roadmap}
\item
  {Criteria for measures of success}
\item
  {Any IP, NDA, or citizenship constraints}
\end{itemize}

The course owner wants any submitted proposal to clearly communicate
the project's big picture, the client's needs, and a general project
roadmap. At the due date, the course owner reviews the submitted
proposals to verify their completeness. The course owner verifies that
the project is an appropriate educational experience, and relies on the
students to filter the projects. The course owner encourages
students to contact the client for clarification, if necessary.

Given the number of students enrolled in the course, the course owner
determines the expected number of teams for the course. In years when
there are significantly more project proposals than expected number of
teams, the students use dot voting to cull the list down to a manageable
number. For example, for 32 students, there might be 6 to 8 teams. If
there are 16 acceptable proposals, the course owner reduces it to 12
practicum proposals by the students dot voting their favorite projects.

Once there is a ``shortlist'' of proposals, the course owner invites
all the clients and students to a practicum fair. The course owner asks
the students to be familiar with all of the practicum proposals. The
purpose is to provide a forum for the students to ask the clients
specific questions about the project, not for the client to give a
complete presentation about the project. The clients introduce themselves
and have a summary slide or two to remind the students about their
project. 

The students then submit a ranked ordering of all the projects from
their number one pick to their least favorite, and the course owner
forms teams. The course owner tries to assign students based upon their
first or second top choice while prioritizing paying clients. However,
this is not always possible when too many students select the same
project or when only one student selects a project.

\subsection{Proposed Practicum Request for Proposal}

We consider a modification to the current practicum Request for
Proposal process by adding a filtering step after the proposal are
received but before the proposals are shown to the students. For each
proposal, we evaluate the project's state described by the proposal
against a green-lighting project state defined using SEMAT Essence Alpha
states. The course owner only offers the students proposals that have
reached or exceeded the green-lighting state. If a proposal does not
meet the minimum criteria, then either the course owner rejects the
proposal or the course owner discusses issues with the sponsor to
illicit more information.

Various elements, like the university relationship with the client, or financial considerations, are not taken into account by the SEMAT Essence Kernel, which is the first identified limitation of the framework for the purpose of green-lighting projects in academia.

\subsection{Study Protocol}

The study protocol is as follows:

\begin{enumerate}
\itemsep1pt\parskip0pt\parsep0pt
\item
  We reviewed 21 submitted practicum proposals and
  identified their initial project state (refer to \textbf{Research Question
  1} for more information). In order to remove anchoring bias, we examined
  and rated each proposal independently. Any discrepancies were
  discussed in person.
\item
  Based upon our experience of observing the practicum course in the
  context of prior investigations \cite{EASE2014, ICSE2014},
  using a brainstorming session, we identified recent issues that
  might be addressable to some extent prior to the start of the course (refer to
  \textbf{Research Question 2} for more information).
\item
  During the same brainstorming session, we recommended a minimum and
  an ideal green-lighting project state, with the purpose of addressing
  the identified issues prior to the start of the course. The minimum
  and ideal states need to be realistic and feasible with respect to the
  kinds of projects that we receive (refer to \textbf{Research Question
  3} for more information).
\item
  We compared the state of 21 submitted proposals against the proposed
  minimum and ideal green-lighting project states (refer to \textbf{Research
  Question 4} for more information). Again, in order to remove
  anchoring bias, we reviewed and rated each proposal independently. Any
  discrepancies were discussed in person.
\item
  We proposed modifications to the current questions in the Request for
  Proposal to better reveal the initial green-lighting project state. We
  evaluated the new questions with the sponsors (refer to \textbf{Research
  Question 5} for more information).
\item
  We analyzed the study results and drew conclusions on the benefits
  and drawback of the proposed approach (refer to \textbf{Research Question 6
  } for more information).
\end{enumerate}

\section{Results and Discussion}
\label{Results}

\subsection{Research Question 1: What is the initial
state of most projects based on the current proposals?}

We started the Green-lighting Approach research by assessing the
initial states of the \textbf{Stakeholders}, \textbf{Opportunity},
\textbf{Requirements} and \textbf{Software System} Alphas for 21 proposals.

For the \textbf{Stakeholder} Alpha, we observed that:
\begin{itemize}
\itemsep1pt\parskip0pt\parsep0pt
\item
  33\% did not achieve any state. These proposals did not identify
  stakeholder groups.
\item
  48\% were in the \textit{Recognized} state. These proposals identified the
  stakeholder groups, but did not appoint the representatives.
\item
  19\% were in the \textit{Represented} state. These proposals
  identified the stakeholder groups and the representatives of each
  group.
\end{itemize}

For the \textbf{Opportunity} Alpha, we observed that:
\begin{itemize}
\itemsep1pt\parskip0pt\parsep0pt
\item
  53\% were in the \textit{Identified} state. These proposals indicated the
  need for a software solution with the stakeholders wishing to make an
  investment.
\item
  33\% were in the \textit{Solution Needed} state. These proposals clearly
  articulated the problem with confirmation on the need for a solution.
\item
  14\% were in the \textit{Value Established} state. These proposals
  established the business value with a clear definition of desired
  outcomes and success criteria.
\end{itemize}

For the \textbf{Requirements} Alpha, we observed that:
\begin{itemize}
\itemsep1pt\parskip0pt\parsep0pt
\item
  5\% did not achieve any state.
\item
  71\% were in the \textit{Conceived} state. These proposals captured the 
  itemize system purpose with the user types involved.
\item
  19\% were in the \textit{Bounded} state. These proposals defined their 
  scope with a clear definition of the success criteria.
\item
  5\% were in the \textit{Coherent} state. These proposals captured and
  prioritized the requirements.
\end{itemize}

For the \textbf{Software System} Alpha, we observed that:
\begin{itemize}
\itemsep1pt\parskip0pt\parsep0pt
\item
  90\% did not achieve any state. These proposals did not describe the
  platforms, technologies or languages for the project.
\item
  10\% were in the \textit{Initiated} state. These proposals identified the
  criteria for selecting the architecture. These proposals described the key technical risks and buy, build or reuse decisions made for the project.
\end{itemize}

\begin{figure}[!t]
\centering
\includegraphics[width=3.45in]{essence_green_lighting_images/ProposalAlphasChartsStacked.png}
\caption{Essence Alphas checked for a proposal and circled for green-lighting project state}
\label{ProposalChart}
\end{figure}


In summary, we identified that 19\% of the proposals had represented
\textbf{Stakeholders}, 14\% of the proposals established the business value
of the \textbf{Opportunity}, 95\% \footnote{95\% is all the projects minus the 5\% that did not achieve any state} of the proposals captured the high level \textbf{Requirements} (and 24\% \footnote{24\% is the percentage of Bounded (19\%) and Coherent (5\%) projects} captured the project scope and success criteria), and 10\% of the proposals had defined criteria for selecting or identifying the \textbf{Software System} architecture. Since we do not expect the proposals to define architecture selection criteria, we are mostly concerned about the low
percentages for \textbf{Stakeholders} and \textbf{Opportunity} Alphas. Regarding
the \textbf{Requirements} Alpha, we would also like to increase the number of
proposals with defined project scope and success criteria.


\subsection{Research Question 2: What are some problems with the practicum that could potentially be mitigated to some extent prior to the start of the project?
}

While experimenting with Essence Reflection Meetings in the
context of practicum projects \cite{EASE2014, ICSE2014}, we
noticed that some practicum issues could potentially be identified and
addressed before the engagement started. These issues typically involve:

\begin{itemize}
\itemsep1pt\parskip0pt\parsep0pt
\item
  Missing stakeholder representation
\item
  Unclear opportunity value, and
\item
  Undefined project scope and success criteria
\end{itemize}

\textbf{Missing stakeholder representation:} Philosophical differences about
stakeholder representation present a challenge. The faculty believe that
most stakeholder groups should be represented and that a single
person cannot represent every stakeholder group because the groups
typically have very different needs. On one project, the team was building
a benchmark for comparing native code versus HTML5 code
on Android devices for the purpose of publishing the results to the
development community. While using the Essence framework for project
monitoring and steering \cite{ICSE2014}, the team
realized that no one represented the development community stakeholder
group. The team proactively interviewed local mobile developers and
presented the findings to the client. The client chose to ignore the
feedback putting at risk the potential benefit of the benchmarking
report to the development community. If a client is unwilling (or does
not understand the need) to have a representative for each (or most)
stakeholder groups, then the Request for Proposal process should
identify this issue and we can discuss our perspective with the client. If
the client is unwilling to change, we can decide to reject the
proposal.

\textbf{Unclear opportunity value and undefined project scope and success
criteria:} On a few projects, the teams spent the first two to three
weeks identifying the underlying problem and
analyzing the market trends and competitive landscape to validate the
need for a solution. Then they spent more iterations agreeing on the
project scope and success criteria. While this is an interesting
exercise, this significantly delayed the implementation work so the teams delivered substantially less functionality than the other
teams. Establishing the value of the opportunity and potentially
defining the project scope and success criteria at the start of the
practicum would have provided the practicum teams enough time to
build an interesting solution.

We believe that improving ``project readiness'' enhances the student experience, which remains to be verified in future research. If the stakeholder groups are not represented, then the
client can find representatives prior to the start of the project. If a client cannot clearly articulate the opportunity, the project's
benefits, the project's scope, or the project's success criteria, then
we can accept other proposals that have a larger impact.

Addressing these issues prior to the start of the practicum would
better align the expectations of the client with our educational goals, give the students a richer experience, provide more value to the client, and further the impact of the university.

\subsection{Research Question 3: From the faculty perspective, what should be the minimum or ideal initial state of a project?}

\begin{table*}
\renewcommand{\arraystretch}{1.3}
\caption{Green-lighting project state for each Essence Alpha (CMU1.1)}
\label{GreenLightingProjectState}
\begin{tabular}{|l|p{2.50in}|p{3.35in}|}
\hline
\textbf{Alpha}  & \textbf{Minimally acceptable}       & \textbf{Ideal}                        \\ \hline
Stakeholders    & \textbf{Recognized (Partial)}       & \textbf{Represented (Partial)}        \\ 
                & (check) Possible stakeholder groups are identified & (check) Stakeholder representatives are appointed \\
                & (not) Team agrees on relevant stakeholder groups & (check) Stakeholder representatives agree to take on responsibilities \\
                & to be represented                 & (not) Stakeholders agree on collaboration approach \\
                & (not) Responsibilities of stakeholder representatives  & (not) Representatives respect team's way of working \\
                & are defined                         & (check) Stakeholder representatives empowered to take on responsibilities \\ \hline
Opportunity     & \textbf{Solution Needed (Complete)} & \textbf{Value Established (Complete)} \\[5ex] \hline
Requirements    & \textbf{Conceived (Complete)}       & \textbf{Bounded (Partial)}            \\
                &                                     & (check) Purpose and extent of system are agreed \\
                &                                     & (check) Success criteria are clear \\
                &                                     & (not) Processes and tools for handling requirements are in place \\
                &                                     & (check) Scope constraints are identified \\
                &                                     & (not) Assumptions made while defining requirements are captured \\ \hline
Software System & None                                & \textbf{Architecture Selected (Partial)} \\
                &                                     & (check) Criteria for selecting architecture are agreed \\
                &                                     & (check) Platforms, technologies, languages are selected \\
                &                                     & (check) Selected architecture addresses key technical risks \\
                &                                     & (check) Buy, build, reuse decisions are made \\
                &                                     & (not) Stakeholders agree on necessary documentation \\
                &                                     & (not) Stakeholders agree on support service levels \\
                &                                     & (check) Non-functional architectural characteristics are considered \\ \hline
\end{tabular}
\end{table*}

After reviewing the practicum proposals and reflecting on practicum
issues listed in \textbf{Research Question 2}, we created green-lighting project
states for each Essence Alpha documented in Table~\ref{GreenLightingProjectState}.

The existing literature \cite{EssenceBook} implies
that staging (green-lighting in our case) can be done at the Alpha state level. In the process of creating our green-lighting project states, we discovered that for some of the Alpha states, we could not select all the checklist items in the entire state. For example, at the proposal stage, we do not expect responsibilities of the stakeholder representatives to be defined; this happens through a conversation between the team and the stakeholder representatives. In creating the green-lighting project state, we ignored some checklist items thus creating partial Alpha states.

The minimally acceptable and ideal states were defined with the goal of mitigating the problems identified in the previous research question, and based on the researchers expertise. Further work is necessary to validate the proposed states for each Alpha in the context of our practicum projects. Note however that other project courses might require different sets of minimally acceptable and ideal states to better address their specific needs.

\subsection{Research Question 4: How do the proposals
compare against the proposed minimum and ideal green-lighting project
states?}

Now that we have evaluated the proposals (Step 1) and created the
green-lighting project states (Step 2), we identify which proposals
meet the green-lighting criteria (Step 3 of the Green-lighting Approach). 

Table~\ref{table_proposal_evaluations} shows the proposals from the Spring 2014 and Summer 2014 semesters that meet the minimally acceptable and ideal criteria for green-lighting. 

\begin{table}
\renewcommand{\arraystretch}{1.3}
\caption{Proposals compared to green-lighting project states}
\label{table_proposal_evaluations}
\begin{tabular}{|p{1.10in}|p{0.95in}|p{0.95in}|}
\hline
Alpha & Number of projects that met minimally acceptable criteria & Number of projects that met ideal criteria \\ \hline
Stakeholders & 14 out of 21 & 4 out of 21 \\ \hline
Opportunity & 10 out of 21 & 3 out of 21 \\ \hline
Requirements & 20 out of 21 & 5 out of 21 \\ \hline
Software System & 21 out of 21 & 2 out of 21 \\ \hline
All (Satisfies all Alphas) & 8 out of 21 & 0 out of 21 \\ \hline
% Overall (Across all Alphas) & 8 out of 21 & 0 out of 21 \\ \hline
\end{tabular}
\end{table}

This analysis revealed a gap between the proposals and our
expectations, specifically in regard to representing stakeholders and
establishing the value of the opportunity. None of the proposals matched our ideal criteria across all Alphas, while 8 out of 21 satisfied our minimal expectation.

Around one third of the proposals did not have appropriate
representation for different stakeholder groups. In these proposals, the
client themselves would define, prioritize and validate the needs of the
different stakeholders groups. In our experience, when a team finishes
this kind of project, there is a high probability that the solution
would not deliver sufficient value for each stakeholders group. In some
cases, the solution may be unusable.

Around half of the proposals did not clearly state the
opportunity. The proposals described the problem to be solved, but did
not articulate the benefit of solving the problem. In a few cases, we
suspect the client has found a ``solution'' but has not yet identified
the problem to solve. In our experience, at the end of the project,
the project would be ``successful'' in delivering code to the client,
but might not solve a real world problem. Alternatively, the client
might have a detailed understanding of the opportunity, but may have not
clearly communicated that opportunity in the proposal.

The current proposals contain the same issues as the previous practicums identified in Research Question 2. The Green-Lighting Approach surfaced that the current proposals could repeat the problems identified in the past.


\subsection{Research Question 5: Do we need to
improve the Request for Proposal?}

In observing the gap between the proposals and our desired minimal
state, we wondered if asking different questions would help the sponsors
to record the kind of information that we need. Since the proposal's
content is a proxy for the sponsor's knowledge, is it possible that the
sponsor has more detailed knowledge that is not recorded in the proposal?
In addition, simply asking a question might cause the client to do some
groundwork not originally considered, which could potentially move the
project to a higher initial state. 

While analyzing the proposals, we had difficulty determining the stakeholder groups in the
 \textbf{Stakeholders} Alpha, the projected value of software system in the
 \textbf{Opportunity} Alpha, the list of high-level features that captures
the system purpose in the \textbf{Requirements} Alpha, and the non-functional
architectural characteristics in the \textbf{Software System} Alphas. The
Request for Proposal does not clearly ask for this information. We need
to ask specific questions to determine if the client can articulate the
answers. Improving the Request for Proposal would also facilitate
easier analysis for these aspects.

In order to help ascertain this information, we asked the sponsors four
questions.

{Survey Question 1. List your stakeholder groups and identify who will be
representing each group. (Note: Ideally there should be a different
person representing each group.)}

In examining the free-text answers, about \sfrac{2}{3} of the sponsors listed
out different stakeholder groups. A few even named who would fulfill the
role. One third of the sponsors listed themselves as the primary and
only stakeholder. In reviewing the proposals, these sponsors are not the
target stakeholder groups. This suggests that these proposals could be
misaligned with our educational goals of involving every stakeholder
group in the process.

{Survey Question 2. What value would the stakeholders receive from a successful engagement? If possible, please quantify the value (e.g. monetary
value, social return).}

In examining free-text answers, about half of the respondents were not
able to articulate the opportunity of the proposal. The other half of
the respondents appealed to social returns such as improving the
emergency response situations which could save lives and reduce injuries
or connecting seniors to their families and caregivers. None of the
respondents were able to put a monetary value to the opportunity.

{Survey Question 3. What features would you like to see implemented during the
practicum? }

In reviewing the free-text answers, about half of the sponsors provided
a feature list. The other half provided vague answers to the question.
This question needs to be rephrased. Perhaps asking the client to
prioritize the features to be delivered would be more effective.

{Survey Question 4. What are the key non-functional requirements (e.g.
scalability, security, performance, etc) that the solution should
satisfy?}

Most of the sponsors were able to clearly articulate the quality
attributes needed for the project. One sponsor said, ``Software will be
written in C/C++, documented well and reusable.'' 

When the answer does not fully or properly articulate the
non-functional requirements of the system, we recommend that the course
owner interviews the sponsor to acquire this information. 

The results from questions 1 and 2 are consistent with our assessment
of the proposals from Research Question 1. Adding these questions would
facilitate analysis but might not cause the sponsors to change the
stakeholder representation. Additional conversations between the course 
owner and the sponsor might be necessary.

Based upon these results, we plan to add questions 1, 2, and 4 to our
Request for Proposal. We will continue to experiment to find a question
that elicits a list of high-level features. These improvements to
our Request for Proposal will help us better assess the initial states
of the proposals.

% \subsection{Research Question 6: What is the effect of
% using the Essence Green-lighting Approach on the time it takes to accept
% or reject a project?}

% With modifications to the Request for Proposal questions, we believe
% that the Essence Green-lighting Approach would structure, simplify, and
% reduce the amount of time it takes to accept or reject a proposal when compared to our current process.
% However this remains to be verified.

In the current Request for Proposal process, the course owner reads
through each proposal to verify its completeness. We noticed that
clients submit ad-hoc proposals without necessarily following the
provided guidelines in terms of structure and content. This makes
assessing the readiness of each proposal arduous. The Request for
Proposal process provides a non-editable document and asks the sponsor
to address the items listed in Section
\ref{ProposalQuestions}. We recommend replacing this
document with a restructured editable template where the sections are
clearly aligned to the four Essence Alphas. Clients would be asked to
fill-in the sections of this template. 

% With the proposed Green-lighting Approach, the course owner still needs
% to read through each proposal, but instead of looking for completeness,
% the course owner assesses the four Essence Alphas in the green-lighting
% project state. This requires the small additional effort of reading
% through at most 23 checklist items, and even less if the proposals are
% less mature. These changes would reduce the time on task for the current
% verification for completeness of ad-hoc proposals by organizing the data
% provided by the client and simplifying the assessment of the
% green-lighting project states.

\subsection{Research Question 6: What are limits to
the approach and limitations to its effectiveness?}

\subsubsection{Green-lighting Approach does not factor important business considerations}
The approach provides data for a structured decision making process for
accepting and rejecting proposals, however, there might be reasons to
have a project go forward even though green-lighting says ``no.''
Several examples include paying clients, a high impact project, career
opportunity for students, and potential partnership for research
collaboration. While the approach provides a black and white answer,
we suspect that sound judgment is required for special circumstances.

\subsubsection{Green-lighting project states do not always
align to Alpha states}

In following the Green-lighting Approach, we discovered that partial
states, not Alpha states, represent our proposal acceptance criteria.
Some of the checklist items on one Alpha state would apply to accepting
a proposal and the rest of the checklist items would not. For example, 
for us to green-light a proposal, we would expect that the
proposal would meet the first checklist item of the
 \textbf{Stakeholders} \textit{Recognized} card, not the second or third as
described in Table \ref{GreenLightingProjectState}. 
Using partial states is possible but inconvenient. 

\subsubsection{Green-lighting Approach relies upon the Essence framework}
The Green-lighting Approach relies upon the Essence kernel for providing 
a systematic framework for evaluating proposals. The case study shows that  
only a subset of the framework is leveraged: Only 4 out of the 7 Alphas are 
relevant, and only the first few states (maximum 3) of each Alpha are necessary 
to green-light a project. In addition, some of the states need to be only partially 
considered as described in the previous section. Therefore, it is possible that a simpler checklist 
could be evolved, rather than relying upon the Essence framework. The extra 
complexity could be justified however if the project team continues to use 
the Essence framework during development to leverage mechanisms like progress 
monitoring and project steering. In that case it might make sense to use the 
same framework throughout the project.

\subsubsection{Green-lighting Approach does not guarantee project success}
Given the variety of issues that can emerge from a student project experience, and the fact that the initial state has only a limited impact on the overall project, the Green-Lighting Approach is not a silver bullet that will solve all team-based issues.
Only some risks might be reduced by systematically examining each proposal to identify
potential problems that the instructor could solve, mitigate against, 
or simply decide not to deal with by rejecting the proposal.

\subsection{Threats to Validity}
Internal Validity: This work assumes that it is desirable to filter
proposals and that is is possible to select proposals that will better achieve the
practicum goals listed in Section \ref{Introduction}. It could be the case
that a rejected proposal would have more significant learning
opportunities for the students than an accepted proposal.

Experimenter bias: We each had one years worth of experience in working with
the Essence framework prior to the start of this research. It is
possible that someone with less experience may encounter different results.

\section{Conclusions and Future Work}
\label{Conclusion}

This paper presents a first attempt to characterize and study a project 
selection process in academia. We proposed a Green-lighting Approach using the Essence framework and
applied the approach to the Request for Proposal process for the
practicum course at Carnegie Mellon University in Silicon Valley. 

After receiving and reviewing a set of proposals, the faculty determined each 
proposal state for each Essence Alpha. Based upon these initial states, prior experience, 
and recent problems, we created a gating function called green-lighting project
state to screen out proposals that are not ready for student teams.

After creating the green-lighting project state for each
Essence Alpha, we realized that our current Request for Proposal did
not prompt the clients to provide enough detail about the stakeholder
representations, the projected value of the opportunity, the project
scope and success criteria, and non-functional architectural
characteristics of the software system. We created draft 
questions and then tested the new questions with the prospective
clients. We recommend modifying our Request for Proposal to include
an editable template structured around the Essence Alphas and to include
several of the new questions identified in this paper. 

Further research is necessary to validate our green-lighting project state 
as well as our modified Request for Proposal. 
Even though green-lighting projects does not guarantee project success, 
the goal is to reduce risks by systematically examining each proposal to identify
potential problems that the instructor could solve, mitigate against, 
or simply decide not to deal with by rejecting the proposal.
More work is necessary to verify that our approach helps us reach that goal.

Using the Essence framework for green-lighting practicum projects in academia presents some limitations. 
First, the approach does not explicitly factor in business forces that affect proposal selection. 
Second, the Essence framework might be overly complex for green-light practicum projects, 
as only a subset of the Essence framework is necessary to perform the task.
In addition, the need to split the checklist items on the Alpha states to represent green-lighting 
project states prevents us from using the Essence cards out of the box hence losing the simplicity of the cards.

Still, project courses that use the Essence framework during development (to leverage mechanisms such as progress 
monitoring and project steering \cite{ICSE2014}) might want to borrow some ideas from the proposed Green-Lighting Approach, 
as it might make sense to leverage the same framework throughout the project.
In that case, we recommend customizing the approach by defining minimally acceptable and ideal states based on the course specific needs. 

For project courses that are not planning on using Essence during development, 
the framework could still be used as an inspiration for deriving simple custom green-lighting checklists for various project dimensions. Better aligning client proposals and educational goals has the potential of enhancing the students learning experience.


% Sample apostrophy's to remove team's 

\chapter{Grounded Theory}
\label{ConstructivistGroundedTheoryChapter}
Constructivist Grounded Theory \cite{Charmaz} provides an iterative approach to data collection, data coding, and analysis resulting in an emergent theory. 

Grounded Theory immerses the researcher within the context of the research subject from the point of view of the research participants. As the research progresses, Grounded Theory allows the researcher to incrementally direct the data collection and theoretical ideas. The process provides a starting place for inquiry, not a specific goal known at the beginning of the research. As the researcher interacts with the data, the data influences how the research progresses and guides the research direction. When starting a Grounded Theory research study, the core question is, \quotes{what is happening here?} \cite{GlaserTheoreticalSensitivity}. The researcher can apply this question to a domain of study, for example, \quotes{what is happening at the studied organization when it comes to software?}

Charmaz encourages the researcher to follow this Grounded Theory strategy: \quotes{seek data, describe observed events, answer fundamental questions about what is happening, and then develop theoretical categories to understand it} \cite{Charmaz}.

\begin{figure}[t]
\centering
\includegraphics[width=6.4in]{grounded_theory_images/tweed_grounded_theory.png}
\captionof{figure}{Visualization of Grounded Theory by Tweed and Charmaz \cite{TweedGroundedTheory} }
\label{TweedVisualizationGroundedTheory}
\end{figure}

Grounded Theory follows an iterative process as illustrated in Figure \ref{TweedVisualizationGroundedTheory} \cite{TweedGroundedTheory}. After identifying a population to study, data is collected from interviews, participant observation, ethnographic study, or existing documents. Each kind of data, such as interviews, are coded and analyzed using constant comparison for the purpose of generating insights and recording memos. The coding process labels data with tags that emerge from the data. Constant comparison compares codes to codes for the purpose of identifying relationships between the codes. Memos are researcher diary notes. The principal activity is creating memos, and it supersedes and interrupts all other activities. The researcher sorts memos into a paper draft with data collection and analysis continuing until theoretical saturation occurs. If a new avenue of research arises,  additional data collection techniques are sometimes useful for generating the needed data. The data advances from the initial codes to focused codes, focused codes to core categories, and core categories to an emergent theory. 

%Research questions dictate which research methods are used. Charmaz argues that interviews are the best kind of data collection mechanism for certain kinds of research questions. (Todd: Need more detail)

Grounded Theory allows the addition of new aspects of the research while gathering data, which can even happen late in the analysis. The research question guides the data collection process, and the accumulation of knowledge alters the data collection process. It is common for early research to illuminate new angles. \quotes{Grounded Theory can give you flexible guidelines rather than rigid prescriptions} \cite{Charmaz}. The researcher's guiding interests can be used \quotes{as \quotes{points of departure} to form interview questions, to look at data, to listen to interviewees, and to think analytically about the data} \cite{Charmaz}.

There are three widely used versions of Grounded Theory: Classic Grounded Theory as championed by Barney Glaser \cite{GlaserDiscovery}, Straussian Grounded Theory as documented by Anselm Strauss and Juliette Corbin \cite{Strauss1988Basics}, and Constructivist Grounded Theory as described by Kathy Charmaz \cite{Charmaz}. 

Barney Glaser and Anselm Strauss invented Grounded Theory while performing the research for the \ul{Awareness of Dying} book \cite{GlaserAwarenessOfDying}. In response to associates asking for more details about the research process utilized in producing the novel results for \ul{Awareness of Dying}, Glaser and Strauss wrote \ul{The Discovery of Grounded Theory} \cite{GlaserDiscovery} in 1967. In 1988, Strauss and Juliet Corbin published \ul{Basics of Qualitative Research} \cite{Strauss1988Basics}. Glaser strongly felt that the book mischaracterized the approach, even asking Strauss to withdraw it from publication. Apparently, both authors had different points of view for what they invented. In a heated response, Glaser published \ul{Basics of Grounded Theory Analysis} in 1992 \cite{GlaserBasics} clearly differentiating his perspective from Strauss. \todo{In time, Glaser's approach became known as \quotes{Classic Grounded Theory} and Strauss's approach as \quotes{Straussian Grounded Theory.}} In time, several researchers evolved the approach using variations (such as recording an interview) that Glaser clearly describes as not part of Classic Grounded Theory. Furthermore, they switched from a positivist point of view to a constructivist perspective. One of these researchers, Kathy Charmaz, a Ph.D. student of Glaser, published \ul{Constructing Grounded Theory} in 2006 \cite{Charmaz}. 

\begin{figure}[h]
\centering
\captionof{table}{Stol's Grounded Theory Comparison}
\label{GroundedTheoryComparison}
\includegraphics[width=6.4in]{grounded_theory_images/stohl_grounded_theory_comparison.png}
\end{figure}

Stol et al \cite{StolGroundedTheory} provide an excellent summary of the differences between the three types of Grounded Theory listed in Table \ref{GroundedTheoryComparison} shown at the end of this section. There are several significant differences which are summarized here:
\begin{itemize}
\item The influence of research questions: emerges from the research (Classic Grounded Theory), may be defined upfront (Straussian Grounded Theory), may be defined upfront and evolves through study (Constructivist Grounded Theory)
\item The role of existing literature: delayed in the process (Classic Grounded Theory), used when needed (Straussian Grounded Theory, Constructivist Grounded Theory)
\item Theoretical Coding (Classic Grounded Theory, Constructivist Grounded Theory) versus Axial Coding (Straussian Grounded Theory)
\item Analytic questions: \quotes{what is this data a study of?} (Classic Grounded Theory, Constructivist Grounded Theory), asking questions of the phenomena such as why, how, and when (Straussian Grounded Theory)
\item Philosophical differences: objectivism (Classic Grounded Theory), pragmatism (Straussian Grounded Theory), and social constructionism (Constructivist Grounded Theory)
\item Evaluation criteria
\end{itemize}


% \begin{table}[h]
% \centering
% \renewcommand{\arraystretch}{1.5}
% \caption{Concise comparison of Grounded Theory Approaches}
% \label{ConciseGroundedTheoryComparison}
% \begin{tabular}{|p{1.3in}|p{1.5in}|p{1.5in}|p{1.5in}|}
% \hline
%                                     & Classic Grounded Theory                                                                              & Straussian Grounded Theory                                  & Constructivist Grounded Theory                                                                      \\ \hline
% The influence of research questions & emerges from the research                                                                            & may be defined upfront                                      & may be defined upfront and evolves through study                                                    \\ \hline
% The role of existing literature     & delays use of literature in the process                                                              & use when needed                                             & use when needed                                                                                     \\ \hline
% Analytic coding                     & Theoretical Coding                                                                                   & Axial Coding                                                & Theoretical Coding                                                                                  \\ \hline
% Analytic questions                  & \quotes{what is this data a study of?}                                                             & hypothesizing as to causes of the data                      & \quotes{what is this data a study of?}                                                            \\ \hline
% Philosophical differences           & objectivism                                                                                          & pragmatism                                                  & social constructionism                                                                              \\ \hline
% Evaluation criteria                 & fits the data, works in explaining main concern, relevance to participants, modifiable with new data & Seven criteria for process and eight criteria for grounding & credibility (enough data), originality, resonance with participants, usefulness with interpretations \\ \hline
% \end{tabular}
% \end{table}


Classic Grounded Theory emphasizes action and process. It looks at the question, \quotes{what is going on here?} and decomposes it into \quotes{what are the basic social processes?} and \quotes{what are the basic social psychological processes?} \cite{Charmaz}.

Straussian Grounded Theory relies on Axial Coding as a key differentiator from Glaser's Theoretical Coding. Axial Coding is a systematic process for looking at the relationship between codes for aiding in the emergence of the theory. Axial Coding answers the questions \quotes{when, where, why, who, how, and with what consequences} \cite{Strauss1988Basics}. Glaser \cite{GlaserBasics} felt that axial coding favored the six C's  (Causes, Contexts, Contingencies, Consequences, Covariances and Conditions) one of the eighteen theoretical families identified in \ul{Theoretical Sensitivity} without allowing emergence of the other theoretical families \cite{GlaserTheoreticalSensitivity}. Charmaz prefers keeping codes \quotes{simple, direct, analytic, and emergent} instead of applying axial coding which can distract the researcher from understanding what the data means by focusing on the process \cite{Charmaz}. Those in favor of Straussian Grounded Theory see Axial Coding as a more systematic process than Theoretical Coding. 
Constructivist Grounded Theory deviates from Classic Grounded Theory by embracing constructivism. Section \ref{Constructivism} explores how constructivism affects the research process. In addition to the differences listed above, the interview process is also different. 

Constructivist Grounded Theory encourages the researcher to record and transcribes interviews. Glaser feels that this slows down the research process, overwhelms the researcher with too many codes to analyze, and elicits properline data from the interviewee since they know they are recorded. (Properline data is when interviewees provided information that is socially acceptable for delicate matters. Social norms and corporate narratives reinforce properly aligned data. Glaser is much more interested in understanding what is actually happening. \cite{GlaserAllIsData}.) Charmaz describes Glaser and Strauss's interview approach as \cite{GlaserDiscovery} \quotes{smash-and-grab} by prioritizing the needs of the researcher over the needs of the participants. Instead, she recommends allowing the participant to set the tone and pace of the interview with the researcher matching and mirroring the participant's tone and pace. Constructivist Grounded Theory specializes in breaking down phenomena and reflecting on dialog with \quotes{what} and \quotes{how} questions.


\section{Interview Technique}
Interviews are the most common technique of gathering data in a Grounded Theory study. Grounded Theory interviews are open-ended explorations of the participant's perspective. Ideally, the interviewer does not force topics but merely follows the path of the interviewee. The interview initially starts with broad, open-ended questions to allow the emergence of unexpected narratives. 

Charmaz suggests \quotes{intensive interviews,} which are \quotes{open-ended yet directed, shaped yet emergent, and paced yet unrestricted} \cite{Charmaz}. Open-ended questions enter into the participant's personal perspective within the context of the research question. The interviewer attempts to abandon assumptions in order to understand and explore the interviewee's perspective. Charmaz \cite{Charmaz} contrasts intensive interviews with informational interviews (collecting facts), and investigative interviews (exposing hidden intentions, practices or policies).

The goal is for the researcher to understand the participant's views and actions. The researcher does not need to agree with those views and actions, just interpret them. The researcher can discover what the participants take for granted.

During the interview, the constructivist explores areas of theoretical interest when a participant mentions them. Opening questions slowly guide the conversation towards an area of interest. After understanding what the interviewee's perspective, the researcher can guide the session to provide more detail.

The constructionist endeavors to learn the meanings behind the participant's words or phrases, and not assume that the interviewer and interviewee share a common lexicon, world view, or shared understanding. The constructionist aims to decrease possible preconceptions by understanding the participants' world views. 

The interviewer balances the needs of entering into the interviewee's narrative with getting enough detail to understand the emergent theory. Charmaz argues against placing time limits on interviews. Sometimes it is hard for the interviewer to balance listening to the participant while exploring topics of interest. For example, I thought that one of my interviews was wasteful since the interviewee spent 16 minutes answering one single question. Several times during the interviewee's answer, I was tempted to interrupt. During data analysis, however, this segment turned into a treasure trove. The researcher's anxiety might lead to prematurely moving on and missing interesting data. 

Sometimes asking a follow-up question such as \quotes{could you tell me more or what did you mean by \underline{\hspace{0.8cm}}} will result in the interviewee provided additional, valuable information.

For organizational studies, Charmaz leads in with questions about collective practices and follows up with questions about the interviewee's participation and point of view. For emotionally charged topics, an interview should proceed with safer questions before diving into personally challenging ones. 

When dealing with hard to discuss topics, Charmaz suggests asking the participant \quotes{
could I ask you about \underline{\hspace{0.8cm}}?} instead of being direct with a question such as \quotes{tell me about \underline{\hspace{0.8cm}}?} \cite{Charmaz}. This question reduces the amount of control the interviewer has in the interview by empowering the interviewee.

When ending an interview, Charmaz prefers the question \quotes{is there something?} over a more traditional \quotes{is there anything?} The former assumes that there is something that the person should reveal. The later tends to signal that the interview is ending and tends to close the conversation. I tried the variant, \quotes{tell me one more thing} and it worked well for me.

A Grounded Theory ethnographer can use a recursive interviewing technique by interviewing a participant multiple times. Each subsequent interview resumes the previous conversation and deepens the relationship between the interviewer and the participant. 

Classic grounded theorists argue for note taking during the interview
so that the researcher documents the essentials without getting lost in the details \cite{GlaserTheoreticalSensitivity, GlaserGroundedTheoryPerspective}.  A constructivist finds value in the details. Constructivist grounded theorists argue that recording the interview enables the interviewer to focus on the narrative without worrying about capturing information during the interview. The transcription and coding process delays concerns about what is relevant and allows researchers to make these decisions at their leisure.  

I found that recording the interview, creating a transcript, and coding while listening to the audio helped me identify some of my preconceived ideas. During the coding process, while listening to audio recordings of the interview, I realized that I occasionally misunderstood what the participant said, interpreted what they said in my frame of reference,  or ignored something the participant said that I could have followed up on. Reviewing the interview audio enabled me to become a better interviewer as I now had a feedback loop. Since classic grounded theorists do not record the interview, this feedback loop is unavailable to them.

As the research progresses, the interview questions change and respond to the research direction emerging from Grounded Theory. This allows the researcher to direct the data collection and theoretical ideas incrementally.
To summarize, Charmaz describes an \quotes{intensive interview} process. The process involves open-ended questions. The purpose is for the researcher to enter into the participant's personal perspective within the context of the research question. The interviewer needs to abandon assumptions and personal presumptive to understand and explore that of the interviewee. % Charmaz contrasts intensive interviews from informational interviews which endeavor to collect accurate \quotes{facts} and investigative interviews that attempt to reveal hidden intentions or expose practices and policies. The participant sets the tone and pacing with the researcher matching. 

\textbf{Interview Guides:} An interview guide is an ordered list of sample questions that the interviewer may use to guide the interview. Interviewers should not see the guides as scripts. The interviewer needs to be present with the interviewee, listening to both verbal and non-verbal communication. The interview guide facilitates a semi-structured interview where the interview asks open ended questions designed to initiate a discussion and follow-up with appropriate questions that emerge from the discussion.

Charmaz advocates that novices rely on interview guides. Creating an interview guide causes the researcher to reflect on sequencing and building of questions, and it reduces the chance of preconceiving the data through leading questions. First, the interview guide serves as a forcing function for the researcher to consider what data they want to collect and the best questions to guide the participant towards those topics. The writing of a guide enables the researcher to consider both how and when to ask difficult questions. Second, relying on an interview guide will decrease the chances of a novice researcher blurting out an inappropriate question, in a moment of panic, or from accidentally asking a leading question. A leading question can bias the interviewee towards a particular point of view by how the question is framed. Grounded Theory emphasizes the emergence of information, not forcing the data through preconceived ideas. 

Experts tend to internalize the interview guide and follow where the conversion is leading them. The interview guide can help researchers remove their preconceived perspective which might taint questions. Reflecting on the interview guide helps the researcher accomplish their research objectives. 

While grounded theorists agree that researchers should not preconceive the data or the analysis, grounded theorists disagree on what techniques constitute \quotes{forcing} the data. Glasser \cite{GlaserIssues} argues against rules for proper memoing, interview guides, and units for data collection, whereas Charmaz argues that an open-ended interview guide for a semi-structured interview would help prevent novices from accidentally asking loaded questions. 

Charmaz crafts questions to elicit certain kinds of information. For example, \quotes{as you look back on your illness, which events stand out in your mind?} to have participants discuss time sequencing. Glaser might argue that this is forcing the data.  Charmaz believes that it opens up commonplace aspects of life.
\subsection{Interviewing Challenges}
There are potential barriers such as power imbalances, social norms, and prior knowledge, and forcing the data, that may affect the interview process and may affect what participants reveal. 

\textbf{The relationship of the researcher and the participant}, called \quotes{identity} by Charmaz, may affect the interview process. For example, power imbalances of a manager and employee or a professor and students.  Different kinds of strategies can be employed. A researcher could distance oneself from positions of authority: a professor interested in learning techniques may decide to interview students in a different degree program. To alter power imbalance, a domain expert researcher might offer \quotes{personal and professional views to encourage reciprocity} \cite{Charmaz}.

\textbf{Social norms and customs}, called \quotes{etiquette} by Charmaz. Participants might not want to reveal information to a stranger. A company rule of secrecy might make it difficult for a researcher to get the needed information. Framing a question is one way to overcoming this challenge. For instance, asking the question,  \quotes{some people have mentioned having negative pair programming sessions; has that happened to you?} normalizes a negative experience, potentially giving permission to the interviewee to discuss an uncomfortable situation.

\textbf{Prior knowledge} can aid or create difficulties for the researcher. A researcher with domain expertise can know about interesting research questions and how to acquire certain kinds of data. That same domain expertise could blind the researcher to possible explanations that an outsider might see.

\textbf{Preconceived ideas} can be used as starting points for open-ended research. A preconceived idea may inspire a researcher to select an area of passion as long as the data guide them. The key is not to force preconceived ideas onto the data. The researcher can initiate with an interesting idea, yet it is critical for the researcher to be flexible with the research topic, listen to the ideas of the participants, and discover the emergent theory. 

\textbf{Inaccurate interviews} could occur as it is possible for the participant to accidentally or intentionally fabricate their experience. In this case, the researcher may be entering an implicit collusion with their participants \cite{Yanos2008CollusiveObjectification}. The constant comparison phase should reveal that the interview is an outlier from the other participants.  Instead of treating such interviews as valueless, Grounded Theorists would examine the interchange trying to understand what is happening which may reveal information about forbidden topics and vulnerabilities of both parties. The after-the-fact analysis may help the researcher from repeating such a performance by understanding how the participant was redirecting the interview. 

\section{Field Notes for Participant Observation or Ethnographic Studies}
The researcher can record observations as field notes, either as a participant involved in the activity or as an ethnographer watching the activity. Recording field notes is complementary to interviewing. 

Field notes might
\begin{itemize}
\item ``Record individual and collective actions 
\item Contain full, detailed notes with anecdotes and observations 
\item Emphasize significant processes occurring in the setting 
\item Address what participants define as interesting or problematic 
\item Place actors and actions in scenes and contexts 
\item Become progressively focused on key analytic ideas.'' \cite{Charmaz}
\end{itemize}

Grounded Theory helps with participant observation and ethnographic studies. \quotes{Grounded theorists select the scenes they observe and direct their gaze within them. Their field-notes show the actions, processes, and events that constitute what is happening in the setting. Grounded Theory methods provide systematic guidelines for probing beneath the surface and digging into the scene. These methods help in maintaining control over the research process because they assist the ethnographer in focusing, structuring, and organizing it} \cite{Charmaz}.

For an ethnographer, the \quotes{explicit integration of observation and interviewing affords immediate materials for analysis} \cite{Charmaz}. Grounded Theory adds checks and rigor to the ethnographer's data collection and analysis. 

Charmaz provides several questions for ethnographers to understand what they are observing:
\begin{itemize}
\item ``What are people doing? Which patterns of actions or events do you discern?
\item What strikes you as most noteworthy, most interesting, or most telling?
\item How would you describe the setting( s)?
\item How do hierarchies affect individual and collective actions?
\item What do different participants / groups in the setting seek to accomplish?
\item What do participants' experiences mean to them?
\item How do participants use language?
\item On what criteria do participants and/ or groups judge actions, events, and products or outcomes?
\item To whom are participants accountable?.
\item How do participants explain their actions to each other?
\item How are material resources involved?
\item How do your understandings shift and change as your research proceeds?
\item What theoretical area does this ethnography address?'' \cite{Charmaz}
\end{itemize}

Collecting and analyzing field notes from participant observation is not easy. At the beginning of her research, Charmaz pictured a research experience where she would observe her research participants during the day and disappear into the coffee room to take detailed notes and write up her experience \cite{Charmaz}. She discovered that this is challenging in practice, as it can be difficult to disappear while observing activities. When the participant activity is intensive, such as the case with pair-programming, recording observations is tricky. I relied on small notes on post-it notes (which were culturally acceptable compared to typing on a laptop) and detailed observations after work.

\section{Documents and other sources of data}
Charmaz suggests that researchers can undervalue documents and encourages researchers to view them as text assets that researchers can mine in the same way as interview notes or field observations. Since the researchers didn't create the document, some might see them  as more \quotes{objective.} 

Charmaz identifies two categories of documents, extant documents which exist prior to the researcher's involvement (examples include stories, drawings, code, retro topics, corporate wiki pages, and contracts), and elicited documents which the researcher creates (the researcher interviews participants to capture an account of their experiences). With extant documents, the researcher needs to understand the context in which the document is situated. The researcher can use both types of documents to satisfy possible research goals. 

Lindsay Prior \cite{Prior2003UsingDocuments} argues that more than another voice, documents can do things that a participant can not. Beyond \quotes{what do they contain?} the researcher can ask the questions, \quotes{what does the document do?}, \quotes{why was it created?}, \quotes{how was it created?}, \quotes{how are the documents used?}, \quotes{how do people interpret the document?} and \quotes{what is not included in the document?} 
\section{Coding}
Coding is the process of taking new data and systematically labeling portions of the data that are relevant to the research study. For example, this portion of the transcript, \participantQuote{Sometimes I kind of feel like a janitor to it.  Maybe caretaker would be better.  Yeah, probably caretaker like I feel like this janitor just cleans up messes but a caretaker like also makes things better.} could be labeled with \quotes{caretaking the code by making it better,} \quotes{cleaning up messes,} and \quotes{dealing with technical debt.} 

The purpose of coding is to fracture the source data into pieces then compare the data against itself during the constant comparison activity. Coding begins the framing from which the researcher builds analysis. In Constructivist Grounded Theory, interviews are recorded, transcribed, and then coded.

Charmaz advises to code everything during the early stages of research and see what emerges. During the initial phase, the researcher should be open to any direction. Once emergent themes arrive, then coding in later parts of the research can be focused around the themes. The focused coding phase allows the researcher to \quotes{sort, synthesize, integrate, and organize large amounts of data} \cite{Charmaz}.

While, there is no perfect coding scheme,  there are different styles of coding \cite{Saldana2012}. Starting with Glasser in 1978 \cite{GlaserTheoreticalSensitivity} some grounded theorists argue for using gerunds (e.g. \quotes{dealing with uncertainty,} \quotes{exploring solutions to a problem}) instead of topic or noun-based coding schemes (e.g. \quotes{uncertainty,} \quotes{solutions}). Relying on gerunds helps encourage the researcher to dig into the data and see the relationship between the participant and their actions. Charmaz strongly argues against labeling events, experiences, or topics as codes as the researcher gains little insight into the participant's meaning, action, or point of view. Charmaz prefers a coding style that is simple, short, direct, advice, analytic and spontaneous. Both the researcher's and the participants' vocabulary influences the coding. Because Strauss and Corbin \cite{Strauss1988Basics} place less emphasis on this distinction, many grounded theorists take a more open stance to coding than relying on gerunds. Both Glaser and Charmaz do not want the researcher to bring preconceived ideas to coding.

There are different granularity of coding. Charmaz prefers line-by-line coding and suggests there are times when even word-by-word coding is helpful. The line-by-line coding helps the researcher slow down and examine for nuanced interactions in the data. By 1992, Glasser prefers topic-by-topic or incident-to-incident coding. He advises against decomposing a single incident. He feels that line-by-line coding produces too many codes, categories, and properties without necessarily supporting the analysis. While line-by-line coding does generate more codes, I appreciated the intimacy and thoroughness of line-by-line coding.

Charmaz suggests these heuristics to guide the researcher in the coding process:
\begin{itemize}
\item ``Remain open
\item Stay close to the data
\item Keep your codes simple and precise
\item Construct short codes
\item Preserve actions
\item Compare data with data
\item Move quickly through the data'' \cite{Charmaz}
\end{itemize}

Charmaz suggests these strategies when examining source materials:
\begin{itemize}
\item ``Breaking the data up into their component parts or properties
\item Defining the actions on which they rest
\item Looking for tacit assumptions
\item Explicating implicit actions and meanings
\item Crystallizing the significance of the points
\item Identifying gaps in the data'' \cite{Charmaz}
\end{itemize}

Whenever an insight arises or a \quotes{instantaneous realization of analytic connections} occurs, the researcher immediately stops the coding process and records the idea as a memo.

For the constructivist, the researcher creates the code when the researcher examines the data and finds meaning within it. \quotes{We \textit{construct} our codes because we are actively naming data \ldots We choose the words that constitute our codes. Thus we define what we see as significant in the data and describe what we think is happening} \cite{Charmaz}.

Sometimes a researcher sees everything as trivial or everything as significant. This is normal and in time, the data collection and coding process will sort out the core concerns of the participants. When observing routine activity, it can be challenging to see anything meaningful in the data. In this situation, the researcher can compare events to events looking for similarities and differences. 

If for some reason the codes remain mundane, Charmaz suggests \quotes{coding the codes.} In examining the codes, there may be a larger process or activity. There might be patterns in the codes. 

Tensions may emerge in the coding. The researcher should embrace tensions rather than avoid or hide them. Tensions may become properties of categories, differentiating key aspects of the data.

At the researcher codes the data, Charmaz encourages the researcher to answer these questions:
\begin{itemize}
\item ``What process(es) is at issue here? How can I define it?
\item How does this process develop?
\item How does the research participant(s) act while involved in this process?
\item What does the research participant(s) profess to think and feel while involved in this process? What might his or her observed behavior indicate?
\item When, why, and how does the process change?
\item What are the consequences of the process?'' \cite{Charmaz}
\end{itemize}

\quotes{Grounded theorists aim to code for possibilities suggested \textit{by} the data rather than ensuring complete accuracy \textit{of} the data} \cite{Charmaz}. This stance provides opportunities for checking envisioned ideas with other data. Constant comparison allows the researcher to invalidate conjecture; an errant code will be detected and unsupported by other data samples. Ideas reflected in the data must earn their way into subsequent analysis. Constructivist grounded theorists acknowledge that both the researcher's and the participants' vocabulary influences the coding.

There are two phases of coding: initial coding and focused coding. 

The initial coding is the process of coding the first set of data. The initial coding names every part of the data. Initial coding is an intimate process for the researcher as the researcher becomes familiar with the data. The researcher can examine the data looking for implied meanings. The emphasis is on summarizing the data and not yet analyzing it. The researcher tries to avoid asking what the data means at this stage. Both actions and processes can be coded. After starting initial coding, the researcher begins using constant comparison described in the next section. Constant comparison helps shape the direction of the research. 

Focused coding is the second phase of coding. Once core categories emerge from constant comparison, the researcher shifts from initial coding to focused coding where the emphasis is on flushing out the core categories and their properties. While it is hard to predict when the transition will occur, Charmaz sees a natural transition when the researcher simply starts performing focused coding. Focused coding continues until theoretical saturation occurs, as described in the section on theoretical sampling.

As a reminder, the data advances from the initial codes to focused codes, focused codes to core categories, and core categories to an emergent theory. 

\section{Constant Comparison}
Constant comparison allows the researcher to identify \quotes{the conceptual relationship between categories and their properties as they emerged} \cite{GlaserBasics}, leading to an emergent theory.

In constant comparison, the researcher examines and compares codes to codes. It may be the case that two codes should be combined since they are describing the same phenomena. Codes that are related form a category. The researcher compares category to category looking for the relationship between them. The researcher periodically audits each category for cohesion by comparing its codes and moves codes that belong to a different category. Constant comparison compares incident to incident looking for patterns. Similarities strengthen the category while differences refine properties of the category. 

As categories emerge from the data, the researcher is looking for the core category as it defines the chief concern of the participants. Once the core category emerges, the researcher continues to collect and analyze data by examining the properties of the core category and its relationship to other categories until theoretical saturation occurs.

The researcher pauses to record memos for any insights that occur during the analysis.
\section{Memo-writing}
Memo-writing is writing down the researcher's insights and analysis. Memoing can occur at any point in the process, and preempts any other activity.  \quotes{Memo-writing is the pivotal intermediate step between data collection and writing drafts of papers} \cite{Charmaz}. 

Memo-writing pauses the other research activities and allows the researcher to reflect on what is going on in the data and analysis. Memo-writing encourages the researcher to analyze the data early in the process. The output of memo-writing can enable the researcher to adjust the course of the research process.

Memos written early in the research process may be more tentative and less theoretically developed than memos written later in the research process. 

Charmaz encourages the researcher to \quotes{engage an emergent category, let your mind rove freely in, around, under, and from the category; and write whatever comes to you. That is why memo-writing forms an interactive space and place for exploration and discovery. You take the time to discover your ideas about what you have seen, heard, sensed, and coded and then you examine these ideas} \cite{Charmaz}.

Some researchers use a \quotes{Methodological Journal} in which the researcher records dilemma, directions, and decisions. A journal helps the researcher reflect about the data and the process.  The key is for the researcher to find a system that works for them.

For early memos, Charmaz recommends that researchers \quotes{record what they see happening in the data} and to {use early memos to explore and fill out qualitative codes} \cite{Charmaz}. 

For memos later in the process, Charmaz suggests to
\begin{itemize}
\item ``Trace and categorize data subsumed by the topic.
\item Describe how the category emerges and changes. 
\item Identify the beliefs and assumptions that support it. 
\item Specify how the category informs action and experience. 
\item If relevant, tell what it looks and feels like from various vantage points. 
\item Place the category or categories within an argument. 
\item Sharpen comparisons of people, data, codes, categories, subcategories, concepts, and analysis with existing literature.'' \cite{Charmaz}
\end{itemize}

There is no standard, correct memo. Contents vary from memo to memo. Memos can have a title. Memos tend to be private, unshared documents enabling the researcher to be perfectly free with thoughts, not worrying about grammar, editing, and future readers. The paper writing process will sort all that out. As such, researchers can record doubts, concerns, and first thoughts. Charmaz lists several ways researchers can use a memo:
\begin{itemize}
\item ``Define each code or category by its analytic properties
\item Spell out and detail processes subsumed by the codes or categories
\item Make comparisons between data and data, data and codes, codes and codes, codes and categories, categories and categories
\item Bring raw data into the memo
\item Provide sufficient empirical evidence to support your definitions of the category and analytic claims about it
\item Offer conjectures to check in the field setting(s)
\item Sort and order codes and categories
\item Identify gaps in the analysis
\item Interrogate a code or category by asking questions of it.'' \cite{Charmaz}
\end{itemize}

In order to facilitate memo-writing, Charmaz suggests relying on several writing exercises to stimulate the writing process such as freewriting (keep writing whatever comes to the mind) and clustering (diagramming relationships).

Memo-writing can raise focused codes to conceptual categories. As the researcher does the analysis on focused codes, the researcher identifies which codes describe what is occurring in the data. A category can span several codes. 

Both Glaser and Charmaz desire conceptual categories with \quotes{abstract power, general reach, analytic direction, and precise wording} \cite{Charmaz} while grounded in the data. Conceptual categories may be applicable in different domains, professions, and fields and help explain substantive processes. Examples include \quotes{getting off the street} \cite{KarabanowGettingOffTheStreet}, \quotes{managing \textit{spoiled} national identity} \cite{RiveraManagingSpoiledNationalIdentity}, and \quotes{living one day at a time}. Focusing on generic processing raises codes to theoretical categories. 

As the key analytic step, memos become the foundation of the emergent theory.  The researcher may deem some memos as complete whereas others raise unanswered questions. Revisiting previous memos allows the researcher to see what is missing in the memo and directs new research directions. Memo-writing helps the researcher to identify additional data to collect. 

\section{Theoretical Sampling}
Theoretical sampling is collecting additional data to develop full and robust categories, identify the relationships between categories, and flush out the main category's properties. 

For example, data and codes may yield unanswered questions and categories may not be definitive and may suggest new avenues of exploration. Additional data collection, coding, and analysis refine the emergent theory which produces a new vantage point for further exploration and refinement. Memo-writing encourages theoretical sampling as the researcher reflects on the data and the analysis causes new questions to emerge. 

Activities include adding new participants, observing in different settings, re-interviewing participants with follow-up questions or ask about different kinds of experiences.

This process continues until the category structure stabilizes and nothing new can be added to the theory, i.e. theoretical saturation. 

Theoretical sampling is not 
\begin{itemize}
\item ``Sampling to address initial research questions 
\item Sampling to reflect population distributions
\item Sampling to find negative cases
\item Sampling until no new data emerge.'' \cite{Charmaz}
\end{itemize}

There's a subtle distinction between \quotes{sampling until no new data emerges} and sampling to flesh out the emerging theory. In discussing theoretical sampling or \quotes{saturation} with software engineering researchers practicing Grounded Theory, I misconstrued that theoretical sampling is the continued collection of data until no new data and thus no new insights emerge. Stopping makes sense if the collection of data results in the same kind of information. However, theoretical sampling has a subtle distinction from this misconception. Grounded Theory is not about asking the same questions over and over again until the same questions result in no new information. Grounded Theory alters the questions in response to the emerging data, and the researcher continues to ask new questions to help fill in the emerging theory. Thus, theoretical sampling is collecting new information to illicit the relationship between codes, between codes and categories, and between categories. The process stops once the researcher has matured the theory, and there is nothing more to ask the participants. 

% For a researcher studying software engineering waste, theoretic sampling is not the continued collection of more and more waste until nothing new emerged, but asking new questions to tease out the how the waste categories relate to each other.

%\strikeout{While Quantitative Researchers aim to have a broad sampling to statistical inferences that describe target populations, qualitative researchers aim to fit their theory to their data. Quantitative researchers aim to test preconceived hypotheses; qualitative researchers aim to generate emergent theories that then become the foundation for new endeavors for quantitative researchers.}

In review, theoretical sampling is collecting additional data to develop full and robust categories, identify the relationships between categories, and flush out the core category's properties.

\subsection{How much data to collect?}
The researcher continues to gather data until theoretical saturation occurs. At the beginning of a research study, the researcher can not predict when saturation will happen. Thus, there are no useful metrics for how much data to collect. Likewise, it is challenging for a reviewer to determine if a study has collected enough data. Clearly, more data is better than less data. Charmaz points out asking how many interviews is enough is asking the wrong question. She would prefer to have enough interviews that strengthen the research with deep and sufficient vigor. 

Studies using mixed research methods may require fewer interviews, as the study relies on data from different data sources.

Data should give a full picture. Two key aspects of data for depicting empirical events are suitability and sufficiency.  The researcher needs to plan to gather sufficient data. 

For Glasser \cite{GlaserIssues} and Stern \cite{SternErodingGroundedTheory}, small data samples, and limited data is not an issue since the purpose of Grounded Theory is to create conceptual categories. Charmaz argues that limited data can lead to weak analysis. 
\section{Memo Sorting}
Memo sorting involves the sorting, comparing, and integrating of the memos to refine the emergent theory. Sorting involves examining different arrangements of the memos to determine which sequence clearly tells the story of the theory. Often memos describing properties of a category are sequenced around the category. If there is a time sequence to a process, memos might be sorted chronologically with the steps of the process. Charmaz suggests printing them out, arranging them on the floor or a dining room table, and rearranging them until a natural progression emerges. Comparing involves juxtaposing two memos and seeing if new insights and thus a new memo emerges. Integrating involves fitting the memos together into a cohesive whole. 

Thus memo sorting is more than simply creating the first draft. The process of memo sorting stimulates additional analytical work, may raise new questions that may stimulate additional data collection, coding, and analysis. 

The memo sorting phase begins the process of helping the researcher articulate the theory.

\section{Theory Construction}
For researchers who do want to generate theory, Charmaz suggests that the researcher attends to four concerns: theoretical plausibility, direction, centrality, and adequacy. These concerns arise because the researcher can direct and control the data generation, and thus the emerging theory. Although decomposed into four concepts, the key idea of some studies may span several of these concerns at the same time.

While Charmaz does not define these terms in her book, the following definitions are gleaned by the contextual usage of each team.

\textbf{Theoretical plausibility} reflects whether an idea might turn into a theory. Often key interview statements might represent some larger notion. The researcher treats repeated themes as theoretically plausible.

\textbf{Theoretical direction} means that the data and the analysis are pointing the researcher towards a particular path. After the initial interviews and the initial coding, a theoretical direction emerges from the data. Certain ideas and codes routinely emerge from the data. The data suggests paths that need exploring. Without picking a direction based on the emergent data, the researcher could languish in the early stages of research without iterating towards an emergent theory. 

\textbf{Theoretical centrality} is the researcher defining what is core to the research. Once the researcher has explored these paths, theoretical centrality emerges from the coding and analysis. Certain codes become core to the research. The researcher drops less fruitful paths and codes as the researcher focuses the interviews on the central theme or categories. 

\textbf{Theoretical adequacy} is the assessment of the robustness of the emergent theory in its categories and properties as verified in theoretical sampling. During later interviews, the researcher introduces questions for theoretical sampling and to gauge the robustness of the emergent categories. Theoretical adequacy is central to theoretical sampling and saturation. 

For the constructivist, theoretical plausibility supersedes interviewee accuracy. Charmaz reminds that definitions of accuracy are social constructs. Charmaz argues that the amount of data collected in a Grounded Theory study typically offsets any adverse effects of \quotes{misleading accounts} and thus decreases the probability that the research's work would have spurious results. \quotes{Grounded Theory aims to make patterns visible and understandable} \cite{Charmaz}. Thus a grounded theorist should strive to have wide and deep coverage of their categories. 

If a participant does offer exaggerated or inaccurate accounts, and the researcher detects this situation, then this can be a research opportunity into understanding how the candidate is creating fictional representations of their situation. Charmaz recounts seasoned citizens retaining identity patterns from earlier in their life; for example, one person described how she daily attended her garden even though she had not done so for years.


Glaser's chief goal is in theory building and routinely emphasizes conceptualization. Theoretical coding requires the researcher to think about the relationship between the codes. Coding families, groups of similar codings, help the researcher consider possible relationships between codes.  In his book, Theoretical Sensitivity \cite{GlaserTheoreticalSensitivity}, Glaser lists 18 theoretical coding families and acknowledges that there may be more:
\begin{enumerate}
\item ``\textit{The Six C's:} Causes, Contexts, Contingencies, Consequences, Covariances and Conditions
\item \textit{Process:} Stages, staging, phases, phasings, progressions, passages, gradations, transitions, steps, ranks, careers, orderings, trajectories, chains, sequencings, temporaling, shaping and cycling.
\item \textit{The Degree Family:} Limit, range, intensity, extent, amount, polarity, extreme, boundary, rank, grades, continuum, probability, possibility, level, cutting points, critical juncture, statistical average (mean, medium, mode), deviation, standard deviation, exemplar, modicum, full, partial, almost, half and so forth.
\item \textit{The Dimension Family:} Dimensions, elements, division, piece of, properties of, facet, slice, sector, portion, segment, part, aspect, section. The dimension family divides the notion of a whole into a parts. 
\item \textit{Type Family:} Type, form, kinds, styles, classes, genre. While dimensions divide up the whole, types indicate a variation in the whole, based on a combination of categories. 
\item \textit{The Strategy Family:} Strategies, tactics, mechanisms, managed, way, manipulation, maneuverings, dealing with, handling, techniques, ploys, means, goals, arrangements, dominating, positioning. 
\item \textit{Interactive Family:} Mutual effects, reciprocity, mutual trajectory, mutual dependency, interdependence, interaction of effects, covariance. 
\item \textit{Identity-Self Family:} Self-image, self-concept, self-worth, self-evaluation, identity, social worth, self-realization, transformations of self, conversions of identity.
\item \textit{Cutting Point Family:} Boundary, critical juncture, cutting point, turning point, breaking point, benchmark, division, cleavage, scales, in-out, intra-extra, tolerance levels, dichotomy, trichotomy, polychotomy, deviance and point of no return.
\item \textit{Means-goal Family:} End, purpose, goal, anticipated consequence, products.
\item \textit{Cultural Family:} Social norms, social values, social beliefs, and social sentiments.
\item \textit{Consensus Family:} Clusters, agreements, contracts, definitions of the situation, uniformities, opinions, conflict, dissensus, differential perception, cooperation, homogeneity-heterogeneity, conformity, nonconformity, and mutual expectation.
\item \textit{The Mainline Family:} Social control (keeping people in line), Recruitment (getting people in), Socialization (training people for participation), Stratification (sorting people out on criteria which rank them), Status Passage (moving people along and getting them through), Social organization (organizing the people into groups, aggregates and divisions of labor) and Social Order, (keeping the organization of life working normatively), Social institutions (clusters of cultural ideas), Social interaction, (people acting with people), Social worlds (symbolic surround of life), Social mobility (patterned paths of people movement through society) and so forth.
\item \textit{Theoretical Family:} Parsimony, scope, integration, density, conceptual level, relationship to data, relationship to other theory, clarity, fit, relevance, modifiability, utility, condensability, inductive-deductive balance and inter-feeding, degree of, multivariate structure, use of theoretical codes, interpretive, explanatory and predictive power, and so forth.
\item \textit{Ordering or Elaboration Family:} Structural, temporal and generality are the three principal ways to order data.
\item \textit{Unit Family:} Collective, group, nation, organization, aggregate, situation, context, arena, social world, behavioral pattern, territorial units, society, family, etc., and positional units: status, role, role relationship, status-set, role-set, person-set, role partners. 
\item \textit{Reading Family:} Concepts, problems, and hypotheses. 
\item \textit{Models:} Another way to theoretically code is to model the theory pictorially by either a linear model or a property space.'' \cite{GlaserTheoreticalSensitivity}
\end{enumerate}

\section{Evaluation}
Glaser provides the criteria of fit, work, relevance, and modifiability for determining how well an emergent theory explains the data \cite{GlaserTheoreticalSensitivity}. Stol et al.  \cite{StolGroundedTheory} explain these criteria as:
\begin{itemize}
\item \textbf{Fit}: \quotes{The generated categories must fit the data} 
\item \textbf{Work}: \quotes{It must be able to explain or predict what will happen}  
\item\textbf{Relevance}: \quotes{The theory must have relevance to the action of the area} 
\item \textbf{Modifiability}: \quotes{The theory must be modifiable as new data appear}
\end{itemize}

Charmaz provides the criteria of credibility, originality, resonance, and usefulness. Stol et al.  \cite{StolGroundedTheory} summarizes these criteria as:
\begin{itemize}
\item \textbf{Credibility}: \quotes{Is there sufficient data to merit claims?} 
\item \textbf{Originality}: \quotes{Do the categories offer new insights?}  
\item\textbf{Resonance}: \quotes{Does the theory make sense to participants?} 
\item \textbf{Usefulness}: \quotes{Does the theory offer useful interpretations?}
\end{itemize}

\section{Tool Support}
Some qualitative researchers prefer using qualitative research tools such as NVivo and Atlas.ti for the process of coding transcriptions and facilitating analysis of codes. Many grounded theory researchers prefer using word processing or spreadsheet software. Before computers, Glaser recommended using carbon copy paper so that one copy of the transcripts and memos could be cut and rearranged for analysis while the other copy preserves the original data. The researcher needs to select the tools that facilitate their workflow and provide easy access to the data they need at each step of the Grounded Theory research process.

While NVivo 11.0.0 for Mac supports adding codes to a transcript,  there is no keyboard shortcut to add a code, and there is no auto-complete for existing codes in the system. The researcher needs to remember all the codes already added to the system. The cognitive overhead of remembering when to add a new code or re-use an existing code could be too much. The tool might be good for the focused coding with a fixed set of codes. The interface for peer reviewing codes was extremely challenging to use. 

Atlas.ti 1.0.24 has keyboard shortcuts and auto-complete suggestions for existing codes thus creating a better initial coding workflow. However, there is no way to edit the source material. While coding a transcript, if the researcher notices a mistake in the transcription, there is no way to fix it.

Microsoft Word is simple to use for transcriptions and initial coding with a three column format.  The first column contains a unique id reference for that row for the purpose of cross-referencing during constant comparison. The second column is for initial coding and focused coding. The third column is for the transcribed interview, breaking sections into new rows. Recording timestamps for each segment allows for cross referencing with the original audio. 

The researcher can export the word document into a comma separated file with the id and coding columns. The researcher then imports the file into a spreadsheet for constant comparison and analysis. Sometimes constant comparison can be easy to do in the spreadsheet. Sometimes it is easy to print index cards for physically sorting the cards around the researcher. The researcher can use mail merge to print specific rows into a Microsoft Word template for Avery shipping labels. After printing the labels, the researcher applies the sticker to index cards. If desired, colored index cards can be used to distinguish each data source. After sorting and clustering, the researcher would need to update the spreadsheet to match the physical arrangement of cards.

Any system could work for a researcher provided that the system facilitates coding and constant comparison. 

%I initially tried the qualitative research tools NVivo and Atlas.ti, but found that instead of making me more intimate with the data, the user interface design separated me from the data. I found digital forms of simple techniques to more effective. I relied on Microsoft word for transcripts, and Google spreadsheets for constant comparison, printing to index cards for key or massive constant comparison.

%At the outset of the research study, I was concerned about efficiently moving through my data. Using a spreadsheet to track constant comparison and flipping back and forth to various word documents, I was worried that it would be inefficient. A tool would allow me just to click on a link. In practice, however, this proved to be not an issue.


\section{Constructivism}
\label{Constructivism}
The Constructivist approach to Grounded Theory \quotes{emphasizes understanding and acknowledges that data, interpretations, and resulting theory depend on the researcher's view} \cite{StolGroundedTheory}. 

The constructivist acknowledges that the \quotes{humanness} of the researcher may affect the research. Constructivists attempt to identify their assumptions and not accidentally reproduce the assumptions in the research results. Relying on and listening to the participants helps counteract researcher bias.

The interplay between the researcher and the participant during an interview is particularly interesting from a constructivist perspective. Interviews are situated within a particular context. Changes in location, time, and setting would result in different data.  Charmaz argues that interviews are a construction between the interviewee and the interviewer. \quotes{The result is a construction - or reconstruction - of reality. Through constructing their respective performances, interviewers and interview participants present themselves to each other. However silent, both the interviewer's and participant's performances make and negotiate identity claims} \cite{Charmaz}.

Because Charmaz has a constructivist perspective, she acknowledges that \quotes{neither observer nor observed come to a scene untouched by the world} \cite{Charmaz}. The research method affects the kind of data observed, the researcher's background affects the observations that the researcher can see. The onus is on the researchers to bring scrutiny and reflective practice to understand how their point of view may bias observations. Charmaz points out that \quotes{\textit{people} construct data} through field notes and interviews. While it is tempting to treat an extent report or document as fact, constructivists remind themselves that people collected and formed them.

The contrasting positions of constructivism and positivism made me wonder about my personal stance and how it might affect the research process. While I believe that there are absolute facts to be obtained in fields such as mathematics and the sciences, understanding people is a messy, organic process. For software engineering research, Constructivist Grounded Theory seems aptly suited since software development is a socio-technical endeavor.



%\section{Similarities with Agile Software Development and User research}
%As an iterative research approach, Grounded Theory shares similarities to agile software development. Both embrace the need for change and start without knowing the final destination. In both, we plan only the next step and adjust our course as we acquire new knowledge and a better understanding of our environment. Just like agile, we need to grow comfortable with ambiguity. 

%The approach is similar to Pivotal's Discovery and Framing process for user research. Interaction designers interview prospective users of the system, attempting to create and validate a persona(s). During each interview, the interaction designer writes notes on post-it notes. After a few interviews, the team does a dump and sort, creating buckets by placing post-it notes into groups and performing constant comparison of each bucket. Key insights are noted. User research is a pragmatic version of Grounded Theory that allows the researcher to explicitly test preconceived ideas and assumptions using explicit questions to force certain topics into the conversation.  Grounded Theory prefers much more open-ended questions that do not force the data. User research tends to rely on a smaller number of interviews to achieve key insights. 
User research is a pragmatic process attempting to identify easy-to-find insights, not generate an exhaustive theory, and thus fewer interviews seem necessary.



\section{Summary}
Starting a Grounded Theory research study, the researcher asks the core question \quotes{what is happening here?}  \cite{GlaserTheoreticalSensitivity}

Grounded Theory allows the researcher to include new aspects of the research while gathering data, even late in the analysis. The data collection process is guided by the research and altered as knowledge is accumulated. It is common for early research to illuminate new angles. \quotes{Grounded Theory can give you flexible guidelines rather than rigid prescriptions} \cite{Charmaz}. The theory provides a starting place for inquiry, not a specific goal known at the beginning of the research. The researcher's guiding interests can be used as points of  \quotes{departure to form interview questions, to look at data, to listen to interviewees, and to think analytically about the data} \cite{Charmaz}.

After interviewing, data analysis begins with line-by-line coding as recommended by Charmaz \cite{Charmaz}. Coding line-by-line helps the researcher identify nuanced interactions in the data and avoid jumping to conclusions. The data then advanced from these initial codes to focused codes, focused codes to core categories, and core categories to an emergent theory. 



% \input{state-of-the-art}
\input{research-method}
\input{sustainable-software-development}
\chapter{Engendering Team Code Ownership}
\label{TeamCodeOwnershipChapter}


\section{Summary}


\textit{Context:} This chapter examines team code ownership, a core category of our theory of sustainable software development presented in the previous chapter. Team code ownership is a software development practice where any team member can modify any part of the team's code. However, many factors beyond official policy affect a developer's sense of ownership. 


\textit{Objective:} The purpose is to understand the factors that affect a team's sense of code ownership.


\textit{Method:} Following Constructivist Grounded Theory, we conducted a \durationOfResearchStudy{} participant-observation of \numberOfObservedProjects{} software development projects, and interviewed \numberOfInterviews{} software engineers, interaction designers, and product managers.  When team code ownership emerged as a core category in our study, additional data was collected to investigate factors associated with perceived code ownership and related phenomena.


\textit{Results:} Team code ownership is a feeling. Developers feel team code ownership more when they understand the system context, have contributed to the code in question, perceive code quality as high, believe the product will satisfy the user needs, and perceive high team cohesion.  


\textit{Limitations:} Outcomes of grounded theory research are not statistically generalizable to defined populations, and may not apply to organizations with different software development cultures.


\textit{Conclusion:} Team code ownership is rooted in numerous cognitive, emotional, contextual and technical factors and cannot be achieved simply by policy. 


\section{Introduction}
\textit{Team Code Ownership} (which is similar to \textit{Collective Code Ownership} and \textit{Shared Code}) is a software development practice where any developer on a team has the right to change any of the team's code. Team code ownership is intended to accelerate development by allowing any developer to fix any team bug and by mitigating delays due to vacations, illness, and other absence \cite{BeckExtremeProgramming2004}.
 
While some research has investigated the effects of different code ownership models, we are unaware of any studies that specifically investigate developers' sense of team code ownership; that is, the complex interactions between developers' knowledge, emotions, and approach to code ownership.  


When team code ownership emerged as a core category of our grounded theory study, additional data was collected to investigate factors associated with perceived code ownership and related phenomena. 


Participant observation quickly revealed that having the right to change a file does not mean that a specific developer will feel empowered to and justified in making a specific change. For example, a developer may feel reluctant to change code that he or she does not really understand. As the research refined this core finding and drove further data collection, five factors affect feelings of team code ownership. 


This chapter consequently reviews existing research connected to team code ownership (Section \ref{TeamCodeOwnnershipRelatedWork}), describes some aspects of the grounded theory approach related to code ownership (Section \ref{TeamCodeOwnershipResearchMethod}), and presents the emerging results: five factors associated with team code ownership (Section \ref{TeamCodeOwnership}). Section \ref{Discussion} discusses the study's implications and limitations, followed by a summary of its contributions (Section \ref{TeamCodeOwnershipConclusion}).
\section{Related Work}
\label{TeamCodeOwnnershipRelatedWork}


\subsection{Team Code Ownership}
In Extreme Programming \cite{BeckExtremeProgramming2004}, Kent Beck describes a set of interdependent practices for managing feature development and facilitating a collaborative team environment. One of these practices is \textit{collective ownership}---\quotes{Anyone can change any piece of code in the system at any time.} \cite{BeckExtremeProgramming1999}. The book contrasts collective ownership against \quotes{no ownership} and \quotes{individual ownership.} In 2004, collective ownership is renamed \textit{shared code} \cite{BeckExtremeProgramming2004}.


In 2006, Martin Fowler defined \textit{collective code ownership}, similarly to Beck \cite{FowlerCodeOwnership}, as a contrasting team position to \quotes{strong code ownership} where each file has one owner and \quotes{weak code ownership} where developers can change files, but an owner keeps an eye on files for which they are responsible. 


Later, Bird et al. \cite{BirdDontTouchMyCode} contrasted the effects of strong- and weak-ownership. They demonstrated that weak ownership leads to more defects than strong ownership for Windows Vista and Windows 7. The study defined ownership for a software component as a percentage of the version control commits for a single developer. They defined a major contributor as someone who has more than 5\% of the git commits. A sensitivity analysis revealed that defining strong code ownership within the range from 2\% to 10\% produced similar results for the study.


Meanwhile, Murphy  \cite{MurphyIEEESoftware} argued that the concept of code ownership must be unpacked and expanded. He argued that the complexities of code ownership are missed by merely examining git commits to determine who modified which files.


This research defines \textit{team code ownership} as \quotes{the ability for any developer on a team to change any of the team’s code.} For small systems and teams, \textit{collective code ownership} and \textit{team code ownership} are practically synonymous since a member of a four person team usually has the ability to modify any part of the team's code. For a large system with multiple teams, in practice, teams would have strong ownership of their portion of the system. Allowing any pair to modify any part of Microsoft Windows or Pivotal Cloud Foundry is impractical. For a large system with poorly defined team boundaries, it is possible for all the teams to adopt \textit{collective code ownership} without achieving \textit{team code ownership} as one team's changes to the code base may negatively affect other teams.


Team code ownership requires more than a team saying, \quotes{everyone can modify anything.} Instead, this research examines how a team feels that they own the code. This research defines \quotes{sense of team code ownership} as the degree to which individual members of the team feel collective ownership.  


\subsection{Psychological Ownership}
Psychological ownership refers to \quotes{the feeling of possessiveness and of being psychologically tied to an object} \cite{Pierce2001}. Targets of ownership, whether physical or immaterial, become the extension of one's self: \quotes{What is mine becomes (in my feelings) part of ME} \cite{Isaacs1933}. Ownership can be attached to a part or the whole. Psychological ownership occurs when the object becomes part of the psychological owner's identity. Psychological ownership answers the question, \quotes{What do I feel is mine?}


Changes in ownership can have strong effects on our self-identity. An increase in the number of possessions can produce positive effects \cite{Formanek1994}, while a diminish can lead to a personality shrinkage \cite{James1890}. Someone threatening a person's ownership can trigger strong emotions and responses.


Peirce \cite{Pierce2001} identifies three sources or \quotes{roots} of psychological ownership: efficacy and effectance, self-identity, and having a place. A major reason for possession of physical goods or abstract ideas is rooted in the innate human desire to be in control; being able to alter one's environment creates feelings of efficacy and pleasure. Ownership fulfills the need for self-identification as people define themselves, express themselves, and ensure their own survival by what they own. Ownership fulfills the need to have a place and a territory to possess. \quotes{Each motive facilitates the development of psychological ownership, rather than directly causes this state to occur.} Psychological ownership occurs with code because creating software can satisfy the desire for efficacy and effectance, self-identity, and having a place.


Peirce identifies three paths or \quotes{routes} to ownership: controlling the target, coming to intimately know the target, and investing the self into the target. With \emphasis{controlling the target}, targets that can be controlled are perceived to be part of the self.  As  individuals repeatedly exercise control of an object, eventually this leads to \quotes{feelings of ownership toward that object.} The higher the autonomy of the job task, the more likely ownership develops toward the activity. When a person has little control over an activity, psychological ownership is unlikely to develop. With \emphasis{coming to intimately know the target}, the association with the object creates feelings of ownership. One example is when a gardener feels that the garden belongs to the gardener. (This happens routinely with software developers who feel that they own part of the code base, when in reality, the company owns the software.) Feelings of ownership increase as one becomes intimately familiar with the object and associated with it. With \quotes{investing the self into the target,} people feel that they own what they create, shape or produce. Spending time, energy, and effort enables people to alter their view of themselves to include identity with the object. The more investing in the object, the stronger the psychological ownership. Nonroutine, complex jobs infuse more of individual's ideas resulting in increased ownership.


\section{Research Method}
\label{TeamCodeOwnershipResearchMethod}


The research method for this chapter is identical to the research method in Chapter \ref{SustainableSoftwareDevelopmentChapter}. When team code ownership emerged as one of the core categories of the Theory of Sustainable Software Development, additional data was collected in order to identify the factors affecting the sense of code ownership. As an example for system context factor, Table \ref{TeamCodeOwnershipChainOfEvidence} shows some example quotes for potential threats that can erode team code ownership for the system context factor. Section \ref{TeamCodeOwnership} introduces the factors (system context, code contribution, code quality, product fit, and team cohesion) and are the main contributions for this chapter.  


\begin{table}[t]
\renewcommand{\arraystretch}{1.5}
\centering
\caption{Examples for Threats for System Context}
\label{TeamCodeOwnershipChainOfEvidence}
\begin{tabular}{|p{\twoColumnWidth{}}|}
\hline
Threat: Increasing knowledge silos  \\
\hline
Sharing knowledge around whole team is important.\\ 
\quotes{Knowledge silos are a red flag for me. For example, we have some siloing around certain parts of the app. Most of it is around the big pain points.} \\ \quotes{I feel that we don't have that context spread around fully.} \\
Hesitancy to jump onto stories without context.\\ 
\quotes{It does make me not completely comfortable to jump into stories on certain aspects of the system. If I was on a story that's going to deal with some tricky part of the system, I would want to be paired with somebody who had more traction on it.} \\
\hline
Threat: Increasing team size              \\
\hline
With a big team, keeping shared context is challenging as so much parallel work is being done. \\ 
\quotes{Having five, sometimes six pairs on the project means the team is making significant progress each week. It is hard to keep context.  If you spend a week on one part of the system, the other pairs are changing the other parts of the system. When you get back to some other place, you don't know what has changed. Because of that speed, it's harder to keep context on everything.}  \\
\hline
Threat: Increasing code base size \\
\hline
Code continues to expand.\\
\quotes{The code is complicated, it is expanding a lot so in my head. I'm thinking like a big bang, where it starts out very small and now it just keeps expanding and growing. We're still adding a bunch of features.}  \\
\hline
\end{tabular}
\end{table}



% Following Charmaz' approach to Grounded Theory \cite{Charmaz}, this research used two primary data sources: field notes collected during continuous participant observations of a 7.5-month project and interviews with \numberOfInterviews{} Pivotal software engineers, interaction designers, and product managers. The initial core question was: \quotes{What is happening at Pivotal when it comes to software development?} This question led to the Theory of Sustainable Software Development presented in Chapter \ref{SustainableSoftwareDevelopmentChapter}. When team code ownership emerged as one of the core categories of the theory, additional data was collected in order to identify the factors affecting the sense of code ownership. The factors are introduced in Section \ref{TeamCodeOwnership} and are the main contributions for this chapter.


%\subsection{Data Collection}
% The primary researcher relied on \quotes{intensive interviews,} which Charmaz summarizes as \quotes{open-ended yet directed, shaped yet emergent, and paced yet unrestricted} \cite{Charmaz}. The technique relies on open-ended questions. The purpose is for the researcher to enter into the participant's personal perspective within the context of the research question. 


% While exploring new emergent core categories, whenever possible, the researcher initiated subsequent interviews with a goal of not forcing the issue. For example, \quotes{please draw your feelings about the code} often resulted in conversations about code ownership. After the interview, the interview was transcribed into a Word document with timecode stamps for each segment.


% The primary researcher collected field notes while working as an engineer. The field notes comprise multiple paragraph entries recorded several times a week collected over a six month period. The notes describe individual and collective actions, captures what participants defined as interesting or problematic, and include anecdotes and observations. 


% \subsection{Research Context: Pivotal}
% \label{TeamCodeOwnnershipResearchContext}
% Pivotal is a large American company with 16 offices around the world. One of its divisions is Pivotal Labs. Pivotal Labs' mission is to both deliver highly-crafted software products and provide a transformative experience for their client's engineering cultures. To change a developer's way of working, Pivotal combines the client's software engineers with Pivotal's engineers at a Pivotal office where they can experience Extreme Programming in an environment conducive for agile development. % Tmp for 6 pages    For startups, Pivotal might be the first engineers working on the project. For enterprise clients, Pivotal provides additional engineering resources to accomplish new business goals. 


% A common team size is six developers plus an interaction designer and a product manager. In the history of the Palo Alto office, the number of developers on a project ranges from 2 to 28. Larger projects are organized into smaller coordinating teams with one product manager per team and one or two interaction designers per team.


% Tmp to fit in 6 pages
%Commonly utilized technologies include Angular, Android, Backbone, iOS, Java, Rails, React, and Spring which are often deployed onto Pivotal's Cloud Foundry. 


% Pivotal Labs has followed Extreme Programming \cite{BeckExtremeProgramming2004} since the late 1990s. While each team is autonomous in making its own decisions as to what is best for a particular project, the company culture strongly suggests following all of the core practices of Extreme Programming. % Tmp to fit in 6 pages  This includes Pair Programming, Test Driven Development, Weekly Retrospectives, Daily Stand-ups, Prioritized Backlog, Whole Team ownership of the project and code base, plus Kanban's notion of work flowing through people.


\section{Team Code Ownership}
\label{TeamCodeOwnership}


In the literature, collective code ownership is often treated as a policy statement. In this case, simply claiming that \quotes{anyone can modify any piece the code} was not sufficient to engender willingness to modify any file. Rather, ownership is an emotional or qualitative attribute that ties all developers on the team to the project and code base. It is a spectrum where, on one side, each individual has ownership of only their code, and on the other side, everyone on the team owns the entire code base. Some events appear to erode the team's sense of ownership over the project's duration, while some practices appear to counteract these erosions. This section details the five factors that appear most related to team code ownership and examples of events or tendencies that erode it. 


\subsection{System Context}
\textbf{Definition:} System context is the knowledge and situational awareness about the code, including the discourse that surrounds the code. System context includes understanding existing design decisions, underlying technologies, the relationship between features and user needs, and the implementation of existing features.


\textbf{Purpose:} Developing an in-depth knowledge of the system exercises the \quotes{intimately knowing the target} path of psychological ownership.


For a pair to work efficiently on any part of the system, one of them needs to have enough context to know how that part of the system works. Without enough context, a pair might struggle, slow down, or be blocked in working on a feature.


Team code ownership seems to vary with the context that the developer has about the code; the more the developer knows, the higher the sense of ownership. Knowledge silos, the size of the code base, or the number of developers working in parallel can make it difficult for a programmer to develop a deep system context level.


\textbf{Threat: Increasing knowledge silos.} When developers routinely work on one part of the code base, they can develop specific system context not shared by the team. Code specialization impedes anyone on the team from modifying any part of the team's code.  One team said \participantQuote{we need Marion on that story, only she knows the Apple watch code base,} and \participantQuote{shea knows the ins-and-outs of the legacy integration, we need him to work on this story,} which means there is a hindering imbalance between the individual and team understanding of the code.


\textbf{Threat: Increasing code base size.} One team started working with a large code base that was over eight years old and the team did not have a full understanding of the system. Initially, the team felt little ownership of the code, even though the team was responsible for it and agreed to \singleQuote{team code ownership.} Often the team would need to ask a product manager why certain features exist in the code to understand the code's purpose and implementation. In time, as the team worked with the code and gained context, the team's sense of ownership improved.


\textbf{Threat: Increasing team size.} We observed the relationship between team size and code context on \numberOfObservedProjects{} Pivotal projects as a participant-observer. As team size increases, the ability to gain system context decreases. Every day, all pairs are adding to the system. On a five pair team, so much work is happening each day that it becomes increasingly difficult to keep track of everything that changes.


One developer on a ten-person project said, \participantQuote{I feel that we don't have the context spread around fully. Having five, sometimes six, pairs on the project makes it go really fast, so it's hard to keep context.}


When developers do not have context about part of a system, or context about what remains to be done to finish a story, reluctance to start the next story at the top of the backlog emerges. It's easier to start a story that touches part of the system that they know. As one developer reflected, \participantQuote{I am not entirely comfortable to jump into stories on certain aspects [of the system].}


As a coping strategy, one developer, before the start of the work day, skimmed the git commits from the previous day to learn about new classes and changes in design and to understand the features the team added. 


As team size grows, there is a potential risk of decreasing an individual developer's sense of team code ownership. 


\subsection{Code Contribution}
\textbf{Definition:} Code contribution is the portion of the code that a given developer has worked on. 


\textbf{Purpose:} Personally contributing to the code base increases a developer's sense of ownership by exercising \quotes{investing in the target} path of psychological ownership. 


As a developer works on the code base, the developer's system context level increases. While code contribution level influences the system context level, it is not necessary related: developers might learn about the code through other means different from direct contribution, including conversations at stand-up, impromptu team huddles, or a pair saying \participantQuote{check out what we did yesterday.}


\textbf{Threat: Inability to contribute.}  A developer's inability to contribute to the code base decreases the developer's sense of ownership. 


This could happen, for instance, during a pair programming breakdown. When the pairing experience breaks down, one person drives the code development while the partner passively watches. (\quotes{Performance Pair Programming} describes when one developer plows through a story and stops listening to the developer's partner.)  When one person is writing all the code, individual code ownership replaces team code ownership.  


%six paper vesion
In one situation, the partner took over and ignored the participant's input. The participant reflected, \participantQuote{I would not be able to explain deeply what we had done. I would not be able to maintain it. I didn't really write it, so I feel very little ownership of it.} 


% full version: In one situation, the partner took over and ignored the participant's input. The participant reflected, \participantQuote{I did not understand what was really going on. I wouldn't be able to explain deeply what we had done. I wouldn't be able to maintain it. I didn't really write it, so I feel very little ownership of it.} 




Ideally, Pair Programming is a collaborative experience where both individuals are unable to tell who wrote which portions of the code. 


\subsection{Code Quality}
\textbf{Definition:} Code quality relates to how well the code satisfies the project's desirable quality attributes. Desirable quality attributes might include design qualities, performance, reliability, scalability, security,  testability, and usability \cite{Meier2009}. 


\textbf{Purpose:} A high quality product satisfies the self-identity motivation of psychological ownership. Developers might not want to be identified with a low quality product.


Low-quality products also tend to involve a disproportionate amount of bug fixes. Developers need a balance between creating new features and fixing bugs each week. Working only on bugs for weeks affects their sense of ownership.    


\textbf{Threat: Pressure to deliver and deprioritizing continuous refactoring.} When developers are pressured to deliver more features at the expense of Continuous Refactoring, the code acquires technical debt, the code becomes more difficult to work with, and developers can begin to feel indifferent about the code. When developers begin to experience code apathy, this decreases their sense of team code ownership. 


When the team neglects refactoring, new code is simply bolted onto the existing design. Each time the team bolts something else on, bolting on the next piece becomes more complicated. Thus, a dilemma arises for the programmers working on the next story that touches this part of the code: do they continue bolting on more code, or do they perform the pretermitted refactoring? A team's avoidance of refactoring may be a sign that code apathy is settling in. Code apathy results in reduced quality, as the developers become less invested in the craftsmanship of the code.


One developer felt \participantQuote{proud and disgusted} about the code base. He is simultaneously proud of each refactoring that the team performed and disgusted by the technical debt the team accrued by taking shortcuts to ship more features. The developer drew Figure \ref{Programmer1} to show his feeling about the code, \participantQuote{it is generally orderly with a few bits that maybe are not as orderly.}


Before the first launch of a product, the product manager suggested that the team deliver more features at the expense of technical debt. For some of the team, this was an unacceptable tradeoff, and those developers decided not to cut corners. Others on the team complied with the request and incurred technical debt. The entire team ended up paying the consequences with extensive refactors after the launch. On a communal code base, one pair adding tech debt affects everyone on the team.


When code apathy settles in, team members adopt the attitude that someone else will solve the problem with the code. When this attitude permeates a team, no one is solving the problems. 


\begin{figure}[t]
\centering
\includegraphics[width=3.45in]{team_code_ownership_images/CodeOwnership.jpg}
\caption{\quotes{Draw how you feel about the code}}
\label{Programmer1}
\end{figure}


The team wants to feel pride in improving code quality.  It feels good to be improving the code design and readability. If the team starts neglecting these concerns, it can engender a sense of disgust and apathy for the code can spread throughout the team.


\subsection{Product Fit}
\textbf{Definition:} Product fit is developers believing that features of the product will satisfy the user's needs.


\textbf{Purpose:} Engineers want to create products that matter to the users. Delivering a product that matters to someone satisfies the self-identity motivation of psychological ownership.


\textbf{Threat: Ignoring user feedback.} When the product manager ignores feedback from user research and usability testing, developers may lose faith in the product's ability to achieve its goals. Developer motivation and engagement can decrease when developers perceive they are building a feature that users have explicitly said they do not want yet is built to solve a business goal. 


\textbf{Threat: Ignoring developers feedback about the product.} Pivotal's balanced team approach is founded on collaboration between product managers, interaction designers, and developers. When product managers or other stakeholders ignore feedback from developers, developers can begin to feel less ownership in the product, and in turn, be less motivated to work on the project. 


%When picking up a story, a developer verifies that the story contains clear acceptance criteria. In the absence of acceptance criteria, the developer typically clarifies what needs to be done with the product manager. On one project during which the stakeholders ignored feedback from the developers, one developer recalled, \participantQuote{In this case, I don't feel like spending the extra energy to go and say, `Hey, are you sure? Is this what you want?'} 


Feature apathy or product apathy can result in a poorly crafted product that does not meet the customer's needs.


\subsection{Team Cohesion}
\textbf{Definition:} Team cohesion is the degree to which team members identify as part of the team, stick together through adversity and take pride in the team's accomplishments \cite{Bollen1990Perceived, Beal2003Cohesion, Whitworth2007Motivation}.


\textbf{Purpose:} Team cohesion satisfies the \quotes{having a place} motivation of psychological ownership.


\textbf{Threat: Distancing a developer from the team.} Team apathy manifests when developers do not feel that they are a part of the team. Developers feel less ownership of the code base when they feel excluded from the team.


Several behaviors were observed that can distance a developer from the team: interrupting the developer during discussions, using poor listening skills so that the developer feels unheard, or talking beyond the developer's level of technical expertise. 


On one team, during discussions, the team talked about code but never looked at the source code. One developer found these abstract discussions difficult to follow. Sometimes the team discussed parts of the code that the individual had not seen recently. When the team discussed two variants of coding practices without showing concrete examples, the programmer could not contribute. When the developer raised this issue to the team and the team continued with the status quo, the programmer felt marginalized by the team.


Poor onboarding of developers can contribute to feelings of isolation. On one project, there was a time crunch and the team was feeling the pressure to deliver stories. When the team added developers, the team had a \quotes{sink or swim} attitude, letting new team members figure things out on their own, hence making them feel unwelcome.


When developers feel that the team does not care about them, their sense of ownership can decrease.


\section{Discussion}
\label{Discussion}
\subsection{Transitioning to team code ownership}
\label{Transitioning}


The above results have numerous implications for teams attempting to transition to team code ownership. 


Some developers effortlessly make the transition to team code ownership. They immediately see the benefits of being able to modify any part of the code base and quickly shift from \quotes{I made this} (personal ownership) to \quotes{we made this} (collective ownership.)


Others may struggle with team code ownership for several reasons:


\begin{itemize}  


\item Developers may struggle to transition to a caretaker mindset.  In one interview, a software engineer struggled to describe the developer's relationship with the code on a very challenging project and settled in on the caretaker metaphor: \participantQuote{Sometimes I kind of feel like a janitor to [the code base].  Maybe caretaker would be better. Yeah, probably caretaker. I feel like a janitor just cleans up messes, but a caretaker makes things better.} 


\item A developer may be distraught at \participantQuote{seeing my work slowly removed from the app.} 


\item Developers can no longer take pride in functionality that they exclusively develop.


\item Existing knowledge silos, which hinder team code ownership, may be slow to break down.
 
\end{itemize}


New hires struggling with the transition slowly realize that \participantQuote{someone else is going to take over and they're going to do fine. I can move onto something else and that's okay.} They recognize the lack of long-term individual authorship, learn to expect their code to be transitory, develop trust in their teammates and thus loosely hold personal contributions. \participantQuote{The code that I write today may be in the code base for a little while, and it will evolve into something better.} Eventually,  they experience the benefits of a collaborative environment: \participantQuote{People are a lot more flexible all across the board, with changing things or accepting feedback or collaborating,} and the team can say \quotes{Hey, this is our code!}


Shifting from individual to team code ownership may requires multiple and complementary practices to actively remove knowledge silos. In this case, daily pair rotation helped combat knowledge silos. Moreover, for developers with strong individual ownership tendencies, sharing ownership first with a small group (where trust and communication come easier) may help. One Pivotal engineer uses improvisation and collaboration games to help teams practice letting go of control, trusting the team, and learning to be pleasantly surprised by what emerges. 


\subsection{Results Evaluation}


The factors influencing team code ownership presented in Section \ref{TeamCodeOwnership}, have emerged from the Grounded Theory research study introduced in Section \ref{SustainableSoftwareDevelopmentTheory}. While other factors may influence team code ownership, this discussion focuses only on those that were observed during the study. Grounded Theory studies can be evaluated using the following criteria \cite{Charmaz}: 


\textbf{Credibility:}  The \numberOfInterviews{} intensive open-ended interviews and numerous field notes from participant-observation serve as a rich and credible data set for the analysis. 


\textbf{Originality:} The research broadens the idea of team code ownership by acknowledging that collective code ownership is more than a policy statement, and by uniquely identifying factors that affect the team's sense of code ownership.


\textbf{Resonance:} Several participants reviewed the findings and indicated that both the factors and threats resonate with their experience.


\textbf{Usefulness:} The study identifies factors associated with ownership and suggests several ways of engendering team code ownership.


This work analyzed software projects at the Silicon Valley office of Pivotal following Extreme Programming. From an \textbf{external validity} perspective, grounded theory is non-statistical, non-sampling research. The results therefore cannot be statistically generalized to a population. Rather, researchers and professionals can adapt the concepts and ideas to other contexts case-by-case. 


Finally, the results might be influenced by \textbf{researcher bias} or \textbf{prior knowledge bias}. A risk of the participant-observer technique is that the researcher may lose perspective and become biased by being a member of the team. While a participant-observer gains perspective an outsider cannot, an outside observer might see something a participant observer will miss. Similarly, while prior knowledge helps the researcher interpret events and select lines of inquiry, prior knowledge may also blind the researcher to alternative explanations \cite{GlaserIssues}. These risks were mitigated by recording interviews and having the researchers review the coding process. 


\section{Conclusion}
\label{TeamCodeOwnershipConclusion}
This chapter reports results from a participant-observation, constructivist grounded theory study at Pivotal, a large American software company employing Extreme Programming practices. It provides three main contributions.


1) The observations clearly indicate that \textbf{team code ownership is a feeling to be engendered not a policy to be decreed}.


2) Meanwhile, both discussions with and observations of participants suggest five factors associated with strong feelings team code ownership. Pivotal developers more acutely feel team code ownership when i) they understand the system context; ii) they have contributed to the code in question; iii) they perceive code quality as high; iv) they believe the product will satisfy user needs; and v) they perceive team cohesion as high.   


3) Moreover, diverse events and trends can undermine sense of ownership, including:  increasing knowledge silos, increasing code base size, increasing team size, inability to contribute, pressure to deliver and deprioritizing continuous refactoring, ignoring user feedback, ignoring developer feedback, and distancing a developer from the team. 


In conclusion, Pivotal's developers find team code ownership highly advantageous; however, transitioning to a team code ownership model is easier for some than others. Some agile practices including continuous pair programming, overlapping pair rotation, continuous refactoring, and test driven development appear to help. Promising angles for future research include more nuanced explorations of the code ownership spectrum, further exploration of the roles of emotion and identity, as well as developing specific practices for facilitating ownership transitions. 



\input{software-engineering-waste-do-not-edit}
% Sample apostrophy’s to remove team's 


\chapter{Conclusions}
\label{ConclusionChapter}


\section{Summary}


Using Grounded Theory, we examined the question, \quotes{what is happening at Pivotal when it comes to software development?} Listening to the concerns of the participants, the emergent research lead to three core concerns: 1) how does Pivotal construct software? 2) how does Pivotal engender team code ownership? and 3) what kinds of waste does Pivotal identify and remove in its continuous improvement? 
 
In exploring the construction of software, Grounded Theory guided us to the theory of Sustainable Software Development, a set of principles, policies, and practices used in the construction of software products so that teams survive team disruptions such as team churn. The theory is rooted in team code ownership, as removing knowledge silos and caretaking the code enable teams to be able to modify any part of the code base. 


In researching how Pivotal engenders team code ownership, we identified five factors that affect team code ownership: system context, code contribution, code quality, product fit, and team cohesion. In reviewing the data, we discovered that transitioning from individual code ownership to team code ownership is not easy for some engineers as psychological ownership fulfills deep psychological needs.


In investigating removing waste, we created a waste taxonomy that identifies the type of waste the participants detect and try to remove in the studied organization. As a model, the waste taxonomy succinctly describes waste types and is more descriptive than previous work. 


\section{Future Research}
Given the emergence of the theory of Sustainable Software Development, a model of team code ownership, and a waste taxonomy, and a deeper understanding of the evolution of Extreme Programming, there are several opportunities for future research.


For the theory of Sustainable Software Development, additional experimentation can be done to assess each practice’s flexibility
while still maintaining the ability for a team to survive disruptions. For example, Pivotal Labs adopts a shared schedule of 9 am to 6 pm. What would happen if teams adopted core hours of 10 am to 5 pm? Some engineers might work 8 am to 5 pm with soloing on simple chores for the first two hours, while other engineers might work 10 am to 7 pm with soloing on simple chores for the last two hours. In this situation, would the teams have the benefits of Sustainable Software Development while introducing one degree of freedom for flexible hours?


We are interested in the tension between individual and team ownership, as well as the factors that foster and decrease the sense of ownership. Developers, interaction designers, and product managers all have different goals for their role. Future work could examine how the sense of ownership is driven by different factors for each role. Some programmers naturally adapt to team code ownership, while others struggle with the transition. Future research could follow new Pivotal engineers and examine their journey in transitioning from individual code ownership to team code ownership. Perhaps there are specific practices that Pivotal or the development team could employ to ease the transition. We could also investigate the optimal team size for team code ownership, or explore whether Sustained Software Development works for a distributed team with a Shared Schedule.


For the waste taxonomy, Pivotal projects can systematically examine potential wastes. Some wastes are rarely salient unless one is specifically looking for them. The waste taxonomy may prove helpful in more readily identifying these waste on a project. 


Pivotal has evolved the Extreme Programming practices for building and managing a backlog. Instead of committing to work to be done each week, teams simply work off the top of the backlog in an iteration-less flow. Additional research could examine how the evolution of the Extreme Programming backlog practices affects software development.
\section{Conclusion}
Pivotal’s continuous improvement practices result in a natural evolution of Extreme Programming.  Each project becomes an experimentation opportunity for the engineers to try new ideas and see the results. As compared to a company with a relatively stable product line, Pivotal Labs has a unique opportunity for continued experimentation as it undertakes scores of projects each year. One Pivotal engineer reflected that each project feels like a hill climbing optimization algorithm exploring the feature set of the product. Each week, stories guide the product towards a delightful product for the user. Likewise, all these teams can feel like hill climbing optimization exploring the software development process. Fundamentally, the teams are on the hill of Extreme Programming. It is unlikely that the teams will accidentally discover a new hill, yet with each new project, the teams have an opportunity to refine Extreme Programming in ways not envisioned by its creator. 


\appendix
% Sample apostrophy's to remove team's

\chapter{Waste Examples and Chain of Evidence}
\label{AppendixChainOfEvidence}

This section lists out the chain of evidence for the software engineering waste taxonomy. This emergent theory comes from participant observation and analysis of retrospection data. For each listed retro item, participant observation confirmed the issue. In a few instances, participant observation provided examples not identified in a retro.

% Technology that only works on one machine
% Poorly documented library
% Poorly tested library
% Dependent technology not ready for prime time
% Harmful side effects of API
% Stories missing release labels and confusion about release branches
% Working on multiple stories at once
% Working on two branches



\underline{Waste Category}: Building the wrong feature or product

\quad \textbf{Cause Category}: User Desiderata

\quad \quad \textit{Cause Property}: Not doing user research, validation, or testing

\quad \quad \quad Retro Topic: Not getting access to actual customers

\quad \quad \quad Retro Topic: Unable to find users to interview

\quad \quad \quad Retro Topic: Not thinking about the user

\quad \quad \quad Retro Topic: No testing for a few weeks

\quad \quad \quad Retro Topic: Implementing features not validated by users

\quad \quad \quad Retro Topic: Not doing usability testing

\quad \quad \textit{Cause Property}: Ignoring user feedback

\quad \quad \quad Retro Topic: Ignoring user feedback

\quad \quad \quad Retro Topic: Ignoring user research

\quad \quad \quad Retro Topic: Building a feature that users do not want

\quad \quad \quad Retro Topic: Redoing mock-ups because stakeholder chooses to ignore user research

\quad \quad \textit{Cause Property}: Working on low user value features

\quad \quad \quad Retro Topic: Unable to articulate the value to the user

\quad \quad \quad Retro Topic: Building unnecessary features

\quad \quad \quad Retro Topic: Working on high effort features for low user value

\quad \quad \quad Retro Topic: Creating demos for stakeholders

\quad \quad \quad Retro Topic: Wanting to build every feature that comes to mind

\quad \quad \quad Retro Topic: Creating mockups for features needed in the distant future, not the present

\quad \quad \quad Retro Topic: Load testing site with expected low usage

\quad \textbf{Cause Category}: Business Desiderata

\quad \quad \textit{Cause Property}: Not involving a stakeholder

\quad \quad \quad Retro Topic: Not closing the loop with a stakeholder

\quad \quad \quad Retro Topic: Disconnect between team and stakeholders

\quad \quad \quad Retro Topic: Disconnect between team and headquarters

\quad \quad \quad Retro Topic: Not enough stakeholder involvement

\quad \quad \quad Retro Topic: Missing product owner

\quad \quad \quad Retro Topic: Not knowing the product owner's goals

\quad \quad \quad Retro Topic: Decision makers elsewhere

\quad \quad \textit{Cause Property}: Slow stakeholder feedback

\quad \quad \quad Retro Topic: Slow feedback from stakeholder

\quad \quad \quad Retro Topic: Wanting more involvement from the product manager

\quad \quad \quad Retro Topic: Slow turn around to request for feedback

\quad \quad \textit{Cause Property}: Unclear product priorities

\quad \quad \quad Retro Topic: No vision for next scope of work

\quad \quad \quad Retro Topic: Lack of focus

\quad \quad \quad Retro Topic: Unfocused scope

\quad \quad \quad Retro Topic: Unclear priorities

\quad \quad \quad Retro Topic: Inconsistent priorities and directions

\quad \quad \quad Retro Topic: Unable to sequence or prioritize the work


\underline{Waste Category}: Mismanaging the backlog

\quad \textbf{Cause Category}: Backlog inversion

\quad \quad Retro Topic: Most important items are not at the top of the backlog

\quad \quad Retro Topic: Difficulty in finding valuable work

\quad \quad Retro Topic: Starting stories that are not at the top of the backlog

\quad \textbf{Cause Category}: Duplicated work

\quad \quad Retro Topic: Duplicated story in the backlog

\quad \quad Retro Topic: Forgetting to start a story

\quad \quad Retro Topic: Pushing code to a branch that is forgotten and reimplemented by team

\quad \quad Retro Topic: Two pairs doing the same needed work without creating a chore

\quad \quad Retro Topic: Not looking at stories in flight to know who is doing what

\quad \textbf{Cause Category}: Starting too many features

\quad \quad Retro Topic: At release turning off feature flags for partially completed work

\quad \textbf{Cause Category}: Not enough ready stories

\quad \quad Retro Topic: Story desert

\quad \quad Retro Topic: Shallow backlog

\quad \quad Retro Topic: Not enough tracks of work

\quad \quad Participant Observation: Engineers writing feature stories

\quad \quad Retro Topic: Ad hoc stories

\quad \quad Retro Topic: Many blocked stories

\quad \textbf{Cause Category}: Imbalance of feature work and bug fixing

\quad \quad Retro Topic: Delivering only bugs and chores this week

\quad \quad Retro Topic: Two weeks of features, two weeks of bugs, repeat

\quad \textit{Cause Category}: Delaying testing or critical bug fixing

\quad \quad Participant Observation: Prioritizing feature work over testing

\quad \quad Retro Topic: Delaying end-to-end testing

\quad \quad Retro Topic: Not looking for defects until end of a release

\quad \quad Retro Topic: Critical bugs not prioritized when detected

\quad \quad Retro Topic: Working on features for two weeks, working on bugs for two weeks

\quad \textbf{Cause Category}: Capricious Thrashing

\quad \quad Retro Topic: Fiddling with the sequence and number of steps in the user registration process

\quad \quad Retro Topic: Repeatedly re-sequencing the order customization process

\quad Participant Observation: Engineers prioritizing stories

\underline{Waste Category}: Unnecessary complex solutions

\quad \textbf{Cause Category}: Unnecessary feature complexity from the user's perspective

\quad \quad Retro Topic: Design and development not discussing upcoming work

\quad \quad Retro Topic: Mockup is too complex and could be simplified

\quad \quad Retro Topic: Business logic is too complex and could be simplified

\quad \quad Retro Topic: Too many edge cases

\quad \quad Participant Observation: Feature does not solve problem in simpliest way

\quad \quad Participant Observation: Cheaper engineering solution to the \quotes{sample} problem

\quad \textbf{Cause Category}: Unnecessary Technical Complexity

\quad \quad Retro Topic: Too much context needed to understand system

\quad \quad Retro Topic: Datacenter architecture could be simplified to achieve same goals

\quad \quad \textit{Cause Property}: Duplicating code

\quad \quad \quad Participant Observation: Two classes do the same thing

\quad \quad \quad Participant Observation: Code logic is not in one place, missing cohesion

\quad \quad \quad Participant Observation: Competing solutions for test setup

\quad \quad \textit{Cause Property}: Lack of interaction design reuse

\quad \quad \quad Retro Topic: Every page is unique

\quad \quad \quad Retro Topic: Page specific stlying means no css reuse

\quad \quad \quad Retro Topic: Team following styleguide-driven-development with limited designer buy-in

\quad \quad \quad Retro Topic: Inconsistent look and feel

\quad \quad \textit{Cause Property}: Overly complex technical design created up-front

\quad \quad \quad Retro Topic: Changing up-front design because of new information

\quad \quad \quad Retro Topic: Created design is too complex for problem

\quad \quad \quad Retro Topic: By over designing, system re-invented the wheel

\quad \quad \quad Retro Topic: Big design up-front feels like busywork






\underline{Waste Category}: Mental workload

\quad \textbf{Cause Category}: Inefficient tools and problematic APIs, libraries, and frameworks

\quad \quad Retro Topic: << many different tools >>

\quad \quad Retro Topic: << many different APIs >>

\quad \quad Retro Topic: << many different libraries >>

\quad \quad Retro Topic: << many different frameworks >>

\quad \textbf{Cause Category}: Suffering from technical debt

\quad \quad Retro Topic: Technical debt

\quad \quad Retro Topic: Code smells

\quad \quad Retro Topic: Creating too many chores than the team can accomplish to deal with technical debt

\quad \quad Retro Topic: Rushing to deliver code creating technical debt

\quad \quad Retro Topic: Delayed refactoring

\quad \quad Retro Topic: Refactoring needed \quotes{everywhere}

\quad \quad Retro Topic: Limits on refactoring

\quad \quad Retro Topic: \quotes {Siteminder makes me angry enough that I want to hack into it, expose how useless and horrible it is and wipe this miserable product off the face of the earth!}

\quad \quad \textit{Cause Property}: Hard to change code

\quad \quad \quad Retro Topic: Code is getting harder to change

\quad \quad \quad Retro Topic: Lots of coupling and dependencies between classes

\quad \quad \quad Retro Topic: High code complexity

\quad \quad \quad Retro Topic: Code base becoming unmaintainable

\quad \quad \quad Retro Topic: Copy and pasting instead of refactoring

\quad \quad \quad Retro Topic: State is everywhere

\quad \quad \quad Retro Topic: Code that does not play to strength of programming language

\quad \quad \quad Retro Topic: Lacking confidence to change code

\quad \textbf{Cause Category}: Unnecessary cognitive effort

\quad \quad \textit{Cause Property}: Inefficient development flow

\quad \quad \quad Retro Topic: Git force push (overrides version control system)

\quad \quad \quad Retro Topic: Not running test before delivering code

\quad \quad \quad Retro Topic: Commits that break the build

\quad \quad \quad Retro Topic: Merge conflicts due to long lived branches or large changes

\quad \quad \quad Retro Topic: Delivering code that is not done

\quad \quad \quad Retro Topic: Broken build

\quad \quad \quad Retro Topic: Pushing code on a broke build

\quad \quad \quad Retro Topic: Continuous integration is red this week

\quad \quad \quad Retro Topic: Deploying software is painful

\quad \quad \textit{Cause Property}: Sluething information

\quad \quad \quad Retro Topic: Unclear bug reports

\quad \quad \quad Retro Topic: Unclear commit messages

\quad \quad \quad Retro Topic: Reverse engineering technical desicions

\quad \quad \quad Retro Topic: Competing code patterns

\quad \quad \quad Participant Observation: Messy code

\quad \textbf{Cause Category}: Cognitive hindrance

\quad \quad \textit{Cause Property}: Overload from context switching

\quad \quad \quad \quad Participant Observation: Jumping between too many tasks \quotes{I am so confused. What were the details about this task? I do no remember.}

\quad \quad \quad \quad Participant Observation: Overwhelmed with state \quotes{What problem are we trying to solve again? I need a break.}

\quad \quad \textit{Cause Property}: Emotional stress

\quad \quad \quad \textit{Cause Sub-Property}: Low team morale

\quad \quad \quad \quad Retro Topic: Frustrated developers

\quad \quad \quad \quad Retro Topic: Not managaging expectations

\quad \quad \quad \quad Retro Topic: Negative attitudes

\quad \quad \quad \quad Retro Topic: Apathy

\quad \quad \quad \quad Retro Topic: Not knowing everyone on the team

\quad \quad \quad \quad Retro Topic: Project feels like it is falling apart emotionally

\quad \quad \quad \quad Retro Topic: Unacknowledged by management

\quad \quad \quad \quad Retro Topic: Messy code

\quad \quad \quad \quad Retro Topic: Imbalance of bugs to feature work

\quad \quad \quad \quad Retro Topic: Poor lighting, lack of windows

\quad \quad \quad \textit{Cause Sub-Property}: Rush mode

\quad \quad \quad \quad Retro Topic: Fixed features with a fixed timeline

\quad \quad \quad \quad Retro Topic: Not knowing features for the release

\quad \quad \quad \quad Retro Topic: Aggressive timelines

\quad \quad \quad \quad Retro Topic: Shifting deadline

\quad \quad \quad \quad Retro Topic: Scope creep

\quad \quad \quad \quad Retro Topic: Prolonged rush mode

\quad \quad \quad \quad Retro Topic: Unrealistic expectations

\quad \quad \quad \quad Retro Topic: Repeatedly saying \quotes{this has to be done today}

\quad \quad \quad \quad Retro Topic: Long days

\quad \quad \quad \quad Retro Topic: Overtime

\quad \quad \quad \textit{Cause Sub-Property}: Lack of empathy

\quad \quad \quad \quad Retro Topic: Not listening

\quad \quad \quad \quad Retro Topic: Criticizing in public

\quad \quad \quad \quad Retro Topic: Difficult pairings

\quad \quad \quad \quad Retro Topic: Pairing fatigue

\quad \quad \quad \quad Retro Topic: Interpersonal conflict

\quad \quad \quad \quad Retro Topic: Kicking product out of the team space

\quad \quad \textit{Cause Property}: Cognitive load

\quad \quad \quad \textit{Cause Sub-Property}: Complex or large stories

\quad \quad \quad \quad Retro Topic: Lots of edge cases

\quad \quad \quad \quad Retro Topic: Mananging too much complexity in one super large story

\quad \quad \quad \quad Retro Topic: Repeated rejection of super large story

\quad \quad \quad \quad Interview: Large stories with lots of comments

\quad \quad \quad \textit{Cause Sub-Property}: Noisy output from tools

\quad \quad \quad \quad Retro Topic: Distracted by useless output when running tests

\quad \quad \textit{Cause Property}: Context Switching from delayed feedback

\quad \quad \quad \textit{Cause Sub-Property}: Product manager taking a long time to accept or reject a story

\quad \quad \quad \quad Retro Topic: Many stories waiting for acceptance

\quad \quad \quad \quad Retro Topic: Long stories acceptance feedback loop

\quad \quad \quad \quad Retro Topic: Builds going out with unaccepted stories

\quad \quad \quad \quad Retro Topic: Hard to figure out why a story is rejected

\quad \quad \quad \textit{Cause Sub-Property}: Unavailable product manager or interaction designer

\quad \quad \quad \quad Retro Topic: Missing product manager

\quad \quad \quad \quad Retro Topic: Missing designer

\quad \quad \quad \quad Retro Topic: Not quickly responding to questions

\quad \quad \quad \quad Retro Topic: Product has alot on their plate

\quad \quad \quad \quad Retro Topic: Discussing about getting more product managers

\quad \quad \quad \quad Retro Topic: Designers are missing in pairs

\quad \quad \quad \quad Retro Topic: Not enough designer/developer pairing happening




\underline{Waste Category}: Rework

\quad \textbf{Cause Category}: Rejected Stories

\quad \quad Retro Topic: Rejected stories

\quad \textbf{Cause Category}: No clear definition of done

\quad \quad \textit{Cause Property}: Ambiguous story

\quad \quad \quad Retro Topic: Unclear acceptance criteria

\quad \quad \quad Retro Topic: Poor acceptance criteria leading to stories getting accepted that do not work

\quad \quad \quad Retro Topic: Bug reports missing repeatable steps

\quad \quad \quad Retro Topic: Rushed stories in the backlog

\quad \quad \textit{Cause Property}: Second guessing design mocks

\quad \quad \quad Retro Topic: Second guessing design mocks

\quad \textbf{Cause Category}: Defects and Bugs

\quad \quad \textit{Cause Property}: Defects and Bugs

\quad \quad \quad Retro Topic: Broken feature

\quad \quad \quad Retro Topic: Application is in an unshippable state

\quad \quad \quad Retro Topic: Adding fake pages or code makes system unshippable

\quad \quad \quad Retro Topic: Finishing stories before delivering

\quad \quad \textit{Cause Property}: Poor testing strategy

\quad \quad \quad Retro Topic: Green build, broken code

\quad \quad \quad Retro Topic: Sparsely tested component

\quad \quad \quad Retro Topic: Not testing code, untested code

\quad \quad \quad Retro Topic: Increasing number of bugs

\quad \quad \quad Retro Topic: No test failure resulting in rework

\quad \quad \quad Retro Topic: Hurrying to get stories finished resulting in more defects

\quad \quad \textit{Cause Property}: No root-cause analysis on bugs

\quad \quad \quad Retro Topic: Not learning from defects and team's mistakes

\quad \quad \quad Retro Topic: Not understanding what is root cause of tech problems








\underline{Waste Category}: Waiting

\quad \textbf{Cause Category}: Slow tests or unreliable tests

\quad \quad Retro Topic: Slow tests

\quad \quad Retro Topic: Flaky tests

\quad \quad Retro Topic: Randomly failing tests

\quad \textbf{Cause Category}: Unreliable acceptance environment

\quad \quad Retro Topic: Acceptance server is unreliable

\quad \textbf{Cause Category}: Missing information, people, or equipment

\quad \quad Retro Topic: Missing equipment

\quad \quad Retro Topic: Missing product

\quad \quad Retro Topic: Missing computer

\quad \quad Retro Topic: Missing headsets

\quad \quad Retro Topic: Missing headsets


\underline{Waste Category}: Knowledge loss

\quad \quad Retro Topic: Knowledge loss from team churn

\quad \quad Retro Topic: Team member departure

\quad \quad Retro Topic: Knowledge silos forming



\underline{Waste Category}: Ineffective communication

\quad \textbf{Cause Category}: Team size is too large

\quad \quad Retro Topic: Improving communication across the whole team

\quad \quad Retro Topic: Improving collaboration between teams

\quad \quad Retro Topic: Entire team not in discussion about changes

\quad \quad Retro Topic: Coordination problems since team is too large

\quad \quad Retro Topic: Difficult retros since team is so large

\quad \textbf{Cause Category}: Asynchronous communication

\quad \quad \textit{Cause Property}: Distributed teams

\quad \quad \quad Retro Topic: No communication channel for remote meeting attendees

\quad \quad \quad Retro Topic: Distributed teams

\quad \quad \quad Retro Topic: Too many asychcronous communications through slack

\quad \quad \quad Retro Topic: Remote pairing is harder

\quad \quad \quad Retro Topic: Distributed retros

\quad \quad \quad Retro Topic: Split across timezones

\quad \quad \quad Retro Topic: Long communication threads in stories in backlog

\quad \quad \textit{Cause Property}: Non-collocated stakeholder

\quad \quad \quad Retro Topic: Non-collocated stakeholder

\quad \quad \textit{Cause Property}: Dependency on another team

\quad \quad \quad Retro Topic: Not communicating changes when making breaking changes

\quad \quad \quad Retro Topic: Coordinating changes in APIs

\quad \quad \quad Retro Topic: Blocked on other teams

\quad \quad \textit{Cause Property}: Opaque processes outside team

\quad \quad \quad Retro Topic: Opaque processes in other teams

\quad \textbf{Cause Category}: Imbalance

\quad \quad Retro Topic: Resistance to quick team huddles

\quad \quad \textit{Cause Property}: Dominating the conversation

\quad \quad \quad Retro Topic: not including people in the conversation

\quad \quad \quad Retro Topic: two people dominating meetings

\quad \quad \quad Retro Topic: quieter voices needing to speak up more

\quad \quad \textit{Cause Property}: Not listening

\quad \quad \quad Retro Topic: Not enough listening

\quad \textbf{Cause Category}: Inefficient meetings

\quad \quad \textit{Cause Property}: Lack of focus

\quad \quad \quad Retro Topic: Inefficient planning meeting

\quad \quad \quad Retro Topic: Meetings turning into long intense discussions with no resolution

\quad \quad \quad Retro Topic: Not reviewing action items

\quad \quad \textit{Cause Property}: Skipping retros

\quad \quad \quad Retro Topic: Skipping retros

\quad \quad \textit{Cause Property}: Not discussing blockers each day

\quad \quad \quad Retro Topic: Not communicating critical status updates

\quad \quad \textit{Cause Property}: Meetings running over (e.g. long standups)

\quad \quad \quad Retro Topic: 35 minute standups

\quad \quad \quad Retro Topic: Long standups



\chapter{Interview Drawings and Partial Transcriptions}
I asked many of the interviewees an open ended drawing question to begin a conversation by eliciting the their perspective on a topic. Here I include the diagrams with an abridged interview transcript where the interviewee describes their drawing.
\section{Interview 1 Abridged Transcript with Product Manager }

\textbf{Todd:} This is a drawing exercise for you. So, pretty open-ended question, could you describe a project work flow by drawing it on that sheet of paper? There's no wrong answers. 02:32

\begin{figure}[ht]
\centering
\includegraphics[width=6.5in]{interviews/drawings/2015_05_29.png}
\caption{\quotes{2015-05-29's drawing of a project work flow}}
\label{2015_05_29}
\end{figure}

\textbf{Interviewee:} Does it matter if there's a DNF first or...? 02:39

\textbf{Todd:} However you want it. Typical, ideal, however you want to draw it. 02:44

\textbf{Interviewee:} We actually show this to clients. We have this drawing here. 02:54

\textbf{Interviewee:} Most of the projects that I've been on start with the DNF. We'll call this DNF. 03:07

\textbf{Interviewee:} It's assuming the project has design in development going. This kind of, like, inception. 03:27

\textbf{Interviewee:} And this is dev inception. 03:32

\textbf{Interviewee:} This is your discovery, and framing. 03:50

\textbf{Interviewee:} Here in this area, we're sort of doing more exploratory research and sort of going wide trying to understand the users. That's something we do at the beginning of every project especially if the client doesn't know who their user is or they have ideas of who they are but it's not validated. 04:10

\textbf{Todd:} 	That's the discovery phase. 04:11

\textbf{Interviewee:} Yes. So, this is like exploratory user research. That's like interviews and maybe on site, shadowing people and things like that. And once we have a good idea of who they are, we start wire framing, maybe like the main flow and doing some wire frame tests like user research, that kind of thing and that's kind of within that, I think, we're kind of iterating as well. 04:45

\textbf{Interviewee:} I guess you can either start with one and iterate on that or you can start with many and narrow down. We've done both. So, when you start with many, you just- So, this is like your different As, research, then research to get to C. That makes sense? 05:10

\textbf{Todd:} It's like a portfolio of ideas? 05:14

\textbf{Interviewee:} Yes. I did it on my first project and it worked out really well and then from there you're kind of iterating on what you've come up with. 05:25

\textbf{Interviewee:} We came up with a couple, different nuance ways of doing something because they all seem good and they're like, two or three different features and we sort of combined them in different ways and then we took our insights from that and narrowed it down two versions, narrowed it down to one version. 05:46

\textbf{Interviewee:} I've also done it where you just start with A and you just iterate. I think both worked but this is fun to do. By the time we get to dev inception, we have prioritized epics. You have a develop persona. And you have wire frames. That's kind of the ideal for this point. At that point, development can start on say feature one and in the meantime, we'll start researching feature two. Then, we'll develop it. Then, feature three. So, it kind of goes like that. 06:37

\textbf{Todd:} Nice. 06:39

\textbf{Interviewee:} Maybe you have feature one here and then we'll go here. While they're doing that, we'll start on the next thing. So, on the next thing it flows down. I hate saying waterfall because I know it's the wrong kind. It's like a different kind of waterfall but it iterates in that way. So, we're like, \quotes{Oh, it's going in the cycle.} At any time, we're going to bring in users to test whatever we want. 07:09

\textbf{Todd:} You're drawing people. 07:12

\textbf{Interviewee:} Yes. These are users. At any given point, we can even do exploratory research. We can do wire frame or visual research or we can show them an actual prototype. 	It's kind of nice because you can, whenever you need people, whenever you're stuck on something, you don't have to wait for the right time to bring people in. 	It's just you can test anything at any given time.  07:34

\section{Interview 6 Abridged Transcript with Anchor }

\begin{figure}[h]
\centering
\includegraphics[width=6.5in]{interviews/drawings/2015_08_12_anchor.png}
\caption{\quotes{2015-08-12 Anchor's drawing of software development process}}
\label{2015_08_12_anchor}
\end{figure}

\textbf{Todd:} So on this sheet of paper, could you draw your perspective on our process for software development. 0:00:09

\textbf{Interviewee:}  Our process for software development.  Where do you want to begin?  Or is that like \ldots 0:00:14

\textbf{Todd:} It's a really open-ended question.  0:00:15

\textbf{Interviewee:}  	So there's a customer, a potential customer.  And they come and there's some kind of vetting to, at least, get them in the door.  We think that they have interesting idea of some kind, and maybe we ask a couple of questions; so this might be the OD or sales.  Just some minimal vetting and then just enough for them to say: \quotes{you know what, I think we could possibly have a conversation about what it is that you want to have built.}  So then we do a scoping with them, and in that scoping, the input is their general idea.  Usually, it's like pretty vague; It could be from very vague to they totally know what they're talking about both on their problem space and with the solution to look like.  So then the output of that is a document that basically tells them this is what it likely is, and this is how much it's gonna cost, roughly speaking, and these are gonna be the roles and responsibilities, this is what you would be buying.  What just really actually, some amount of hours of some amount of people, different skill, that's what we're actually promising and we'll aim for this thing, but we'll always constantly trying to give you the best value thing as we go along.  So that's some kind of proposal, I don't know exactly what we call that, but it is all part of the scoping.  0:02:00

\textbf{Interviewee:}   And then, I'm gonna draw, like, maybe a number of stops here; by the time I see it, somebody signed, like, an SOW or similar that was possibly, like, but influenced by that scoping value but who knows what negotiations or whatever happened after that; and probably informed by that document, like, what we say we're aiming to deliver.  But this is, I think, this is more like a legal document or as the SOW is more like a legal document whereas the thing that came out of scoping is more of like a \quotes{come work with us} and some of the other stuff around that.  So then, we go into an all of these assumes it's ok.  I'm talking happy path.  Cool?  0:02:59

\textbf{Todd:}  	Yeah, I'm good.  0:03:00

\textbf{Interviewee:}  	So then we do an inception, and this is where, regardless of what we said over here, we get them to get really clear about who is the person that they're targeting; like, ok so it's about a product market fit, who's the market, what's the product gonna be.  So here's this person or persons or personas, and then for each of one those, like, what are the kinds of things that they need to get done, right?  So they've got things that need to happen, that ultimately will result in to some kind of outcomes, and it almost always like it's gonna convert to dollars in some way, shape, or form.  Sometimes it's a non-profit thing and that's slightly different, there's mixed motivation.  So then what we do is we say, \quotes{ok, in order to meet those tasks, what we need is features from the product} and so we're calling out, like, epics, if you will, feature sets, whatever, and then when we start breaking those down into individual stories and these are just sketches at this point.  We're not gonna get all into acceptance criteria right away, it's just like trying to enumerate coz really the outcome of the inception is a backlog. 0:04:43

\textbf{Interviewee:} So you've got some product owner, who culls the outcome of that into a prioritized list of stories, each one of those describes at tiny interaction between this person and the software that we're building.  Ideally, there's variations on that but then, so that's what an individual user story is.  0:04:58

\textbf{Interviewee:}   And then we go to kick off.  So then, we have our first iteration planning meeting where we take as input the prioritized backlog and for each story we go through them.  And by then, the product owner has got them into a point where I call it readied, they've met the definition of ready, which is they have clear acceptance criteria.  There might even be, a pre-IPM where that work is done, as well as, so this is the input is the backlog and the output is the backlog in the pre-IPM.  There are two things that happen, one is that we get clarity early on what the requirement is, and the other is that we get technical input to help with that prioritization and viability, like, \quotes{ok this may seem simple, like on the face of it}, but actually, there's a whole of things that has to happen to make that work, etc. So we surface some of that, the pre-IPM includes somebody from the product side of the house and someone from development, so product owner, development.  Sometimes depending on those, the story can get more complicated, the more requirements, with the greater the variety or the more exotic the product is.  So you might even need someone in from design who is kind of help guiding how this should unfold and the interactions between things because there's all kinds of decisions from that end.  I Imagine, although, we haven't done this yet, in an enterprise client that you might even have somebody from like architecture there, business architecture.  The point of this conversation is to get all of the perspectives that need to be folded into the prioritization and the validation of, that the stories are legit.  0:07:11

\textbf{Todd:}  	Yes.  0:07:12

\textbf{Interviewee:}  	So then that goes into IPM, and this is a straight-forward roll through - the from top to bottom - the backlog, and we go over each story.  We read out title, we read through the acceptance criteria and then the team points each of the stories, the purpose of that is to surface complexity and to get a general, like, understanding, like, common understanding of what this thing actually is, what it is, what's gonna be involved to building it.  0:07:52

\textbf{Todd:}  	Yes.  0:07:53

\textbf{Interviewee:}   	We don't give in to like, we try not to give in to implementation details.  Sometimes, we have to dip down for a second to like verify that we're all speaking the same language, that we're all really envisioning the same kind of thing.  But that usually is surfaced by like, \quotes{I pointed at one and you pointed at five,} and then I would like \quotes{what?}  So then that hopefully prompts a conversation.  If we all said three, it's good enough for now we move on, if we all think it's about the same thing and the product owner doesn't lose their mind hearing that number, then we're all kind of okay.  Then the output of the IPM is individual points on the stories, and we typically go for some amount of horizon so the minimum that I feel comfortable walking out with is at least for the week, so we don't have to, like, break the flow of getting work done before the next IPM because we're gonna do this once a week, these IPMs we're gonna do this once every week. But ideally, a little bit more runway so that we mitigate against that the team haven't jumped out of the flow and also to, like, help the product owner have some room to sort of steer.  If they only have so many points in the stories, then they have to kind of pay the price of reprioritizing. So that's IPM.  Haven't have written a line of code yet.   So then, after IPM, we get working. So out of the backlog, a pair picks up a story and so then that pair looks at the story...  How deep do you want me to go, because I can go all the way down to like testing and stuff.  0:09:48

\textbf{Todd:}  	Sure.  0:09:49

\textbf{Interviewee:}	Ok.  0:09:50

\textbf{Todd:}	This is your diagram.  There's no wrong answer.  0:09:54

\textbf{Interviewee:}  	Alright. Ok.  So, the pair looks at the acceptance criteria and they say, \quotes{hmmm.... what test do I need to write that will help?  Or a set of test that will help if those tests ran green?  I feel really confident that we've met the spirit of that story.}  So they start there.  Usually at pivotal will do typically outside in so that means if we write something that looks like an acceptance test, something that describes very closely what this person with the persona experiences both in what they do, and what they see back from the software.  So we write the starts, we start with a very low fidelity version of that. It's not gonna describe the whole interaction, it might describe the smallest piece of we could possibly articulate; so we start there.  And then we run that test and it fails.  Then we say, \quotes{hmmm, ok, so we're in this architecture, what's the next part that we need to build in order to begin to meet those needs?}  And we work our way down the architecture.  In a typical web app, there's something that's displaying, an HTML page, and then there's something that is probably orchestrating the generation of that HTML, like a controller and usually we try and separate our concerns. We think about these things as we work our way down driven by trying to meet just this one acceptance test.  0:11:26

\textbf{Interviewee:} So along the way what we do is we write individual unit test for the components that have interesting behavior.  The controller does have interesting behavior.  It takes in some input and it makes some decision about what should be the output, what should be the resulting HTML.  And that controller also interacts with other collaborators.  So we have tests that say, are you properly handing off these parameters?  So these really fine-grain unit tests. But the key is that these things are, each time we write these tests, we're setting an expectation on that little tiny piece of the system, in the same way that our acceptance criteria are setting an expectation on the software, and our user stories are setting expectations on the feature .  So forth, all the way back up. We're trying to be needs-based  all the way down as we do this; that's probably good enough.  The details of how that happens varies wildly and even like within this, there are different schools of thought about how that happens.  There's people who believe in writing units for every little thing and people who say \quotes{no, you can set certain bullwarks} and write tests around the bullwarks.  Let everything sort of float in between.  0:12:43
\section{Interview 7 Abridged Transcript with Product Manager }

\begin{figure}[h]
\centering
\includegraphics[width=6.5in]{interviews/drawings/2015_08_12_pm.png}
\caption{\quotes{2015-08-12 Product Manager's drawing of software development process}}
\label{2015_08_12_pm}
\end{figure}


\textbf{Todd:} If you're open to it, could you, on that sheet of paper, draw out how you view the way we build software? This is completely open-ended. There's no wrong answers. And just take a stab at it.  0:00:21

\textbf{Interviewee:} So I'm drawing a line, it's like product on a continuum. We're gonna have vision here. We're gonna have working. Can you tell I'm a PM because I'm writing in lines, boxes can't do? 0:00:45

\textbf{Todd:} I love it.  0:00:46

\textbf{Interviewee:} So let's see here. So let's do a couple of lines here. Alright, so let's say, each of these represents two weeks.  So we'll say that the first point, so how we build it at pivotal is that what you asked?  0:01:17

\textbf{Todd:} Yes.  0:01:18

\textbf{Interviewee:} So it starts with having clients.  0:01:20

\textbf{Todd:} How would you build it too?  0:01:21

\textbf{Interviewee:} Yeah, yeah. Totally. The reason I asked about Pivotal is 'cause we're doing consulting. So if you're thinking of product as a continuum, our clients come in in all these different ways of where they the insert. But it starts with a conversation, kind of like a qualifying conversation of like, \quotes{hey, what do you wanna build?}, and \quotes{what's the fidelity of your idea?} And from there, we have a good understanding of if they're at a stage where they can build something, or if they maybe really need to think about it further.  But once we realize ok, they're ready for Pivotal , we'll start with a Discovery and Framing and not every project gets Discovery and Framings, but in the LA office, a lot of them do.  We're trying to get most projects getting some a semblance of Discovery and Framing.   0:02:04

\textbf{Todd:} When would you not do one?  0:02:07

\textbf{Interviewee:} So we won't do one... I think we can make the case really for all projects could do with a little bit of DNF right? So even if you're like, I know where to build from. So with FYI rather, but for a PM, I'm thinking, okay my role is to help the client understand who are we building for and then what we are building and when? So for the, who are we building for, I think all clients.  It'd be great if we could do some user research with them even if they are like, we did all these user research just to do a quick gut check, like hey this makes sense, that'll be great.   0:02:42

\textbf{Interviewee:} But I think a lot of the times when we don't do them, it tends to be convincing the client or if there's a budget concern.  So they have enough of a fidelity of understanding who their target user is and who the secondary users are and have a persona and are focused on stuff like kinda key factors, ok, if you're saying you want to build an app for the millennials and for college-age students and for the ages 25-35, that's pretty vague.  We do get that a lot, but we're saying, \quotes{ok, it's for millennials, so what gender are we going for?} Or is it, \quotes{what are their behaviors like?}  There's a lot of questions that we can ask on this conversation. and these qualifying calls to say, to kind of fish out, to think if they are ready to work with Pivotal, or if they have enough information for DNF or not. So, does that answer?  I guess that was a very specific answer.  0:03:44

\textbf{Todd:} That was great.  0:03:46

\textbf{Todd:} I interrupted your flow.  0:03:48

\textbf{Interviewee:} No, no, this is fine.  So I'm of the opinion that I would love it if we could have some sort of research with all of our clients and users. Typically when we do a DNF, Discovery and Framing, we do that for 4 weeks on average, sometimes it's up to 6 weeks depending on how many users they have and how many people they want us to focus on.  But we typically really try to work on 2 to 3 target users being 1 primary user and maybe 2 users of the system that kind of insert in their day.  And then we've done a 2 week of design first or Discovery and Framing but really it's just \quotes{let's talk to some users and validate some ideas.} But the whole goal of this Discovery and Framing process is to do a couple of weeks of talking to users, of about 2 weeks and that's when we're doing some of those exploratory interviews, kinda turn it into elicit narratives to understand what their behaviors and what their days are like.   0:04:51

\textbf{Interviewee:} And then from there, we can isolate what are some of their pain points and what are some of the frictions and inefficiencies and how are they capturing data and what are the tools that they use and who are the people they're talking to.  From there, I guess one thing I didn't mention which is important is to say, what are the product goals that our clients have?  And what are some of the assumptions that they have about their products?  About their users where their products can help solve their needs.  We go into user research, alright, here are some assumptions, hypothesis that we have, let's test them. Out of that output of user research is that we have this, we do some synthesis and analysis of all of the things they're talking about.  We record the things that we saw, so if they're in a cubicle and they have tons of printed out papers because what they do is they get stuff by mail and then they have to scan it in.  Those are the things that we're seeing that are part of their day that can affect them.   0:05:54

\textbf{Interviewee:} What did we hear, like things that they're telling us about their day, and things like we felt that they're telling us. But maybe there are some subtext and some nuance there saying everything's great but their faces are really strained and you can tell they're really frustrated and their posture has changed when they're talking about certain subjects. Kind of taking all of the things that recording from our user sessions and then coding it by what we saw, what we heard, what we felt and then finding themes. What are users talking about? Usually when we do research, we try to do 3 to 5 users, so that way we have a good cross-sample and in case there's any extreme people that we meet, it kind of helps us give a better analysis, better data sample. So we'll kind of call all the information that we have and we'll go to them, too.  So we'll go to their cubicles or wherever their workspaces are. So we want to get a sense of their environment. Cause there's so much contextual information there that you can't get just from having a phone call with someone.  0:07:02

\textbf{Todd:} Yes.  0:07:03

\textbf{Interviewee:} So, then, that's usually about a couple of weeks doing some researching.  As we're doing the researching, we're capturing all of our information on notepads and then we're doing what we call infinity mapping, affinity mapping.  We'll get one of those big foam cork white boards and we'll take a listen to the recordings of when we're doing user research or if we can't record just looking at our notes because our notes are like, you're almost writing verbatim what people are saying. Because you don't want to put all your analysis in there at this point, you just want to get them to talk and get it all in, and then we'll take each kind of idea and we'll put it on a post-it note and we'll have a bunch of post-it notes around and they'll be coded by what we saw, what we heard, what we felt, and then we'll start seeing themes around this post-it notes.   0:07:54

\textbf{Interviewee:} So there's this one user that said there is a lot of things around education and tools and timeline and whatever it is.  Then we started noticing trends, so we'll start taking those nuggets and we put them across this themes and then we'll do that to all our users; and then from there, we'll do another round of synthesis and the ideas are going to keep condensing.  So then we have a synthesis of all the people we talk to and what the overlaps are and the themes of what their day is like, and the behaviors they drawn during that day.  And then after that, we'll map out their day, like what are the tasks that they're doing, and then map out from the tasks all those insights rather not insights but nuggets of what we heard from them, that we kind of collectively called down and condensed and we'll put those against the tasks.    0:08:43

\textbf{Interviewee:} So when they need to schedule a user for an event, these are all the things they said about it, or these are the things we saw and the things that we felt.  So then what's cool about that is we do that for every target user and you can map them out.  So you say 'here are three target users, here are their days and here are all the intersections of their days'.   So you can tell visually, \quotes{oh, you know what, when this person does something, you notice the next thing, these two people are affected.} And then you can start isolating pain points and inefficiencies and you get these really nice overlay of what the system, not the digital system but just the users in the workplace and what their days look like.  So when I'm mapping out the process, I'm thinking of conversations, that's a little talk bubble.  0:09:43

\textbf{Todd:} Ok, I like it.  0:09:44

\textbf{Interviewee:} And then I'm having doing some synthesis. Someone do a little post-it map.  0:09:51

\textbf{Todd:} Ooh, I see the post-its.  0:09:53

\textbf{Interviewee:} There you go.  And then from that, we say, \quotes{ok, based on our research, this is what we say a persona or what this user looks like.}  We think of \quotes{what do they need? What are the tasks in their day and what do we need.} So we pull out insights from that. This user really has trouble communicating with the other people on this team because they don't have the right communication tools setup or whatever that is. They need a better communication system.  Once we have these needs, and bits and insights of who they are and what they need, then we can say, \quotes{all right, how can our products solve these needs?} Then we say, \quotes{we have some product ideas based on what their needs are, let's validate them.}   0:10:55

\textbf{Interviewee:} Then we'll go back and talk to the users. We'll drop some wireframes and say, \quotes{all right, based on what you guys had said, we feel that here's a quick prototype, clickable prototype} And we'll use invision.  We'll say, \quotes{Why don't you click around? What do you think about these things?} we'll do some user testing there.  And then from that information, we'll further validate or dis-validate our product ideas and we'll do another product evolution but kind of the output of this 4 week on average DNF cycle as that you'll have Wireframes and then you'll have some personas. You'll know who your end user is.  You'll have empathy drawn for your end user which is the whole goal. You'll have a problem that's been framed and validated.   0:11:38

\textbf{Interviewee:} So that way, it really de-risks development cause it's pretty great to come in to development and we know that when we're making product decisions, we can go back to this research.  We're like, \quotes{oh, if we're going to do this or this, then what do they actually need? What was that they talked about that really indicated this is the right approach?} It also helps us speak a language that our users speak.  Which is really important for the development team but all the other stakeholders involved in the process.  0:12:05

\textbf{Interviewee:} Words are everything, right? So we want people to be kind of on the same communication levels of talking how their users would talk, so that way it helps us draw all that empathy throughout the entire development process because you still need to draw on that, you know 6 weeks, 3 months, however long into the development cycle until you're releasing. Right to be able to have that insight of what they want. I think the cycle is... Let's just say that you've done some analysis and you've done some wireframing, and then after that, we'll talk to users again.  After that, we'll do another set of wireframes. We'll also do some persona mapping here as well as making these wireframes.  And then after we do that, we will create a feature list.  0:13:13

\textbf{Todd:} Yes.  0:13:14

\textbf{Interviewee:} We'll say,  we're not gonna\ldots  What could the next you know 2 to 3 kind of epic areas. We'll say, let's give some insights here.  Maybe, we have 3 insights and 3 needs.  Let's do 3 feature ideas and how do these feature ideas map these needs?  So we write these feature ideas and we'll say, down in the documents and user needs this and we'll help them achieve their goals.  Business needs this and this will help them achieve their goals.  We're always aligning user business needs throughout the entire process. So then when we get into development, we'll just make a little terminal, I wish I had multiple colors.  0:14:00

\textbf{Todd:} Next time, you'll have to bring your can.  0:14:03

\textbf{Interviewee:} So, when you're getting into the computer, when you're getting into development, you're able to have a backlog that's been built.  You have the first couple of weeks of features.   0:14:17

\textbf{Todd:} Yes.  0:14:18

\textbf{Interviewee:} You have some ideas and you start building and then once you release something, which could be in a week or could be a couple of weeks depending on what you're doing.  But once you get into the first bit of business value working software, then you can go and you could talk to your users again and then you learn from them and then you continue on.  So at that point, what you're doing is you're building something and then you're measuring it.    0:14:45

\textbf{Todd:} Yes.   0:14:46

\textbf{Interviewee:} And then you're learning from it right?   0:14:48

\textbf{Todd:} Uh-huh. Yes.  0:14:50

\textbf{Interviewee:} And then you're going back to building, right? So that's what we're doing for this whole process. So the feedback loop is really important. When I'm thinking about the extent, the depth of all this research, then you don't need to have a DNF to do any of these research.  You don't have to sell that in, you can do...  So the training that our designers and our PMs have, and now we're trying to have our developers be exposed to it, as well.  To say, \quotes{ok, developer, you get a client project. Development starts tomorrow, you can still talk to some users.  You can still setup user testing.}  And you can do that throughout the entire process.  So it's almost like this whole upfront part, with the DNF.  You're making little DNFs through it, right?  So whoops, it's really you're taking this and you're kind of doing a DNF cycle.   0:15:42

\textbf{Todd:} Yeah.   0:15:43

\textbf{Interviewee:} Whoops, and you're doing that here, and then you're going to do it again. So each week, you might be bi-weekly.   And that is ideal, right?  But sometimes you don't get the users?  So there's a lot of, kind of…  You have to be pretty scrappy with how you do this sometimes, you don't just get this nice set of users at your disposal, you know; but it's so important and I think that's something we do a good job with is convincing our clients, we wanna de-risk this for you so this is how we can do it.  I mean go, it makes sense.  It's very practical stuff.  It's not rocket science honestly.   0:16:20

\section{Interview 15 Abridged Transcript with Interaction Designer }

\begin{figure}[h]
\centering
\includegraphics[width=6.5in]{interviews/drawings/2016_01_08.png}
\caption{\quotes{2016-01-08 Interaction Designer's drawing of software development process}}
\label{2016_01_08}
\end{figure}

\textbf{Todd:} My first question for you is very open ended.  00:02

\textbf{Interviewee:} Okay.  00:03

\textbf{Todd:} There is no wrong answer.  I was hoping if you could draw how you feel about the product.  00:09

\textbf{Interviewee:} Okay.  This may get elaborate.  00:27

\textbf{Todd:} Fantastic.  00:29

\textbf{Interviewee:} This is my new pen so.  [Pause] Here we are.  I did sort of a story board-ish type of thing.  02:54

\textbf{Todd:} I love it.  02:55

\textbf{Interviewee:} So, do you want me to explain it?  02:58

\textbf{Todd:} Please.  02:58

\textbf{Interviewee:} Okay.  On one hand superficially, I'm happy because aesthetically, I think it's nice.  I think it's, when you compare it to some of the projects we do, it's been going on for so long and there's such a huge team of smart people.  I feel like we got so much done and it's complex and interesting and there's lot of thought that went into it and it's a really robust app but I'm worried that even though it's pretty and we built a lot of features and the technology is cool, not all of it is necessarily useful for end users and I didn't even give the user name anything because I don't even know that we're designing for the right person all the time with some of our features.  I don't know.  So, I'm worried there were will be more confusion in the marketplace than I would like there to be in a product that I've worked on.  03:57

\textbf{Todd:} Anything else?  04:01

\textbf{Interviewee:} In the drawing or in general?  04:05

\textbf{Todd:} Both.  04:06

\textbf{Interviewee:} Not so much in the drawing.  I guess the big question mark is just that I don't know, I think it's pretty usable but I'm worried there's going to be features for the users or going to be like what is this or why do I care or they'll have question marks around there's something really obvious to me like \ldots I would like you heard a lot of feedback, I want a light telling me if my oil is low and that's just not something we could do because of constraints on the technology I think.  So, I'm worried if people will look at it and say why is there all this stuff that I don't want and there's some stuff that maybe feels really obvious to some users that we haven't provided for one reason or another but overall, I still feel happy.  I think we created a solid product.  04:59

\textbf{Todd:} Good.  When you were describing your \quotes{you} picture, you used the word \quotes{superficially}.  I don't remember the exact word but something like given this superficially I feel happy about it and I was curious if there was like an under feeling of the product.  05:17

\textbf{Interviewee:} Yeah, currently to some degree, this is the under feeling.  The superficial part is a little bit as a designer is a normal person walking around, you feel like people look at it and like it's so beautiful and that might be the beginning and the end of what they think of the app.  They might not use it.  Maybe, it's not for them.  I feel like I could put it in a portfolio or take some of those App Store screens and show it to people and maybe like oh my God, this is the nicest product, you must have done a great job or your team must have worked really hard but if we built something that's really beautiful but doesn't meet the needs of our users, it's kind of I'm still superficial.  I guess part of me is still happy it's beautiful at least or that there's parts of it that are really pretty but at the end of the day as a designer, it's kind of a big fail to build something that's pretty but not the right thing.  It should be a big fail for everybody but especially as the designer, that's what you want to avoid.  06:21
\section{Draw your view of Pivotal's software development process }

\begin{figure}[H]
\centering
\includegraphics[width=6.5in]{interviews/drawings/2015_06_02.png}
\caption{\quotes{Interview 2: Product Manager's drawing of software development process}}
\end{figure}

\begin{figure}[H]
\centering
\includegraphics[width=6.5in]{interviews/drawings/2015_06_29a.png}
\caption{\quotes{Interview 3: Product Manager's drawing of software project workflow}}
\end{figure}

\textbf{Interviewee 3:} I try to first come up with some type of hypothesis that I want to test. Then, I'm going to build something that's going to test this, then I'm going to try to get a result. 

% \begin{figure}[H]
% \centering
% \includegraphics[width=6.5in]{interviews/drawings/2015_06_29b.png}
% \caption{\quotes{Interview 3:  Product Manager's drawing of software development process}}
% \end{figure}

\begin{figure}[H]
\centering
\includegraphics[width=6.5in]{interviews/drawings/2015_07_31a.png}
\caption{\quotes{Interview 5: Product Manager's drawing of software development process}}
\end{figure}

\begin{figure}[H]
\centering
\includegraphics[width=6.5in]{interviews/drawings/2015_07_31b.png}
\caption{\quotes{Interview 5: Product Manager's drawing of software development process}}
\end{figure}

\begin{figure}[H]
\centering
\includegraphics[width=6.5in]{interviews/drawings/2015_08_12_se.png}
\caption{\quotes{Interview 8: Software Engineer's drawing of software development process}}
\end{figure}

\textbf{Interviewee 8:} We iterate on stuff that we have a touch point and we're going to go away and do work and come back to that touch point. To me, it sort of looks more like a spring, you pull the slinky or you have a pig's tail that you stretched out. We are trying to do is orbit this idea of always working with each other. The red, green, refactor is a very tiny cycle that we do. We're trying to setup ourselves up, a check-in point where we can start somewhere and move a little bit and make sure we're okay and come back again and have another starting place and you can do that in 5 minutes with a test. Then you see us play that out at higher level of stories and then higher level again with our daily stand-ups, and higher level again with our retros and our check-ins there and higher level again with our inceptions. It's really about managing the feedback cycle or the check-in points to make sure that we're kind of all fluidly communicating about how we're affecting and doing things. Because the minute we stop communicating, is the minute we start to get off and be in the weeds somewhere.



\begin{figure}[H]
\centering
\includegraphics[width=6.5in]{interviews/drawings/2015_09_02.png}
\caption{\quotes{Interview 9: Software Engineer's drawing of project workflow}}
\end{figure}


% \begin{figure}[H]
% \centering
% \includegraphics[width=6.5in]{interviews/drawings/2015_12_18a.png}
% \caption{\quotes{Software Engineer's drawing of software development?'}}
% \end{figure}

\begin{figure}[H]
\centering
\includegraphics[width=5.5in]{interviews/drawings/2016_01_15.png}
\caption{\quotes{Interview 18: former software engineer's drawing for the Pivotal software development approach}}
\end{figure}

\textbf{Interviewee 18:} [The circles are arranged from an inner core practices to outer rings that are more negotiable.] These are the things I think are core to how Pivotal Labs does software development from the most important.  Basically, if some client wanted to drop things outside of it, this will be the order that I think we'd agree to have it be dropped.  I don't think we'd ever drop retros but I think we might drop the backlog or something that we probably still would.

\section{Draw how you feel about your current product?}

\begin{figure}[H]
\centering
\includegraphics[width=6.5in]{interviews/drawings/2016_01_08_designer2.png}
\caption{\quotes{Interview 16: interaction designer's drawing for `how you feel about your current product?'}}
\end{figure}

\textbf{Interviewee 16:} The peak  was very like a successful moment for Pivotal. It's really cool to see how many people appreciated what we've done. Against all odds, we released this app. I felt proud to be on that team. \ldots 

[The graph goes down] as the design started getting cut. The only thing we're doing is bugs. We designed all these cool features that we're excited about. Almost none of them are getting into the app. It's been almost every week that a feature in the new design is no longer in [the release]. We don't need [what I worked on] anymore.  It feels very frustrating as a designer. I can't affect that that much.

\begin{figure}[H]
\centering
\includegraphics[width=6.5in]{interviews/drawings/2016_02_25.png}
\caption{\quotes{Interview 21: Product Manager's drawing for `how you feel about your current product'}}
\end{figure}

\section{Draw how you feel about the code}
\label{AppendixFeelAboutTheCode}

\begin{figure}[H]
\centering
\includegraphics[width=6.5in]{interviews/drawings/2015_12_18b.png}
\caption{\quotes{Interview 14: Software Engineer's drawing for `how do you picture the code?'}}
\end{figure}


\begin{figure}[H]
\centering
\includegraphics[width=6.5in]{interviews/drawings/2015_12_03.png}
\caption{\quotes{Interview 10 Software Engineer's drawing for `how you think of the code on this project?'}}
\end{figure}


\begin{figure}[H]
\centering
\includegraphics[width=6.5in]{interviews/drawings/2015_12_08.png}
\caption{\quotes{Interview 11: Software Engineer's drawing for `how you think about the code on this project?'}}
\end{figure}

\textbf{Interviewee 11:} It's kind of complicated. We have so many different pieces and this is what the connection feels like to me in my head where everything is jumbled together but at the same time, I do feel like the structure is clean but I'm having a hard time thinking of what to draw for like what represents clean. .  I'm just going to draw a water droplet to show that it looks clean.

It's more than just it's complicated, it's expanding a lot so in my head.  I'm thinking more like a big bang kind of thing where it starts out very small and now it just keeps expanding and growing and then we're adding a bunch of features. 

It's solid and clean. I feel like the project is very well tested so there's less chances of major breakings. It is solid. So I draw a rock cube.

Our codebase is pretty young, pretty flexible in ways that when we want to do refactors, it's not super complicated and not super hard to do. We are pretty good at like separating all the logic. It was really easy to do refactoring on the stuff that we want to do.  So, I guess it's pretty flexible of something flexible. So I draws a rubber band.

\begin{figure}[H]
\centering
\includegraphics[width=6.5in]{interviews/drawings/2015_12_10.png}
\caption{\quotes{Interview 13: Software Engineer's drawing for `how you think about the code on this project?'}}
\end{figure}


\begin{figure}[H]
\centering
\includegraphics[width=6.5in]{interviews/drawings/2016_01_14.png}
\caption{\quotes{Interview 17: Software Engineer's drawing for `How you feel or how you think about the code'}}
\end{figure}


\begin{figure}[H]
\centering
\includegraphics[width=6.5in]{interviews/drawings/2016_07_05a.png}
\caption{\quotes{Interview 25: Interaction Designer's drawing for `how you feel about your current product'}}
\end{figure}

\begin{figure}[H]
\centering
\includegraphics[width=6.5in]{interviews/drawings/2016_07_05b.png}
\caption{\quotes{Interview 25: Interaction Designer's drawing for `how you feel about your current product'}}
\end{figure}


\begin{figure}[H]
\centering
\includegraphics[width=6.5in]{interviews/drawings/2016_07_05_designer2.png}
\caption{\quotes{Interview 26: Interaction Designer's drawing for `How you feel or how you think about the code'}}
\end{figure}


\begin{figure}[H]
\centering
\includegraphics[width=6.5in]{interviews/drawings/2016_08_17.png}
\caption{\quotes{Interview 28: Software Engineer's drawing for `How you feel or how you think about the code'}}
\end{figure}

\begin{figure}[H]
\centering
\includegraphics[width=6.5in]{interviews/drawings/2016_08_18.png}
\caption{\quotes{Interview 29: Software Engineer's drawing for `How you feel or how you think about the code'}}
\end{figure}

\begin{figure}[H]
\centering
\includegraphics[width=6.5in]{interviews/drawings/2016_09_26.png}
\caption{\quotes{Interview 31: Software Engineer's drawing for `How you feel or how you think about the code'}}
\end{figure}

\begin{figure}[H]
\centering
\includegraphics[width=6.5in]{interviews/drawings/2016_09_26_engineer2.png}
\caption{\quotes{Interview 32: Software Engineer's drawing for `How you feel or how you think about the code'}}
\end{figure}


\begin{figure}[H]
\centering
\includegraphics[width=5.6in]{interviews/drawings/2016_09_29.png}
\caption{\quotes{Interview 33: Software Engineer's drawing for `How you feel or how you think about the code'}}
\end{figure}




% \chapter{Interview Transcriptions}
\section{Appendix Section}
Test in main
\section*{Appendix}
\chapter{Interview Transcriptions}
This is a test
\input{interviews/interview_2015_05_29}



\bibliographystyle{IEEEtran}
\bibliography{bibliography}

\backmatter


\end{document}