% Sample apostrophy's to remove team's 

\chapter{Conclusions}
\label{ConclusionChapter}
\section{Summary}
Using Grounded Theory, we examined the question, \quotes{what is happening at Pivotal when it comes to software development?} Listening to the concerns of the participants, the emergent research lead to three core concerns: 1) how do the observed Pivotal teams effectively construct software in a sustainable manner? 2) how do the observed teams engender team code ownership? and 3) what kinds of waste do the observed teams identify and remove in its journey of continuous improvement? 

\subsection{How does Pivotal construct software?}
In exploring the construction of software, Grounded Theory guided us to the theory of Sustainable Software Development, a set of principles, policies, and practices used in the construction of software products so that teams survive team disruptions such as team churn. The principles include engendering a positive attitude toward team disruption, encouraging knowledge sharing and continuity, and caring about code quality. The policies are team code ownership, shared schedule, and avoid technical debt. The removing knowledge silos practices are continuous pair programming, overlapping pair rotation, and knowledge pollination. The caretaking the code practices are test-driven development / behavior-driven development and continuous refactoring which are supported by live on master.

Conventional wisdom says that team disruptions should be avoided, and that extensive documentation is needed to prevent knowledge loss during team churn. Unfortunately, documentation often quickly becomes out-of-date and unreliable. The theory positions team code ownership with overlapping pair rotation and knowledge pollination as an alternative and potentially more effective strategy to mitigate against knowledge loss.

The primary benefits to the software developer are the ability to understand the entire system, the ability to work on every story, increased in teaching opportunities to share one's expertise, and more nuanced understanding of the utilized technologies.
The primary benefit to the employer is business agility. The engineering team continues to deliver software week after week, month after month, while surviving cataclysmic events. Things do not fall apart when the superstar developer leaves because features or components are not critically tied to a particular individual. Critical feature work can be parallelized since anyone can work on any feature. The whole team's talents are leveraged.

The theory is rooted in team code ownership, as removing knowledge silos and caretaking the code enable teams to be able to modify any part of the code base. 

\subsection{How does Pivotal engender team code ownership?}
In researching how Pivotal engenders team code ownership, the observations clearly indicate that team code ownership is a feeling to be engendered not a policy to be decreed.

Meanwhile, both discussions with and observations of participants suggest five factors associated with strong feelings of team code ownership. Pivotal developers more acutely feel team code ownership when i) they understand the system context; ii) they have contributed to the code in question; iii) they perceive code quality as high; iv) they believe the product will satisfy user needs; and v) they perceive team cohesion as high.

Moreover, diverse events and trends can undermine sense of ownership, including: increasing knowledge silos, increasing code base size, increasing team size, inability to contribute, pressure to deliver and deprioritizing continuous refactoring, ignoring user feedback, ignoring developer feedback, and distancing a developer from the team.

In reviewing the data, we discovered that transitioning from individual code ownership to team code ownership is not easy for some engineers as psychological ownership fulfills deep psychological needs.

\subsection{What kinds of waste does Pivotal identify and remove in its journey of continuous improvement? }

In investigating removing waste, we created a waste taxonomy that identifies the type of waste the participants detect and try to remove in the studied organization. The wastes are building the wrong product or feature, mismanaging the backlog, unnecessary complexity, rework, unnecessary cognitive effort, cognitive hindrance, waiting, and ineffective communication. 
Contrary to the Lean Software Development's taxonomy of wastes, which is top-down and created by mapping manufacturing wastes to software wastes, our taxonomy is bottom-up as it is grounded in empirical data. The comparison of the two models shows some alignment. However, our taxonomy expands the Lean Software Development's taxonomy by broadening the definition of most wastes and introducing additional wastes not previously identified. As such, our taxonomy is more expressive and more accurately describes the observed data.

As a model, the waste taxonomy succinctly describes waste types and is more descriptive than previous work. 

\section{Future Research}
Given the emergence of the theory of Sustainable Software Development, a model of team code ownership, a waste taxonomy, and a deeper understanding of the evolution of Extreme Programming, there are several opportunities for future research.

The theory shows how the principles, policies and practices work together to achieve the business goal of sustainability. It would be interesting to understand how variations of the principles, policies and practices might influence the overall process. For example, what would be the impact of removing or altering the shared schedule? What would be the impact of not doing test driven development? Additional experimentation can be done to assess each practice's flexibility while still maintaining the ability for a team to survive disruptions. 


We are interested in the tension between individual and team ownership, as well as the factors that foster and decrease the sense of ownership. Developers, interaction designers, and product managers all have different goals for their role. Future work could examine how the sense of ownership is driven by different factors for each role. Some programmers naturally adapt to team code ownership, while others struggle with the transition. It would be interesting to follow new Pivotal engineers and examine their journey in transitioning from individual code ownership to team code ownership. Perhaps there are specific practices that Pivotal or the development team could employ to ease the transition. We could also investigate the optimal team size for team code ownership, or explore whether Sustained Software Development works for a distributed team with a Shared Schedule.

At Pivotal, the waste taxonomy could be used to systematically examine potential wastes. Since some wastes are rarely salient unless one is specifically looking for them, future research could use the waste taxonomy to see how helpful it is in identifying these waste on a project. We could investigate how Pivotal relies on feedback loops as a mechanism for identifying, dealing with, and reducing waste. Also, the waste taxonomy could be verified at additional companies using different development methodologies. It is possible that a team following scrum or waterfall might generate waste not encountered on the observed teams. 

Pivotal has evolved the Extreme Programming practices for building and managing a backlog. Instead of committing to work to be done each week, teams work off the top of the backlog in an iteration-less flow. Additional research could examine how the evolution of the Extreme Programming backlog practices affects software development.

It would be interesting to validate the emergent theories at other organizations and compare the results, as this would lead to a deeper understanding of software development. Now that extensive field work and Grounded Theory has produced these theories, additional research could validate each theory in a very different context.
\section{Conclusion}
Pivotal's continuous improvement practices have resulted in an evolution of Extreme Programming.  Each project becomes an experimentation opportunity for the engineers to try new ideas and see the results. As compared to a company with a relatively stable product line, Pivotal Labs has a unique opportunity for continued experimentation as it undertakes scores of projects each year. One Pivotal engineer reflected that \participantQuote{each project feels like a hill climbing optimization algorithm, exploring the feature set of the product. Each week, stories guide the product towards a delightful user experience.} \todo{Fundamentally, the teams are on the hill of Extreme Programming. It is unlikely that the teams will accidentally discover a new hill, yet} With each new project, the teams have an opportunity to refine Extreme Programming in ways not envisioned by its creator. 
