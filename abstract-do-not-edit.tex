% Sample apostrophy's to remove team's 

\newpage
\section*{Abstract}
\textit{Context:} Software development is a complex socio-technical endeavor that involves coordinating different disciplines and skill sets. Practitioners experiment with and adopt processes and practices with a goal of making their work more effective.


\textit{Objective:} The purpose of this research is to observe, describe, and analyze software development processes and practices in an industrial setting. Our goal is to generate a descriptive theory of software engineering development, which is rooted in empirical data.


\textit{Method:} Following Constructivist Grounded Theory, we conducted a two-year five-month participant-observation of \numberOfObservedProjects{} software development projects at Pivotal, a software development company. We interviewed \numberOfInterviews{} software engineers, interaction designers, and product managers, and analyzed one year of retrospection topics. We iterated between data collection, data analysis and theoretical sampling until achieving theoretical saturation and generating a descriptive theory.


\textit{Results:} 1) This research introduces a descriptive theory of Sustainable Software Development. The theory encompasses principles, policies, and practices aiming at removing knowledge silos and improving code quality (including discoverability and readability), hence leading to development sustainability. 2) At the heart of Sustainable Software Development is team code ownership. This research widens the current definition and understanding of team code ownership. It identifies five factors that affect ownership. Developers achieve higher team code ownership when they understand the system context, have contributed to the code in question, perceive code quality as high, believe the product will satisfy the user needs, and perceive high team cohesion. 3) This research introduces the first evidence-based waste taxonomy, identifying eight wastes along with causes and tensions within wastes. It also provides a comparison with the taxonomy of wastes found in Lean Software Development.


\textit{Limitations:} While the results are highly relevant to the observed company, Pivotal, the outcomes might not apply to organizations with different software development cultures.


\textit{Conclusion:} The Sustainable Software Development theory refines and extends our understanding of Extreme Programming by adding new principles, policies, and practices (including Overlapping Pair Rotation) and aligning them with the business goal of sustainability. One key aspect of the theory is team code ownership, which is rooted in numerous cognitive, emotional, contextual and technical factors and cannot be achieved simply by policy. Another key dimension is waste identification and elimination, which has led to a new taxonomy of waste. Comparing this taxonomy to Lean Software Development's list of wastes revealed our taxonomy's parsimony and expressiveness while illustrating wastes not covered by previous work. Overall, this research contributes to the field of software engineering by providing new insights, rooted in empirical data, into how a software organization leverages and extends Extreme Programming to achieve software sustainability.
